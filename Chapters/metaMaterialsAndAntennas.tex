\chapter{MetaMaterials and Antennas}
\authortoc{John Schultz and James G. Maloney}
\chapterauthor{John Schultz and James G. Maloney}

\section{Introduction To MetaMaterials}

Bla Bla BLA

\section{Slicing}
\label{sec:slicing}	% specifying a section label to be used by a \ref
In the Yee algorithm, both space and time are discretized, with the increments in space for rectangular coordinates being $\Delta x, \Delta y, \Delta z$  and the increment in time being $\Delta t$ \cite{Yee66}-\cite{Maloney1}.

\begin{figure}
\includegraphics[angle=0,width=\linewidth]{/Users/jim.maloney/Book/images/AntChapOrigFig1.png}
\caption{Schematic drawing showing the computational volume, FDTD spatial lattice, and unit cell.}
\label{fig:ACF1}		% figure references have to be below the \caption line
\end{figure}

Figure \ref{fig:ACF1} is a schematic drawing showing a typical volume in which Maxwell's equations are to be solved. The volume is divided into unit cells each of volume. 
The partial derivatives in Maxwell's equations are approximated by ratios of differences, as shown in \ref{eq:exFirstDiff}.

\begin{equation}
\label{eq:exFirstDiff}
\frac{\partial E_x}{\partial z} \approx \frac{\Delta E_x}{\Delta z}, \quad
\frac{\partial H_y}{\partial t} \approx \frac{\Delta H_y}{\Delta t}
\end{equation}

\subsection{examples}

This section contains some example designs based on Figure \ref{fig:ACF1}

\section{Dicing}

In section \ref{sec:slicing}, slicing was introduced.   In this section we will discuss dicing the material in a cubical fashion.

\begin{thebibliography}{99}

 \bibitem{Yee66} K. S. Yee, ``Numerical Solution of Initial Boundary Value Problems Involving Maxwell's Equations in Isotropic Media,'' IEEE Trans. Antennas Propagat., Vol. AP-14, pp. 302-307, May 1966.

\bibitem{Maloney1} J. G. Maloney, G. S. Smith, and W. R. Scott, Jr., ``Accurate Computation of the Radiation from Simple Antennas Using the Finite-Difference Time-Domain Method,'' IEEE Trans. Antennas Propagat., Vol. AP-38, pp. 1059-1068, July 1990.

\bibitem{BoonPist} J. J. Boonzaaier and C. W. Pistorius, ``Thin Wire Dipoles ? A Finite-Difference Time-Domain Approach,'' Electronics Lett., Vol. 26, pp. 1891-1892, 25 October, 1990.

\bibitem{KatzHorn} D. S. Katz, M. J. Picket-May, A. Taflove, and K. R. Umashankar, ``FDTD Analysis of Electromagnetic Wave Radiation from Systems Containing Horn Antennas,'' IEEE Trans. Antennas Propagat., Vol. AP-39, pp. 1203-1212, August 1991.

\end{thebibliography}


