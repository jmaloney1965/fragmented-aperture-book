\chapter{Reconfigurable Fragmented Aperture Antennas}

\section{Introduction}

In the preceding chapters, we have shown how the fragmented aperture concept can be used to design antennas that meet particular performance specifications.  A planar sheet of conductor is divided into many sub-wavelength pixels, and a genetic algorithm working with an FDTD simulation determines which pixels should be conducting and which should not.  Different designs can be obtained to meet different specifications: for example, one design might produce an antenna with a broadside beam optimized for a particular bandwidth, while a second design might produce an antenna with the beam steered to $45^o$ from broadside.

Of course, once a fragmented aperture antenna is fabricated, it can only meet one set of specifications---either the broadside design or the steered design, but not both.  It would be enormously useful if a single fragmented aperture could be electronically switched between different configurations to meet different requirements on demand.  This would require a mechanism for dynamically changing individual pixels from conducting to non-conducting and vice versa.  Recent work has demonstrated various reconfigurable pixelated antenna implementations using MEMS switches \cite{Wright2018Mems,Ali2014Mems}, phase transition materials \cite{Lou2025Frequency}, and magnetically actuated mechanisms \cite{Pal2018Magnetically}.

One can imagine, for example, making the cladding on a circuit board from a photoconductive material and using a laser to selectively illuminate the pixels that need to be conducting for a particular design.  Changing the illumination pattern would reconfigure the antenna.  While this particular approach remains impractical with current technology, the underlying idea---a reconfigurable fragmented aperture---motivated the development of the Agile Aperture Antenna described in this chapter.

\section{The Agile Aperture Antenna Concept}

The concept of a reconfigurable fragmented aperture antenna was first published in 2000 under the name ``switched fragmented aperture antenna'' \cite{MaloneyRECAP}.  Subsequently, DARPA funded a solicitation for ``reconfigurable aperture'' antennas and coined the acronym RECAP.  To distinguish the fragmented aperture approach from other reconfigurable antenna concepts, we later adopted the term ``Agile Aperture Antenna'' (A3), emphasizing that the purpose of reconfiguration is to make the antenna \emph{agile}---able to dynamically change its frequency of operation, beam direction, polarization, or other characteristics.  The seminal IEEE paper on this work \cite{Pringle2004Reconfigurable} and a comprehensive book chapter \cite{Balanis2007ReconfigChapter} provide detailed treatments of reconfigurable aperture antenna technology.

The Agile Aperture Antenna implementation that was successfully demonstrated is shown schematically in Figure~\ref{fig:RECAP1} \cite{RECAP}.  A thin dielectric substrate supports an array of square metallic pads.  The pads are electrically small, with side length $l$ satisfying $l/\lambda_o \ll 1$, where $\lambda_o$ is the free-space wavelength at the operating frequency.  Each pad is connected to its neighboring pads by switched links, indicated by the arrows in the figure.  Each switch may be independently set to open or closed depending on the desired antenna configuration.  A single feed point (pair of terminals) is located near the center of the antenna.

\begin{figure}\begin{center}
\includegraphics[angle=0,scale=0.7]{/Users/jim.maloney/Book/images/RECAP1.png}
\caption{Schematic drawing of the Agile Aperture Antenna in dipole form.  Square metallic pads are connected by switched links (arrows).  The state of each switch (open or closed) determines the antenna configuration \cite{RECAP}.}
\label{fig:RECAP1}
\end{center}\end{figure}

The Agile Aperture Antenna can be understood as a variant of the fragmented aperture antenna in which the fundamental unit is not a single pixel but a metallic pad composed of a group of pixels.  The pads are not contiguous; they are separated by narrow dielectric gaps.  The antenna structure for any given configuration consists of the conducting pads that are connected by closed switches, together with all of the unconnected pads that remain present on the substrate.  This is an important distinction from a conventional fragmented aperture: in the Agile Aperture Antenna, the unconnected pads are always physically present and contribute to the electromagnetic behavior of the antenna through scattering, even when they are not part of the connected conducting structure.

\section{Static Proof of Concept}

\textcolor{red}{INSERT: Description of the first two static (hard-wired) pixelated designs that demonstrated the unconnected pads did not prevent good antenna performance.  Include figures showing the two designs and their measured performance.}

\section{Reconfigurable Proof of Concept}

To prove the validity of the Agile Aperture Antenna concept, a detailed study was conducted using a prototype antenna with hard-wired switches---gaps that were either closed by a soldered wire or left open.  This study not only validated the design approach but also identified areas where future research would be needed to extend the concept to practical, electronically reconfigurable antennas.

\subsection{Prototype Description}

For all of the antennas discussed in this section, the aperture was formed from a printed circuit board 22.5~cm $\times$ 22~cm in size, with the pads etched from the copper cladding on one side of the board.  The pad side length and spacing were both $l = s = 1.0$~cm (see Figure~\ref{fig:RECAP1}).  The board contained a total of 120 pads and 208 switches.  The dielectric substrate was 1.7~mm thick FR4 circuit board with measured electrical properties $\varepsilon_r = 4.27$ and $\tan\delta = 0.07$ \textcolor{red}{(verify loss tangent value)}.

The frequency of operation was in the range $0.85~\text{GHz} \leq f \leq 1.45~\text{GHz}$, so the pads were electrically small: $0.028 \leq l/\lambda_o \leq 0.048$.  All of the antenna designs described in this section have mirror symmetry about the horizontal line through the feed point, including the states of the switches.  This symmetry allows the antennas to be analyzed and measured in either the ``dipole form'' shown in Figure~\ref{fig:RECAP1} or the ``monopole form'' shown in Figure~\ref{fig:RECAP2}.

\subsection{Measurement Setup}

\begin{figure}\begin{center}
\includegraphics[angle=0,scale=0.7]{/Users/jim.maloney/Book/images/RECAP2.png}
\caption{Experimental arrangement for measuring the Agile Aperture Antenna in monopole form.  The antenna is mounted vertically on a rotatable disc centered in a large metallic image plane \cite{RECAP}.}
\label{fig:RECAP2}
\end{center}\end{figure}

Figure~\ref{fig:RECAP2} shows the experimental setup used for all of the measurements reported in this section.  The monopole version of the Agile Aperture Antenna was mounted vertically on a rotatable disc centered in a large metallic image plane \textcolor{red}{(insert image plane dimensions)}.  The antenna was fed from below the image plane by a 50~$\Omega$ coaxial line, with the center conductor connected to the bottom pad in the center column of the antenna.  A calibrated TEM horn antenna was located at a distance of \textcolor{red}{(insert distance)} from the antenna \cite{RECAP}.  The scattering parameters for the two-port network formed by the Agile Aperture Antenna and the TEM horn were measured with a network analyzer and used to determine the absolute gain of the Agile Aperture Antenna \cite{LeeSmith2004}.  The horizontal radiation pattern ($|E|$ versus azimuth angle $\phi$) was obtained by rotating the disc while recording the output signal from the horn.

\subsection{Design Procedure}

The procedure for designing a switch configuration for the Agile Aperture Antenna is conceptually the same as for a conventional fragmented aperture: a rigorous FDTD simulation of the antenna is run in conjunction with a genetic algorithm optimizer (see Appendix~A for an introduction to the FDTD method).  In all of the FDTD simulations reported here, cubical Yee cells with a side length of 2.5~mm were used.  More recent optimization approaches for reconfigurable pixelated antennas include beam steering techniques \cite{Towfiq2018Reconfigurable} and quantum genetic algorithms \cite{Bichara2021Miniaturized}.

A performance goal is first established---for example, maximum broadside realized gain over a specified bandwidth.  The GA then searches for the switch configuration (which switches should be open and which should be closed) that best meets this goal.  Taking into account the mirror symmetry of the antenna, there are $2^{104} \approx 2 \times 10^{31}$ possible switch configurations---far too many to evaluate exhaustively.  The GA provides an efficient, though approximate, method for searching this enormous design space.

\subsection{Broadside Design}

The design goal for the first example was to maximize the broadside realized gain over the frequency range $0.85~\text{GHz} \leq f \leq 1.25~\text{GHz}$ (a fractional bandwidth of 38\%).  The target was that the realized gain should equal or exceed the directivity of a uniform sheet of vertically directed current occupying the same aperture area.

Figure~\ref{fig:RECAP3}(a) shows the switch configuration obtained by the GA for this broadband, bidirectional, broadside design.  Notice that this configuration has right-left symmetry in addition to the imposed top-bottom symmetry; all of the broadside designs discussed in this chapter are constrained to have this additional symmetry.

\begin{figure}\begin{center}
\includegraphics[angle=0,scale=0.7]{/Users/jim.maloney/Book/images/RECAP3.png}
\caption{Switch configurations for the Agile Aperture Antenna (monopole form) with hard-wired switches.  (a) Broadband, bidirectional, broadside design.  (b) Narrowband, unidirectional, end-fire design.  The two configurations are strikingly different, yet both are realized on the same physical antenna \cite{RECAP}.}
\label{fig:RECAP3}
\end{center}\end{figure}

Figure~\ref{fig:RECAP4}(a) shows the realized gain versus frequency for this design.  The dashed blue line is the design goal (uniform aperture directivity), the solid black line is the FDTD simulation, and the red line with markers is the measured result.  All realized gain values are for the antenna in the dipole configuration.  The simulated and measured realized gains are in good agreement over the design bandwidth, with a maximum difference of approximately 1~dB.  The realized gain falls approximately 0.5--1.5~dB below the goal; a portion of this difference is attributable to impedance mismatch at the antenna feed.

Figure~\ref{fig:RECAP4}(b) shows the mismatch factor $(1 - |\Gamma_A|^2)$ as a function of frequency, where $\Gamma_A$ is the voltage reflection coefficient at the antenna terminals.  Within the design bandwidth, this factor ranges from 0.0~dB to $-1.5$~dB, confirming that mismatch accounts for a significant portion of the difference between the realized gain and the goal.

\begin{figure}\begin{center}
\includegraphics[angle=0,scale=0.7]{/Users/jim.maloney/Book/images/RECAP4.png}
\caption{Results for the broadband, bidirectional, broadside design with hard-wired switches.  (a) Realized gain versus frequency.  (b) Gain reduction due to impedance mismatch: $(1-|\Gamma_A|^2)$ \cite{RECAP}.}
\label{fig:RECAP4}
\end{center}\end{figure}

Figure~\ref{fig:RECAP5} shows the horizontal radiation pattern at the center frequency $f = 1.05$~GHz.  The simulated and measured patterns are nearly identical, with both normalized to a maximum of 0~dB.  The heavy line at the center of the pattern indicates the orientation of the dielectric substrate.

\begin{figure}\begin{center}
\includegraphics[angle=0,scale=0.5]{/Users/jim.maloney/Book/images/RECAP5.png}
\caption{Horizontal radiation pattern at $f = 1.05$~GHz for the broadband, bidirectional, broadside design with hard-wired switches.  Both patterns are normalized to 0~dB \cite{RECAP}.}
\label{fig:RECAP5}
\end{center}\end{figure}

\subsection{End-Fire Design}

To demonstrate the versatility of the Agile Aperture Antenna concept, a second design was produced for a completely different objective: a narrowband, unidirectional, end-fire beam over the frequency range $1.0~\text{GHz} \leq f \leq 1.1~\text{GHz}$ (a fractional bandwidth of 9.5\%).  The goal was again that the realized gain should equal or exceed the directivity of a uniform sheet of current, but now with the current phased to produce end-fire radiation.

Figure~\ref{fig:RECAP3}(b) shows the switch configuration for this end-fire design.  It is immediately apparent that this configuration is strikingly different from the broadside configuration in Figure~\ref{fig:RECAP3}(a)---the end-fire design does not have right-left symmetry and produces a fundamentally different current distribution on the aperture.  Yet both configurations are realized on the same physical hardware simply by changing which switches are open and which are closed.

Figure~\ref{fig:RECAP6}(a) shows the realized gain versus frequency for the end-fire design.  The simulated and measured results are again in good agreement over the design bandwidth, with a maximum difference of approximately 1~dB.  The realized gain falls approximately 1.0--2.0~dB below the goal.  The mismatch factor shown in Figure~\ref{fig:RECAP6}(b) is within the range 0.0~dB to $-0.8$~dB over the design bandwidth.

\begin{figure}\begin{center}
\includegraphics[angle=0,scale=0.6]{/Users/jim.maloney/Book/images/RECAP6.png}
\caption{Results for the narrowband, unidirectional, end-fire design with hard-wired switches.  (a) Realized gain versus frequency.  (b) Gain reduction due to impedance mismatch: $(1-|\Gamma_A|^2)$ \cite{RECAP}.}
\label{fig:RECAP6}
\end{center}\end{figure}

Figure~\ref{fig:RECAP7} shows the horizontal radiation pattern at $f = 1.05$~GHz.  The simulated and measured patterns are in excellent agreement, and both clearly show the characteristic end-fire beam directed to one side of the antenna.

\begin{figure}\begin{center}
\includegraphics[angle=0,scale=0.6]{/Users/jim.maloney/Book/images/RECAP7.png}
\caption{Horizontal radiation pattern at $f = 1.05$~GHz for the narrowband, unidirectional, end-fire design with hard-wired switches.  Both patterns are normalized to 0~dB \cite{RECAP}.}
\label{fig:RECAP7}
\end{center}\end{figure}

\subsection{Observations on the Designed Configurations}

In the configurations studied, approximately 30\% to 60\% of the switches were closed.  One might expect that examination of the switch states for a particular design would reveal recognizable antenna structures---for example, the end-fire design might show strings of pads forming linear elements arranged like the driven element, reflector, and director of a Yagi-Uda array.  However, as seen in Figure~\ref{fig:RECAP3}(b), this is not the case.  In general, there is no simple, discernible relationship between the switch states and the design goal.

This lack of an obvious physical interpretation is consistent with the experience from conventional fragmented aperture design (Chapter~2): the GA discovers complex, non-intuitive structures that exploit the full electromagnetic physics of the problem.  In the case of the Agile Aperture Antenna, the optimization is further complicated by the presence of the unconnected pads, which scatter electromagnetic energy and must be accounted for in the design.  It is clear, however, that the switch connections near the feed point are often arranged to improve the impedance match between the antenna and the transmission line.

\section{Discussion}

The broadside and end-fire examples presented above, along with several other designs not shown, demonstrate that the Agile Aperture Antenna concept is viable: a single physical antenna can be reconfigured via its switch states to meet fundamentally different performance specifications.  The excellent agreement between FDTD simulations and measurements further validates the design procedure.

However, for the Agile Aperture Antenna to transition from a laboratory concept to a practical technology, several challenges must be addressed.  Most critically, a switch technology must be developed that can be electronically controlled without interfering with the electromagnetic performance of the antenna.  The switches must introduce minimal insertion loss when closed, provide high isolation when open, and the control circuitry (bias lines, drivers) must not distort the antenna's radiation characteristics.

Subsequent research has explored several promising switch technologies for reconfigurable pixelated antennas.  MEMS-based switches have been successfully demonstrated for reconfigurable patch antennas \cite{Wright2016Effect,Ali2014Mems,Wright2018Mems}, offering excellent RF performance with low insertion loss and high isolation.  Phase transition materials such as vanadium dioxide (VO2) provide another approach, enabling frequency reconfiguration through voltage-controlled phase changes \cite{Lou2025Frequency}.  Other techniques include magnetically actuated pixels \cite{Pal2018Magnetically} and tunable designs using varactors \cite{Ouedraogo2014Tunable}.  More recent work has focused on dual-port mmWave reconfigurable designs with optimized pixel configurations \cite{Tang2023Dual-port} and novel frequency reconfigurable implementations \cite{Qiao2023Novel,Bichara2021Miniaturized}.

\section{Acknowledgement}
The author would like to thank Professor Glenn Smith for his dedication in writing the original IEEE paper \cite{RECAP} on which much of this chapter is based.  The author also acknowledges the contributions of the full research team as described in \cite{MaloneySmithAntChapters}.

\addcontentsline{toc}{section}{References}
\FloatBarrier

% Bibliography for Chapter 5
% Uses chapter-specific .bib files organized by topic
\bibliography{../Literature/master_bibliography,%
              ../Literature/fragmented_aperture_core,%
              ../Literature/reconfigurable_ch5,%
              ../Literature/ml_ai_optimization_methods}
\bibliographystyle{IEEEtran}


