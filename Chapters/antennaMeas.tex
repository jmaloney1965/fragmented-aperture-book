\chapter{Focused Beam Measurement of Antenna Gain}
%\authortoc{James G. Maloney}
%\chapterauthor{James G. Maloney}

\section{Introduction}

Over the last few years, the authors have been involved with developing the use of the focused beam measurement system to measure antenna properties such as gain and pattern [10].  A series of improved, fragmented aperture antenna designs will be fabricated and measured with the Compass Tech Focused Beam System and compared with the design predictions to validate the designs.  This comparison will be included in the poster presentation.

\section{Standard Focused Beam System Configurations}

Measuring antenna properties, such as gain and antenna pattern, traditionally require large anechoic chambers, expensive compact ranges or near-field scanning facilities.  The standard, two-foot (61 mm), focused beam system, shown in Figure~\ref{fig:EVF16}, has proven to be a reliable apparatus for measuring dielectric/magnetic materials, meta-materials, and FSS over the last several decades [11]. 

\begin{figure}
\begin{center}
\includegraphics[angle=0,width=\linewidth]{/Users/jim/Book/images/improvedAMTAFig16.png}
\caption{Standard, 2-foot (61 mm) Focused beam system.}
\label{fig:EVF16}		% figure references have to be below the \caption line
\end{center}
\end{figure}

The antenna under test (AUT) can be simply mounted in the center of the sample holder using a 2 foot x 2 foot foam sheet and the turntable enables the measurement of antenna pattern cuts as shown in Figure~\ref{fig:EVF16}.

The lenses in the standard focused beam system can be reconfigured to achieve different degrees of focusing. The three possibilities are termed ?standard beam?, ?expanded beam?, and ?collimated beam.?  For a complete description of the shape the standard lenses, see \cite{ReidThesis}(page 89). 

In Figure~\ref{fig:EVF17}, we compare the maximum antenna size that can be measured with each lens configuration.  In addition, we also look at the no-lens case to illustrate where the use of the lens provides increased capability over a simple two antenna method.  For all practical purposes Figure ~\ref{fig:EVF17} shows that only the collimated beam option allows measuring a larger antenna than simply having no lenses.  

In effect, the collimated beam configuration removes the quadratic phase across the test volume in a similar manner to how the reflector in large compact ranges mitigates the quadratic phase.

\section{Experimental Measurements}

\begin{figure}
\includegraphics[angle=0,width=\linewidth]{/Users/jim/Book/images/improvedAMTAFig17.png}
\caption{Maximum Antenna Size for 4 lens configurations}
\label{fig:EVF17}		% figure references have to be below the \caption line
\end{figure}

Figure~\ref{fig:EVF17} also shows that if we build experimental fragmented aperture prototypes that are no larger than 10? (~0.25 m) for use at frequencies below 10 GHz we should be able to safely use the collimated beam to measure the gain and pattern.



\section{Gain Measurements of Compass Technology SP2-18 Probe Antenna}


\section{Gain Measurements of Fragmented Aperture Antennas}

\textcolor{red}{We will include, measurements of  series of experimental fragmented prototypes.} 

\textcolor{red}{We will compare the gain measured using no-lens (i.e. two-antenna far-field method) with the gain measured using the collimated lens}

\textcolor{red}{NEED TO BUILD AND MEAS A FRAGMENTED SAMPLE FROM CHAPTER 4}
 

\begin{thebibliography}{99}

\bibitem{MaloneyAMTA2012} J. Maloney, J. Fraley, M. Habib, J. Schultz, K. C. Maloney, ``Focused Beam Measurement of Antenna Gain Patterns'', AMTA, 2012

\bibitem{SchultzBook} John W. Schultz, Focused Beam Methods, 2012.

\bibitem{ReidThesis} David Reid, A full electromagnetic analysis of fresnel zone plate antennas and the application to a free-space focused-beam measurement system, PhD. Thesis, Georgia Tech, Nov. 2.

\end{thebibliography}





