\documentclass{book}
\renewcommand\bibname{References}
\usepackage{etoolbox}
\patchcmd{\thebibliography}{\chapter*}{\section*}{}{}	% this removes blank pages before references
%\usepackage[nottoc,notlot,notlof]{tocbibind}		% this forces placement of references in TOC


\usepackage{geometry}
\usepackage{xcolor,placeins}
\usepackage{amssymb,amsmath}
\geometry{paperwidth=6.14in,paperheight=9.21in,
  left=0.75in,right=0.60in,top=0.75in,bottom=0.75in,twoside}

\usepackage{times} % default: times 11pt, \baselinestretch=1.06
\usepackage{fancyhdr} % headers
\pagestyle{fancy} % style for headers
\usepackage{pifont} % for the copyright symbol
\headheight 0.25in % height of header
\headsep 0.25in % dist between top of text and bottom of header
\parskip 0in
\parindent 0.25in % = 6.35mm
\newcommand{\str}{\rule{0pt}{16pt}} % strut used in toc
%%% header stuff
\renewcommand{\chaptermark}[1]{\markboth{\MakeUppercase{#1}}{}}
\newcommand{\currpage}{\footnotesize\thepage} 	% use in headers
\fancyhead{} 								% clear header fields, required
\fancyfoot{} 								% clear footer fields, required
\fancyhead[RO,LE]{\currpage} 					% Odd page, Right side; Even page, left side
\newcommand{\lo}{\footnotesize{}\textcolor{black}{FRAGMENTED APERTURE ANTENNAS}}
\fancyhead[LO]{\lo} 							% Odd page, Left side
\newcommand{\re}{\footnotesize{}James G. Maloney}
\fancyhead[RE]{\footnotesize\re} 				% Even Page, Right side
%\renewcommand\headrulewidth{0pt} 			% for no rule line under header


\usepackage{etoolbox}		% This makes chapter first page have fancy headers
\patchcmd{\chapter}{\thispagestyle{plain}}{\thispagestyle{fancy}}{}{}

\makeatletter
\DeclareRobustCommand\authortoctext[1]{{\addvspace{0pt}\nopagebreak\leftskip0em\relax\rightskip\@tocrmarg\relax\noindent\hspace{15pt}\itshape#1\par\addvspace{10pt}}}
\makeatother
\newcommand\authortoc[1]{\addtocontents{toc}{\authortoctext{#1}}}
% \gdef\chapterauthor{#1}
\usepackage{graphicx}
\graphicspath{/Users/jim.maloney/Book/images}

\usepackage{chapterbib}

\title{Fragmented Aperture Antennas:  Computational Design of Antenna Structure}

\date{Sep 3, 2016}

\author{Dr. James G. Maloney}

\begin{document}

\fancyhead{} 								% clear header fields, required
\fancyhead[RO,LE]{\currpage} 					% Odd page, Right side; Even page, left side
\newcommand{\ko}{\footnotesize{}{\leftmark}}
\fancyhead[LO]{\ko} 							% Odd page, Left side
\newcommand{\rf}{\footnotesize{}FRAGMENTED APERTURE ANTENNAS}
\fancyhead[RE]{\footnotesize\rf} 				% Even Page, Right side

\tableofcontents{}
\listoffigures{}
\listoftables{}

\newcommand{\chapterauthor}[1]{{\parindent0pt\vspace*{-25pt}\linespread{1.1}\large\scshape#1\par\nobreak\vspace*{35pt}}}

\clearpage
\thispagestyle{plain}
\par\vspace*{.35\textheight}{\centering Dedicated to my wife Kate, my children Kelsey, Emily, Grace and Sam.\par}

\chapter*{Acknowledgements}
Over the years, I have had the pleasure to work with many smart individuals who each contributed to the development of Fragmented Aperture Antennas.

For my whole career, Prof.\ Glenn Smith has always be a major supporter of my research efforts.  Starting with teaching me how to do research during my Ph.D. thesis, to discussing many research issues over coffee during my university lab years, to co-authoring many book chapters with me, to helping invent the Fragmented Aperture, Dr Smith became a treasured collaborator and friend.

In the early years, I had the good fortune to work with one of the brightest minds, Dr. Morris Kesler.  Morris was also one of the co-inventors of the Fragmented Aperture.  Morris also assisted me in writing book chapters on Modeling Periodic Structures that are used in the analysis and design of phased array antennas and other periodic structures.

I had many conversations with Dr. Eric Kuster when stuck on a research topic, and his non-engineer view point was very helpful to me over the years.

I would like to thank Mr. Paul Friederich, Dr. Lon Pringle, Mr. Jim Acree for their dedicated support as project directors for both the research on Fragmented Apertures and for helping advocate for customer solutions based on Fragmented Apertures.

In the later years, the development of Fragmented Aperture solutions for many diverse applications were impacted by the assistance of Mr. Brad Baker, Mr. Kevin Cook, Dr. Doug Denison, Ms. Lynn Fountain, Mr. James Fraley, Mr. David Landgren, and Dr. Todd Lee.

The fabrication of many of the Fragmented Apertures relied heavily on the machining skills of the best machinist I ever met, Mr. Kurt Weismayer.  Without Kurt's skills, many of the antennas certainly would not have been built exactly, and the measured RF performance would have been in worse agreement with the model predictions. 

I also have greatly enjoyed working closely with Dr. John Schultz over the last 10+ years.  Collaborating with John on many research projects has been rewarding.  Specifically, learning from John about the design of metamaterials and collaborating on their use with Fragmented antennas was always a pleasure. 

Lastly, I want to thank Ms. Rebecca Schultz and Dr. Kate Maloney for allowing me to assist Compass Technology Groups research efforts over the last few years.  Also, I would like to thank them for their support in further developing the Fragment Aperture antenna.

\chapter{Introduction to Fragmented Aperture Antennas}
%\authortoc{James G. Maloney}
%\chapterauthor{James G. Maloney}

\section{Overview}

This book describes a class of antennas called Fragmented Aperture Antennas.  Unlike traditional antennas whose physical shapes are guided by analytical insight and engineering intuition, fragmented aperture antennas are designed computationally.  A planar conducting surface is divided into many sub-wavelength regions, or pixels, each of which may be either conducting or non-conducting.  A genetic algorithm, working in concert with a full-wave electromagnetic simulation, determines which pixels should be conducting and which should not, so as to best satisfy a given set of antenna performance requirements.  The resulting antenna structures are complex, non-intuitive metallic patterns that often approach the theoretical limits of antenna performance for a given aperture size.

The fragmented aperture concept was invented in the late 1990s \cite{MaloneyKeslerHarms}, and the term ``Fragmented Aperture Antenna'' was coined by the author upon visual inspection of the optimized designs, which consistently showed metallic pixels forming many connected and disconnected fragments across the aperture surface.  The original concept was disclosed in U.S.\ Patent 6,323,809 \cite{MaloneyFragPatent}.  Since then, the concept has been extended to reconfigurable antennas \cite{Pringle2004Reconfigurable}, ultra-wideband phased arrays with bandwidths exceeding 33:1 \cite{Landgren2019Broadband}, and wide-scanning array designs \cite{Maloney2011Wide}.  Other research groups have successfully adopted the approach for broadband phased array element design using genetic algorithms \cite{Thors2005Broad-band} and fragmented antennas based on coupled small radiating elements \cite{Barani2018Fragmented}.  Fragmented aperture antennas have been successfully designed, fabricated, and measured for a wide variety of applications spanning frequencies from UHF through millimeter wave.

This introductory chapter provides background on fundamental antenna concepts, motivates the need for a computational approach to antenna design, introduces the fragmented aperture concept, and concludes with a roadmap for the remainder of the book.

\section{Antenna Fundamentals}

An antenna is a transducer between guided electromagnetic waves, such as those on a transmission line or waveguide, and free-space electromagnetic waves that propagate away from the antenna.  When used for transmission, an antenna converts a guided signal into radiation; when used for reception, it captures incident radiation and converts it into a guided signal.  By the principle of reciprocity, the properties of an antenna are the same whether it is transmitting or receiving \cite{IEEEstd}.

Several key parameters characterize the performance of an antenna:

\subsection{Radiation Pattern}
The radiation pattern describes the spatial distribution of electromagnetic energy radiated by an antenna as a function of direction.  It is typically represented as a plot of radiated power (or field strength) versus angle, normalized to the direction of maximum radiation.  Important features of the radiation pattern include the main beam (or main lobe), which is the angular region of strongest radiation; the sidelobes, which are regions of lesser radiation surrounding the main beam; and the back lobe, which is radiation in the direction opposite the main beam.  The beamwidth, usually defined as the angular width between the half-power ($-3$~dB) points of the main beam, quantifies how directional the antenna is.

\subsection{Directivity and Gain}
Directivity is a measure of how effectively an antenna concentrates radiated energy in a particular direction compared to an isotropic radiator (a hypothetical antenna that radiates equally in all directions).  The directivity $D$ in a given direction is defined as the ratio of the radiation intensity in that direction to the radiation intensity averaged over all directions:
\begin{equation}
D = \frac{U(\theta,\phi)}{U_{\text{avg}}} = \frac{4\pi U(\theta,\phi)}{P_{\text{rad}}}
\end{equation}
where $U(\theta,\phi)$ is the radiation intensity (power per unit solid angle) and $P_{\text{rad}}$ is the total radiated power.

Gain is closely related to directivity but also accounts for losses within the antenna.  The gain $G$ is related to the directivity $D$ by the radiation efficiency $\eta$:
\begin{equation}
G = \eta \, D
\end{equation}
where $\eta$ accounts for ohmic and dielectric losses in the antenna structure.

In practice, it is often useful to work with the \emph{realized gain}, which further accounts for impedance mismatch between the antenna and its feed:
\begin{equation}
G_{\text{realized}} = (1 - |\Gamma|^2) \, G
\end{equation}
where $\Gamma$ is the voltage reflection coefficient at the antenna terminals.  The realized gain captures the overall effectiveness of the antenna in converting guided-wave power into radiation in a given direction.

\subsection{Aperture and Aperture Efficiency}

A fundamental result in antenna theory relates the maximum achievable gain of an antenna to its physical aperture area $A$:
\begin{equation}
\label{eq:apGain}
G_{\text{max}} = \frac{4\pi A}{\lambda^2}
\end{equation}
where $\lambda$ is the free-space wavelength.  This result applies to a uniformly illuminated aperture radiating into one hemisphere (i.e., with a ground plane or reflector behind it).  For an aperture that radiates equally into both hemispheres (no ground plane), the limit becomes $2\pi A / \lambda^2$.

The aperture efficiency $\eta_a$ describes how closely an antenna approaches this theoretical limit:
\begin{equation}
\eta_a = \frac{G}{G_{\text{max}}} = \frac{G \lambda^2}{4\pi A}
\end{equation}
Traditional antenna designs rarely achieve aperture efficiencies above 50--70\%.  As will be shown throughout this book, fragmented aperture antennas routinely approach the theoretical aperture gain limit, often achieving efficiencies that exceed those of conventional designs.

\subsection{Bandwidth}

Every antenna has a finite bandwidth over which it operates satisfactorily.  Bandwidth may be defined in terms of several criteria, but the most common is the impedance bandwidth: the range of frequencies over which the antenna maintains an acceptable impedance match to its feed line.  A common threshold is a voltage standing wave ratio (VSWR) of 2:1, corresponding to a return loss of approximately 10~dB, meaning that no more than 10\% of the incident power is reflected back to the source.  More demanding applications may require a VSWR below 1.5:1 (return loss better than 14~dB) or even 1.3:1.

Bandwidth can be expressed as a ratio of the upper to lower frequency limits (e.g., 10:1 bandwidth for an antenna operating from 1 to 10~GHz) or as a fractional bandwidth:
\begin{equation}
\text{BW}_{\text{frac}} = \frac{f_H - f_L}{f_c}
\end{equation}
where $f_H$ and $f_L$ are the upper and lower frequency limits and $f_c$ is the center frequency.  Antennas with bandwidths of 2:1 or greater are often called wideband, while those exceeding roughly 10:1 are called ultra-wideband.

Achieving wide bandwidth while maintaining high gain and an acceptable impedance match is one of the central challenges in antenna design and a particular strength of the fragmented aperture approach.

\subsection{Polarization}

The polarization of an antenna describes the orientation of the electric field vector of the radiated wave.  Common polarizations include linear (vertical or horizontal), circular (right-hand or left-hand), and elliptical.  The polarization of the radiated field is determined by the currents flowing on the antenna structure, and achieving a desired polarization is an important design goal.  Fragmented aperture antennas can be designed for any of these polarizations, including cases where the polarization varies with beam direction.

\section{Antenna Arrays}

A single antenna element has a radiation pattern determined by its geometry and size.  To achieve higher gain, narrower beams, or the ability to steer the beam electronically, multiple antenna elements are arranged in an array.  In an array, the signals from the individual elements combine coherently, and the resulting radiation pattern is the product of the individual element pattern and the array factor, which depends on the element spacing, number of elements, and the relative amplitude and phase of the excitation at each element.

Electronic beam steering is accomplished by adjusting the relative phases of the signals at each element.  This is the basis of the phased array, which can rapidly redirect its beam without physically moving the antenna.  Phased arrays are essential in modern radar, communications, and electronic warfare systems.

However, the design of wideband phased arrays presents significant challenges.  Mutual coupling between array elements---the electromagnetic interaction between neighboring elements---can cause scan blindness, a condition where the array is poorly matched at certain combinations of frequency and scan angle.  Traditional array design approaches attempt to minimize mutual coupling, but as will be shown in this book, the fragmented aperture approach embraces mutual coupling, and in some cases exploits direct electrical connections between elements to achieve bandwidths that far exceed those of conventional array elements.

\section{Limitations of Traditional Antenna Design}

Traditional antenna design relies on a library of known antenna types---dipoles, patches, horns, spirals, log-periodic structures, and others---each with well-understood behavior that can be predicted analytically or with simple numerical models.  The designer selects an antenna type appropriate for the application and then adjusts a relatively small number of geometric parameters (lengths, widths, feed positions, spacings) to optimize performance.

This approach has been enormously successful and has produced the antenna designs in widespread use today.  However, it is inherently constrained by the set of geometries that have been studied and understood.  The designer is limited to exploring variations within known antenna topologies.  The number of degrees of freedom available for optimization is small---typically fewer than a dozen parameters---and the design space is correspondingly limited.

Consider, by contrast, an antenna aperture divided into a grid of 200 sub-wavelength pixels, each of which may be independently set to conducting or non-conducting.  This seemingly simple description defines a design space of $2^{200} \approx 10^{60}$ possible antenna geometries.  The vast majority of these configurations have never been conceived by any antenna designer, and many of them produce antenna characteristics that are unlike any known antenna type.  The challenge, of course, is finding the configurations that produce useful antennas among this enormous number of possibilities.

This is precisely the challenge that the fragmented aperture design approach addresses.

\section{The Fragmented Aperture Concept}

The fragmented aperture antenna design approach combines three essential elements:

\begin{enumerate}

\item \textbf{A pixelated aperture.}  The antenna surface is divided into a grid of sub-wavelength regions, or pixels.  Each pixel is assigned a binary state: conducting (metal) or non-conducting (absent).  The set of all pixel states defines the antenna geometry.  Early fragmented apertures used rectangular pixels on a rectilinear grid, but as described in Chapter~3, improved pixel shapes and lattice geometries have been developed to address fabrication challenges.

\item \textbf{A full-wave electromagnetic simulation.}  A rigorous numerical solution of Maxwell's equations is used to predict the antenna performance for any given pixel configuration.  The finite-difference time-domain (FDTD) method has been used exclusively in the author's work because a single time-domain simulation efficiently produces antenna characteristics across the entire frequency band of interest.  A detailed description of the FDTD method as applied to antennas is provided in Appendix~A.

\item \textbf{An evolutionary optimization algorithm.}  Because the design space is far too large for exhaustive search, a genetic algorithm (GA) is used to efficiently explore the space of possible pixel configurations.  The GA maintains a population of candidate antenna designs, evaluates each design using the FDTD simulation, and evolves the population over many generations using selection, crossover, and mutation operations.  The algorithm converges toward designs that best satisfy the specified performance goals.

\end{enumerate}

The result of this design process is an antenna whose physical structure has been computationally optimized to meet a particular set of performance requirements.  The antenna shapes that emerge from this process are invariably complex and non-intuitive.  Inspection of the metallic regions on the aperture reveals many interconnected and isolated fragments of conductor---hence the name \emph{fragmented aperture}.

A critical advantage of this approach is that the full-wave simulation captures all of the relevant physics: mutual coupling, surface wave effects, feed interactions, dielectric loading, and diffraction from edges and discontinuities.  The optimizer therefore has access to the true electromagnetic behavior of each candidate design, not an approximate or simplified model.  This is what allows fragmented aperture designs to routinely approach theoretical performance limits.

\section{Novelty and Significance}

The fragmented aperture antenna represents a fundamentally different philosophy of antenna design.  Rather than starting from an analytical understanding of how a particular geometric shape radiates and then perturbing that shape to improve performance, the fragmented aperture approach starts from a general description of the design space (the set of all possible pixel configurations) and uses computation to discover structures that meet the desired specifications.  In this sense, the antenna structure itself is the output of the design process, not the input.

This computational approach to designing antenna structure has produced several notable results:

\begin{itemize}
\item Single-element fragmented aperture antennas that approach the theoretical aperture gain limit $2\pi A / \lambda^2$ for apertures without a ground plane, across bandwidths exceeding an octave.

\item Reconfigurable fragmented aperture antennas (Agile Aperture Antennas) that can electronically switch between different operating modes---changing beam direction, bandwidth, or polarization---by opening and closing switched links between metallic pads.

\item Ultra-wideband phased array elements based on the fragmented aperture concept that achieve bandwidths of 33:1, with preliminary work suggesting that 100:1 bandwidths are achievable.  A key insight enabling these designs was that electrical connections between array elements should be exploited rather than avoided.

\item Improved pixel geometries that eliminate the fabrication issue of ``diagonal touching''---a problem that plagued early fragmented aperture designs and caused poor agreement between modeled and measured antenna performance.

\item An improved mutation algorithm for the genetic algorithm that significantly accelerates convergence for designs with large numbers of pixels.
\end{itemize}

These results, and others, are described in detail in the chapters that follow.

\section{Organization of the Book}

The remainder of this book is organized as follows:

\textbf{Chapter~2: Original Approach to Design Fragmented Apertures.}  This chapter describes the original fragmented aperture concept as disclosed in U.S.\ Patent 6,323,809.  The genetic algorithm design approach is presented, along with early design results that demonstrated the viability of the concept.

\textbf{Chapter~3: Improved Approach to Design Fragmented Apertures.}  The original fragmented aperture approach suffered from the problem of diagonal touching, where pixels that touch only at corners lead to fabrication difficulties and poor model-measurement agreement.  This chapter presents three improved pixel geometries that inherently avoid diagonal touching, along with an improved mutation algorithm that accelerates the convergence of the genetic algorithm for designs with many pixels.

\textbf{Chapter~4: Sample Antenna Designs.}  This chapter presents a gallery of fragmented aperture antenna designs that illustrate the versatility of the approach.  Topics include various feed strategies, bandwidth tailoring, fixed beam steering, polarization control (linear and circular), beamwidth tailoring, and out-of-band rejection.

\textbf{Chapter~5: Reconfigurable Fragmented Aperture Antennas.}  This chapter describes the Agile Aperture Antenna, a reconfigurable antenna in which switched links between metallic pads allow the antenna to be electronically reconfigured to meet different performance specifications.  Static and reconfigurable proof-of-concept designs are presented with measured results.

\textbf{Chapter~6: Fragmented Array Elements.}  This chapter extends the fragmented aperture concept to the design of individual elements for phased array antennas.  The approach for incorporating scan performance into the design process is described, and example designs spanning multiple octaves of bandwidth are presented.

\textbf{Chapter~7: Wideband Antenna Arrays.}  This chapter describes the development of ultra-wideband phased arrays using fragmented aperture design principles.  Key innovations include connected arrays, broadband screen backplanes using resistive card layers to mitigate ground-plane nulls, and multi-layer radiators for improved front-to-back ratio.  Measured results for arrays with bandwidths up to 33:1 are presented.

\textbf{Chapter~8: Designing Wide Scan Fragmented Array Antennas.}  This chapter describes techniques for designing fragmented aperture array elements with wide scan volumes ($\pm 60^{\circ}$ and beyond) using spectral-domain FDTD simulation within the genetic algorithm design process.  A laminated printed circuit board fabrication approach is presented, and results for a whole X-band (8--12~GHz) array element are shown.

\textbf{Chapter~9: Reconfigurable Arrays.}  This chapter describes reconfigurable phased array antennas that combine the fragmented aperture design approach with electronic reconfiguration, enabling improved scan performance and dynamic polarization control.

\textbf{Appendix~A: Computational Modeling of Antennas.}  This appendix provides a self-contained introduction to the finite-difference time-domain (FDTD) method as applied to antenna analysis.  The basic algorithm is described, the formulation for both transmitting and receiving antennas is presented, and several examples are provided to illustrate the accuracy and utility of the method.


% Bibliography for Chapter 1
% Uses chapter-specific .bib files organized by topic
\bibliography{../Literature/master_bibliography,%
              ../Literature/fragmented_aperture_core}
\bibliographystyle{IEEEtran}


		% ch 1	
\chapter{Original Approach to Design Fragmented Apertures}
%\authortoc{James G. Maloney}
%\chapterauthor{James G. Maloney}

\section{Aperture Utilization}

A central goal in antenna design is to make effective use of the available aperture area.  As discussed in Chapter~1, the theoretical maximum gain for a uniformly illuminated planar aperture of area $A$ radiating into one hemisphere is $4\pi A/\lambda^2$, and for an aperture radiating into both hemispheres (no ground plane), the limit is $2\pi A/\lambda^2$.  Traditional antenna designs---dipoles, patches, spirals, and the like---typically utilize only a fraction of the available aperture area at any given frequency.  For example, a spiral antenna uses an ``active region'' whose size scales with frequency; at any given operating frequency, much of the physical aperture is not contributing to the radiation.

The original fragmented aperture concept was motivated by a simple question: \emph{can one design a planar antenna that utilizes the entire aperture area across a wide frequency band, thereby approaching the theoretical gain limit?}  The answer, as demonstrated in this chapter, is yes---provided one is willing to abandon traditional antenna geometries in favor of computationally optimized structures.

\section{Original Genetic Design Approach}

\subsection{Binary Encoding of Antenna Geometry}

The fundamental idea behind the fragmented aperture antenna is to represent the antenna geometry as a binary string that can be manipulated by an evolutionary optimization algorithm.  The antenna surface is divided into a grid of sub-wavelength rectangular pixels, as illustrated in Figure~\ref{fig:OFA1}.  Each pixel is assigned a single binary value: 1 for conducting (metal present) and 0 for non-conducting (metal absent).  The collection of all pixel states defines the antenna geometry and constitutes the ``genetic code'' of the antenna.

\begin{figure}
\includegraphics[angle=0.5,trim={0 160bp 0 18bp},clip,width=\linewidth]{/Users/jim.maloney/Book/images/origPatentFig1.png}
\caption{The fragmented aperture concept: a planar surface divided into a grid of sub-wavelength pixels, each either conducting (black) or non-conducting (white).  The pattern of conducting and non-conducting elements defines the antenna geometry \cite{MaloneyFragPatent}.}
\label{fig:OFA1}
\end{figure}

This binary representation maps naturally onto a genetic algorithm (GA).  The GA maintains a population of candidate antenna designs, each represented by a binary string.  Over successive generations, the population evolves toward better antenna designs through the standard genetic operations of selection, crossover, and mutation.

\subsection{Two-Stage Optimization}

The original design process employed a two-stage optimization approach, as illustrated in Figure~\ref{fig:OFA4}.  In the first stage, the aperture area is described using a relatively small number of trapezoidal conducting strips arranged symmetrically about a coaxial feed point.  Each strip has a variable length that is encoded in the binary representation.  This coarse description of the antenna geometry allows the GA to quickly explore the design space and identify promising regions.

\begin{figure}
\includegraphics[angle=0.5,trim={0 120bp 0 30bp},clip,width=\linewidth]{/Users/jim.maloney/Book/images/origPatentFig4.png}
\caption{First optimization stage: the aperture is described using trapezoidal conducting strips of variable length arranged about a coaxial feed.  This coarse parameterization enables rapid exploration of the design space \cite{MaloneyFragPatent}.}
\label{fig:OFA4}
\end{figure}

In the second stage, the best design from the first stage is converted to the full pixel representation and the GA continues to optimize at the pixel level, refining the antenna geometry to improve performance.  The flowchart for this second stage is shown in Figure~\ref{fig:OFA5}.

\begin{figure}
\includegraphics[angle=0.5,trim={0 400bp 0 0},clip,width=0.85\linewidth]{/Users/jim.maloney/Book/images/origPatentFig5.png}
\caption{Flowchart of the genetic optimization process for fragmented aperture design.  The algorithm iteratively toggles pixels between conducting and non-conducting states, evaluating the antenna performance at each step using full-wave electromagnetic simulation \cite{MaloneyFragPatent}.}
\label{fig:OFA5}
\end{figure}

\subsection{Fitness Evaluation with FDTD}

At each generation of the GA, every candidate antenna in the population must be evaluated to determine how well it meets the design objectives.  This evaluation requires a full-wave electromagnetic simulation of each candidate antenna.  The finite-difference time-domain (FDTD) method was used exclusively for this purpose because a single time-domain simulation produces the antenna response across the entire frequency band of interest via Fourier transformation (see Appendix~A for details).

The fitness function used to evaluate each candidate antenna was typically based on the broadside realized gain across the design bandwidth.  Designs that achieved good impedance match (low VSWR) and high broadside gain over the specified frequency range received higher fitness scores.  The GA then preferentially selected high-fitness individuals for reproduction, driving the population toward better antenna designs over successive generations.

Even with the efficiency of the FDTD method, the computational cost of evaluating hundreds or thousands of candidate antennas over many GA generations was substantial.  This was one of the earliest applications of large-scale parallel computing to antenna design, with populations of antennas evaluated simultaneously on clusters of workstations.

\subsection{Symmetry Constraints}

To reduce the size of the design space and to ensure that the resulting antenna designs had desirable radiation characteristics, symmetry constraints were typically imposed during the optimization.  For a vertically polarized broadside antenna, left-right and top-bottom symmetry were enforced, reducing the number of independent pixels (and hence the length of the binary string) by a factor of four.  For example, an aperture with 400 pixels and both symmetries enforced has only 100 independent degrees of freedom, corresponding to a design space of $2^{100}$ possible configurations---still enormous, but significantly more tractable for the GA.

\section{First Success}
\label{sec:origFirstSuccess}

The first successful fragmented aperture antenna design was a planar aperture optimized to operate from 800~MHz to 2.5~GHz (a bandwidth of approximately 3:1).  The aperture was 10 inches $\times$ 10 inches (25.4~cm $\times$ 25.4~cm) and was excited at a feed point near the center of the aperture.  The optimized design is shown in Figure~\ref{fig:OFA3}.

\begin{figure}
\includegraphics[angle=0.5,trim={0 580bp 0 250bp},clip,width=\linewidth]{/Users/jim.maloney/Book/images/origPatentFig3.png}
\caption{The first successful fragmented aperture antenna: a 10-inch $\times$ 10-inch aperture optimized for 800~MHz to 2.5~GHz.  The complex pattern of conducting (black) and non-conducting (white) regions was determined entirely by the genetic algorithm and FDTD simulation.  The feed is located at the right side of the aperture.  Left-right and top-bottom symmetry lines are indicated by the dashed lines \cite{MaloneyFragPatent}.}
\label{fig:OFA3}
\end{figure}

The visual complexity of the design in Figure~\ref{fig:OFA3} is striking.  The conducting regions form an intricate pattern of connected and disconnected fragments that bears no resemblance to any traditional antenna geometry.  It was this visual character---the many fragments of conductor scattered across the aperture---that led to the name ``Fragmented Aperture Antenna.''

Despite the non-intuitive appearance of the design, the measured performance was excellent.  Figure~\ref{fig:OFA7} compares the measured broadside gain with the FDTD prediction and with two reference curves: the uniform aperture gain limit ($2\pi A/\lambda^2$, since there is no ground plane) and the gain of a spiral antenna of the same aperture size.

\begin{figure}
\includegraphics[angle=0.5,trim={0 900bp 0 120bp},clip,width=\linewidth]{/Users/jim.maloney/Book/images/origPatentFig7.png}
\caption{Measured and predicted broadside gain for the first fragmented aperture antenna (Figure~\ref{fig:OFA3}).  The fragmented design closely approaches the uniform aperture gain limit across the 800~MHz to 2.5~GHz optimization range, and significantly outperforms a spiral antenna of the same aperture size.  The measured and FDTD-predicted gains are in excellent agreement \cite{MaloneyFragPatent}.}
\label{fig:OFA7}
\end{figure}

Several important observations can be drawn from Figure~\ref{fig:OFA7}:

\begin{itemize}

\item Within the optimization range of 800~MHz to 2.5~GHz, the fragmented aperture closely approaches the uniform aperture gain limit.  This demonstrates that the GA/FDTD design process is effective at utilizing the full aperture area.

\item The measured and FDTD-predicted gains are in excellent agreement across the entire frequency range, validating the accuracy of the FDTD model used in the design process.

\item The fragmented aperture significantly outperforms a spiral antenna of the same physical aperture size.  The spiral, being a traveling-wave antenna with a frequency-dependent active region, does not utilize the full aperture at any given frequency.

\item Outside the optimization range, the antenna performance degrades, as expected.  The GA optimized the design specifically for the 800~MHz to 2.5~GHz band, and performance outside this band was not part of the fitness function.

\end{itemize}

\section{Bidirectional Radiation}

Because the first fragmented aperture antennas were single-layer planar structures with no ground plane, they radiated into both hemispheres.  Figure~\ref{fig:OFA8} shows the H-plane radiation pattern of the first successful design, comparing the FDTD prediction with the measured pattern.

\begin{figure}
\includegraphics[angle=0.5,trim={0 680bp 0 60bp},clip,width=\linewidth]{/Users/jim.maloney/Book/images/origPatentFig8.png}
\caption{H-plane radiation pattern of the first fragmented aperture antenna, comparing the measured pattern with the FDTD model prediction.  The pattern is clearly bidirectional, with roughly equal radiation into the forward and backward hemispheres \cite{MaloneyFragPatent}.}
\label{fig:OFA8}
\end{figure}

The bidirectional radiation pattern is clearly visible in Figure~\ref{fig:OFA8}, with the antenna producing roughly equal radiation in the forward and backward directions.  The model-measurement agreement is again excellent.  This bidirectional behavior is a natural consequence of the single-layer planar geometry: since the antenna structure is symmetric about the plane of the aperture (to within the thickness of the conductor), there is no physical mechanism to preferentially direct radiation into one hemisphere.

For many applications, bidirectional radiation is undesirable---half of the radiated power is directed away from the intended coverage area.  This motivates the use of a ground plane behind the aperture.  However, as will be discussed in detail in Chapter~7, a simple conducting ground plane introduces half-wave nulls at frequencies where the aperture-to-ground-plane spacing is an integer multiple of $\lambda/2$.  Addressing this challenge led to the development of broadband screen backplanes and multi-layer radiating structures that are key features of the wideband fragmented array designs.

\section{Fragmented Broadband Ground Planes}

An interesting early application of the fragmented aperture concept was the design of broadband ground planes.  Just as the pixel pattern on the radiating aperture can be optimized to achieve desired antenna characteristics, the pixel pattern on a surface behind the aperture can be optimized to function as a broadband ground plane.

A conventional conducting ground plane placed a quarter wavelength behind a radiating aperture provides constructive interference at the design frequency: the backward-radiated wave reflects off the ground plane and, after traveling an additional half wavelength (round trip), arrives back at the aperture in phase with the forward-radiated wave.  However, this constructive interference is inherently narrowband.

By replacing the solid conducting ground plane with a fragmented surface---a pixelated pattern of conducting and non-conducting regions---it is possible to design a reflector that provides a more uniform phase response over a wider bandwidth.  Figure~\ref{fig:OFA2} shows the transmission phase through a fragmented surface compared with a reference.

\begin{figure}
\includegraphics[angle=0.5,trim={0 1100bp 0 0},clip,width=\linewidth]{/Users/jim.maloney/Book/images/origPatentFig2.png}
\caption{Transmission phase comparison demonstrating the broadband properties of a fragmented surface \cite{MaloneyFragPatent}.}
\label{fig:OFA2}
\end{figure}

This early exploration of fragmented ground planes laid the groundwork for the more sophisticated broadband screen backplane designs described in Chapter~7, which use resistive card (r-card) layers in combination with a conducting ground plane to achieve wideband operation.

\section{The Original Patent and Early Publications}

The original fragmented aperture antenna concept, including both the radiating aperture and the broadband ground plane, was disclosed in U.S.\ Patent 6,323,809, ``Fragmented Aperture Antennas and Broadband Ground Planes,'' granted November 27, 2001 \cite{MaloneyFragPatent}.  The inventors were J.~G.~Maloney, M.~P.~Kesler, P.~H.~Harms, and G.~S.~Smith.

The first public presentation of the fragmented aperture concept occurred at the ICAP/JINA Conference on Antennas and Propagation in 2000 \cite{MaloneyKeslerHarms}.  The reconfigurable version of the concept (switched fragmented apertures) was presented at the IEEE Antennas and Propagation Symposium later that same year \cite{MaloneyKeslerLust}.  The concept was subsequently described in a number of conference papers and symposium presentations \cite{FriederichPringle}, and was included as part of a chapter on wideband arrays in the Modern Antenna Handbook \cite{BalanisHB12}.

Since the original publications, several other research groups have independently adopted the fragmented aperture design approach for their own applications.  Herscovici et al.\ applied the concept to aperture-coupled microstrip antennas \cite{Herscovici}.  Thors et al.\ used genetic algorithms for broadband fragmented aperture phased array element design \cite{Thors2005Broad-band}.  Ellgardt and Persson investigated wide-angle scanning fragmented aperture arrays \cite{Ellgardt2006Characteristics}.  More recent work has explored fragmented antennas based on coupled small radiating elements \cite{Barani2018Fragmented} and optimized designs with integrated baluns \cite{Zang2019Optimum}.  These and other efforts confirm the broad applicability of the fragmented aperture design philosophy.

\section{Lessons Learned}

The success of the original fragmented aperture antenna validated the fundamental premise that computational optimization could discover antenna geometries far beyond those accessible through traditional design approaches.  However, the early work also revealed important limitations that would drive subsequent research:

\begin{itemize}

\item \textbf{Diagonal touching.}  The original rectangular pixel geometry led to situations where conducting pixels touched only at their corners.  In the FDTD simulation, these diagonally touching pixels are always electrically connected, but when fabricated (e.g., by printed circuit board etching), the connection is unreliable.  This issue, and the solutions to it, are the subject of Chapter~3.

\item \textbf{Convergence for large pixel counts.}  As the number of pixels increased beyond approximately 100, the standard GA mutation operator became increasingly ineffective at exploring the design space.  An improved mutation strategy tailored for fragmented apertures is described in Chapter~3.

\item \textbf{Bidirectional radiation.}  Single-layer fragmented apertures without a ground plane radiate equally into both hemispheres, limiting the achievable gain to $2\pi A/\lambda^2$.  Addressing this limitation motivated the development of broadband backplanes (Chapter~7) and multi-layer radiating structures (Chapter~7).

\item \textbf{Fixed designs.}  Once fabricated, a fragmented aperture antenna operates with a single set of characteristics.  The desire for antennas that could dynamically change their operating characteristics led to the development of reconfigurable fragmented apertures (Chapter~5).

\end{itemize}

Despite these limitations, the original fragmented aperture concept established a powerful new paradigm for antenna design: one in which the physical structure of the antenna is determined by computation rather than by analytical insight alone.  The remaining chapters of this book describe how this paradigm has been extended and refined to address an increasingly wide range of antenna design challenges.


\FloatBarrier

% Bibliography for Chapter 2
% Uses chapter-specific .bib files organized by topic
\bibliography{../Literature/master_bibliography,%
              ../Literature/fragmented_aperture_core}
\bibliographystyle{IEEEtran}


 				% ch2
\chapter{Improved Approach to Design Fragmented Apertures}
%\authortoc{James G. Maloney}
%\chapterauthor{James G. Maloney}

\section{Overview}

In the late 1990s, Maloney et al.\ began investigating the design of highly pixelated apertures whose physical shape and structure are optimized using genetic algorithms (GA) and full-wave computational electromagnetic simulation (FDTD) to best meet a required antenna performance specification---gain, bandwidth, polarization, pattern, and so on \cite{MaloneyKeslerHarms}--\cite{BalanisHB12}.  Visual inspection of the optimized designs revealed that the metallic pixels formed many connected and disconnected fragments across the aperture, which led to the name \emph{Fragmented Aperture Antennas} for this new class of antennas.  A detailed description of the original design approach is disclosed in the original patent \cite{MaloneyFragPatent}.  Since the original publications, other research groups have successfully applied the fragmented aperture design approach to their own applications \cite{Herscovici}--\cite{EllgardtPersson}, including the use of genetic algorithms for broadband fragmented aperture phased array design \cite{Thors2005Broad-band} and the development of fragmented antennas based on coupled small radiating elements \cite{Barani2018Fragmented}.

However, the original fragmented design approach suffers from two significant deficiencies.  First, the placement of pixels on a generalized rectilinear grid leads to the problem of \emph{diagonal touching}---pixels that share only a corner vertex are electrically connected in the numerical model but may be disconnected when fabricated.  Other research groups have also encountered this diagonal touching problem \cite{EllgardtThesis}.  Second, the convergence of the GA becomes increasingly poor as the pixel count grows large ($\gg 100$).  Recent work has explored various optimization strategies to address convergence challenges, including the use of integrated baluns to simplify the design space \cite{Zang2019Optimum} and automated optimization techniques \cite{Li2019Automated}.

This chapter presents solutions to both of these shortcomings.  First, alternate approaches to the discretization of the aperture area that inherently avoid diagonal touching are presented.  Second, an improved mutation operator tailored for fragmented aperture design that significantly improves convergence for large pixel counts is introduced.

\section{The Diagonal Touching Problem}
\label{sec:diagtouch}

Originally, fragmented aperture antennas were envisioned as a planar surface partitioned into a grid of rectangular pixels, each either conducting or non-conducting, as shown in Figure~\ref{fig:IFAF1}.  A genetic algorithm and a computational electromagnetic model determined which pixels should be conducting and which should be non-conducting to form an antenna surface suitable for a given application.  This concept was generalized to parallelogram-shaped pixels, as shown in Figure~\ref{fig:IFAF2}, in the original fragmented aperture patent \cite{MaloneyFragPatent}.

\begin{figure}
\includegraphics[angle=0,width=\linewidth]{/Users/jim.maloney/Book/images/improvedPatentFig1.png}
\caption{Original fragmented aperture approach based on a lattice of rectangular pixels.  An example of diagonal touching is shown in the top right of the figure.}
\label{fig:IFAF1}
\end{figure}

As discussed in Chapter~2, the rectangular pixel approach was very successfully used to design novel antennas \cite{MaloneyKeslerHarms}--\cite{BalanisHB12}.  However, many of these designs proved troublesome to fabricate and measure.  The primary problem is \emph{diagonal touching} of pixels, as illustrated in the top right of Figures~\ref{fig:IFAF1} and~\ref{fig:IFAF2}.

\begin{figure}
\includegraphics[angle=0,width=\linewidth]{/Users/jim.maloney/Book/images/improvedPatentFig2.png}
\caption{Generalized fragmented aperture approach based on parallelogram-shaped pixels.  Again, an example of diagonal touching is shown in the top right of the figure.}
\label{fig:IFAF2}
\end{figure}

Diagonal touching is not a problem during the design phase because in the FDTD numerical model, diagonally adjacent conducting pixels are always electrically connected.  However, when fabricated using processes such as printed circuit board etching, the diagonal contact is often broken due to over-etching, as illustrated in Figure~\ref{fig:IFAF3}(a).  Disconnecting metal that should be connected can seriously degrade the antenna's impedance match and gain characteristics.

\begin{figure}
\includegraphics[angle=0,width=\linewidth]{/Users/jim.maloney/Book/images/improvedPatentFig3.png}
\caption{(a) Over-etching causing diagonal elements not to touch, as shown in the top center. (b) Close-up photograph of an etched copper fragmented antenna showing over-etch disconnecting diagonal fragments.}
\label{fig:IFAF3}
\end{figure}

Other researchers have also observed the difficulties caused by diagonal touching.  A close-up photograph of an etched copper fragmented surface is shown in Figure~\ref{fig:IFAF3}(b), where the disconnection of diagonally adjacent pixels is clearly visible \cite{Ellgardt2006Characteristics}.

In fact, nearly every antenna design included in the original fragmented aperture patent suffers from the diagonal touching problem.  Figure~\ref{fig:IFAF4} shows a few examples from U.S.\ Patent 6,323,809 \cite{MaloneyFragPatent}, with the most troublesome diagonal-touching locations near the antenna feed circled.

\begin{figure}
\includegraphics[angle=0,width=\linewidth]{/Users/jim.maloney/Book/images/improvedPatentFig4.png}
\caption{Two sample designs from the original fragmented aperture patent exhibiting diagonal touching \cite{MaloneyFragPatent}.  The most troublesome examples of diagonal touching near the antenna feed are circled.}
\label{fig:IFAF4}
\end{figure}

The root cause of the problem lies in the fundamental approach to area partitioning.  When the pixel edges are parallel to the lattice forming vectors, as in Figures~\ref{fig:IFAF1} and~\ref{fig:IFAF2}, the issue of diagonal touching is unavoidable.  The next two sections present solutions: Section~\ref{sec:mitigation} describes several practical mitigation strategies, while Section~\ref{sec:IFAS4} presents three improved pixel geometries that inherently eliminate diagonal touching.

\section{Mitigation Strategies for Diagonal Touching}
\label{sec:mitigation}

Before presenting the improved pixel geometries, it is instructive to review several practical strategies that have been used over the years to mitigate diagonal touching in designs based on the original rectangular pixel approach.  Four such strategies are illustrated in Figure~\ref{fig:IFAF5}.

\begin{figure}
\includegraphics[angle=0,width=\linewidth]{/Users/jim.maloney/Book/images/improvedPatentFig5.png}
\caption{Four approaches to mitigating diagonal touching: (a) super-cell approach using $3\times 3$ plus signs, (b) fabricating each pixel approximately 10\% larger than designed, (c) placing a small metal square at diagonal contact points to ensure connection, and (d) random coin-flip approach to resolve diagonal ambiguity.}
\label{fig:IFAF5}
\end{figure}

\subsection{Super-Cell Approach}

One approach that was successfully exploited is the super-cell strategy shown in Figure~\ref{fig:IFAF5}(a).  A super-cell is a collection of smaller sub-elements---for example, a $3 \times 3$ grid of sub-pixels.  To avoid diagonal touching, the conducting area is defined as the five sub-elements that form a plus sign, leaving the four corner sub-elements always non-conducting.  This guarantees that no diagonal touching can occur.

The super-cell approach successfully produced antennas with good correlation between measurement and model.  However, restricting the conducting area to plus-sign shapes forces electrical currents to flow only in directions aligned with the grid, which over-constrains the design and can lead to suboptimal antenna performance.

\subsection{Oversized Fabrication}

Another successful approach was to intentionally fabricate every pixel approximately 10\% larger than designed, as illustrated in Figure~\ref{fig:IFAF5}(b).  This enlargement ensures that diagonally adjacent pixels overlap slightly, guaranteeing electrical contact.  This strategy was found to yield a high percentage of successfully fabricated antennas.

However, oversized fabrication results in the antenna having roughly 10--20\% more conductor than originally designed, which can alter the antenna characteristics from the design predictions.  It is worth noting that fabricating pixels 10\% \emph{smaller} would guarantee that diagonal pixels never touch, but this would mean that conducting regions could never extend beyond a single pixel---a condition that would render the antenna virtually useless.  Moreover, smaller fabrication would be inconsistent with the FDTD model used during design, in which diagonally adjacent conducting pixels are always connected.

\subsection{Metal Bridge and Coin-Flip Approaches}

Other research groups have developed additional strategies.  Ellgardt and Persson \cite{EllgardtPersson} considered both the oversized pixel strategy of Figure~\ref{fig:IFAF5}(b) and a variant in which a small square of metal is placed at each diagonal contact point to ensure connection, as shown in Figure~\ref{fig:IFAF5}(c).  Rahmat-Samii and colleagues at UCLA have used a random coin-flip process to resolve diagonal ambiguity: when two non-conducting pixels are diagonally adjacent to two conducting pixels, a random selection determines which non-conducting pixel is made conducting to complete the connection, as illustrated in Figure~\ref{fig:IFAF5}(d).

While each of these mitigation strategies can be effective, they all represent workarounds for a fundamental problem.  The proper solution is to change the pixel geometry itself so that diagonal touching cannot occur, as described in the next section.

\section{Three Improved Pixel Geometries}
\label{sec:IFAS4}

The proper approach to eliminating diagonal touching is to break the dependence between pixel edges and lattice directions that is implicit in the rectangular and parallelogram geometries of Figures~\ref{fig:IFAF1} and~\ref{fig:IFAF2}.  The following three subsections present three different pixel geometries that accomplish this, each leading to improved fragmented aperture antennas.  Recent studies have examined important design considerations such as pixel size, symmetry constraints, and their effects on antenna performance \cite{Mair2024Design}.

\subsection{First Approach: Skewed Lattice}

In the first approach, the individual conducting/non-conducting elements are defined using lattice vectors that are not both parallel to the element edges, as illustrated in Figure~\ref{fig:IFAF6}.  The example shown uses square elements arranged on a skewed lattice with a skew angle of $\psi \approx 63.4^\circ$.

\begin{figure}
\includegraphics[angle=0,width=\linewidth]{/Users/jim.maloney/Book/images/improvedPatentFig6.png}
\caption{First approach to improved fragmented aperture antennas: square elements on a skewed lattice.  Skewing the lattice vector $\vec{V}_2$ eliminates the possibility of diagonal touching.}
\label{fig:IFAF6}
\end{figure}

Notice that skewing the lattice vector $\vec{V}_2$ completely eliminates the possibility of diagonal touching.  Every adjacent pixel pair shares a full edge, so the electrical connection between adjacent conducting pixels is always robust and unambiguous.

\subsection{Second Approach: Alternating Pixel Shapes}

In the second approach, two complementary pixel shapes are used that alternate across the aperture, as illustrated in Figure~\ref{fig:IFAF7}.  The shapes are chosen so that adjacent pixels always share a definite edge rather than merely a corner point.

\begin{figure}
\includegraphics[angle=0,width=\linewidth]{/Users/jim.maloney/Book/images/improvedPatentFig7.png}
\caption{Second approach to improved fragmented aperture antennas: alternating pixel shapes that tessellate the plane while ensuring definite edge contact between adjacent pixels.}
\label{fig:IFAF7}
\end{figure}

The key requirement is that the two shapes together tessellate the plane---that is, they tile the aperture surface without gaps or overlaps.  When this condition is met and the shapes are designed so that adjacency always involves a shared edge, diagonal touching is inherently impossible.

\subsection{Third Approach: Single Self-Tessellating Shape}

In the third approach, a single pixel shape is chosen that tessellates the plane by itself while ensuring that adjacent pixels share edges rather than corner points.  Figure~\ref{fig:IFAF8} shows one example of such a shape, although many other shapes satisfying these requirements exist.

\begin{figure}
\includegraphics[angle=0,width=\linewidth]{/Users/jim.maloney/Book/images/improvedPatentFig8.png}
\caption{Third approach to improved fragmented aperture antennas: a single self-tessellating pixel shape that tiles the aperture surface while ensuring definite edge contact between all adjacent pixels.}
\label{fig:IFAF8}
\end{figure}

The advantage of a single-shape tessellation is simplicity of implementation: only one pixel geometry needs to be defined and manufactured.  However, depending on the chosen shape, this approach may not naturally support the left-right and top-bottom symmetry constraints that are commonly imposed for broadside, linearly polarized antenna designs.  This makes the third approach particularly well suited for designs where symmetry is not required, such as beam-steered or circularly polarized antennas.

\section{Improved Mutation Algorithm for Better Convergence}
\label{sec:mutation}

In addition to the diagonal touching problem, the original fragmented aperture design approach suffered from poor convergence of the genetic algorithm when the number of pixels became large.  This section describes an improved mutation operator that addresses this limitation.  More recent work has explored alternative optimization approaches including topology optimization with level set methods \cite{Howard2024Topology}, machine learning and neural network techniques \cite{Li2025Inverse,Wang2024Machine,Jacobs2022Accurate}, and reinforcement learning \cite{Wang2024Bandwidth}; however, genetic algorithms remain popular due to their simplicity and effectiveness for pixelated antenna design \cite{Mair2022Evolutionary,Zeghdoud2025Accelerated}.

In a standard genetic algorithm, mutation is a random process in which a small number of genes are changed each generation to help the algorithm avoid convergence to suboptimal solutions.  For a fragmented aperture antenna, mutation randomly toggles a few pixels between conducting and non-conducting states.  However, many of these random mutations produce only an isolated metal pixel in a non-conducting region or a small hole in an otherwise solid metal area.  Such changes have negligible effect on the antenna performance, and the mutation step is effectively wasted.

The improved mutation algorithm is biased so that mutations preferentially extend the boundaries of existing conducting fragments into empty regions, or enlarge holes within large metal regions.  This is accomplished using an adjacency matrix that describes which pixels are in contact with each other.  By analyzing the local neighborhood of each pixel, the algorithm identifies pixels at the boundaries between conducting and non-conducting regions and preferentially selects these boundary pixels for mutation.

To demonstrate the effectiveness of this adjacency-based mutation strategy, three independent design trials were conducted with the traditional mutation algorithm and three with the improved mutation algorithm.  Figure~\ref{fig:IFAF9} shows the convergence of the fitness score as a function of generation count, averaged over the three trials for each algorithm.  The improved mutation algorithm (green line) converges to a better fitness score in fewer generations than the traditional approach (blue line).

\begin{figure}
\includegraphics[angle=0,width=\linewidth]{/Users/jim.maloney/Book/images/improvedPatentFig9.png}
\caption{Convergence comparison between the traditional mutation algorithm (blue) and the improved adjacency-based mutation algorithm (green), averaged over three independent design trials.  The improved algorithm converges to a better fitness score in fewer generations.}
\label{fig:IFAF9}
\end{figure}

As shown in Table~\ref{tab:IFAT1}, the improved mutation algorithm produced a better result than the traditional algorithm in every one of the three trials.  The best trial with the improved algorithm achieved a fitness score of $-1.684$, compared to $-2.238$ for the best trial with the traditional algorithm---an improvement of more than 0.5~dB.

\begin{table}
\caption{Fitness score comparison across three convergence trials for the traditional and improved mutation algorithms (higher score is better).}
\includegraphics[angle=0,width=\linewidth]{/Users/jim.maloney/Book/images/improvedPatentFig16.png}
\label{tab:IFAT1}
\end{table}

The results in Table~\ref{tab:IFAT1} also illustrate an important practical point: when using an evolutionary algorithm to design an antenna, more than one design trial should always be executed.  As these results show, the variation between trials can exceed 1~dB, so running multiple trials and selecting the best result is essential for obtaining high-quality designs.

\section{Sample Improved Fragmented Aperture Designs}

This section presents sample antenna designs produced using the improved pixel geometries described in Section~\ref{sec:IFAS4}.  All designs use the improved mutation algorithm of Section~\ref{sec:mutation}.

\subsection{First Approach Designs}

The skewed-lattice approach of Figure~\ref{fig:IFAF6} was used to design a series of fragmented aperture antennas spanning from 500~MHz to 2.0~GHz.  The lattice skew angle was chosen to be $\psi = \tan^{-1}(2) \approx 63.4^\circ$, which provides the left-right physical symmetry needed for linearly polarized broadside designs.  The square pixels were 10.8~mm on a side, and the total aperture area was 25.4~cm $\times$ 25.4~cm.  Each antenna was excited at a terminal pair at the center of the aperture with a $100~\Omega$ transmission line.

The aperture designs were performed using a genetic algorithm with FDTD evaluation of each candidate antenna (see Appendix~A for details on FDTD modeling of antennas).  For these designs, the 25.4~cm $\times$ 25.4~cm aperture contained 663 individual pixels.  Enforcing left-right and top-bottom symmetry reduced the number of independent degrees of freedom to 169.  With a single bit representing the state of each pixel (1 = conducting, 0 = non-conducting), this yields a 169-bit genetic code.  Using a population size of 32 antennas, approximately 100 GA generations were required to produce each design.  The fitness function rewarded good impedance match (return loss better than 15~dB) and maximum broadside realized gain.

Four representative aperture designs are shown in Figure~\ref{fig:IFAF10}.  The physical shapes clearly demonstrate that none of the designs suffer from diagonal touching---every connection between adjacent conducting pixels involves a full shared edge.

\begin{figure}
\includegraphics[angle=0,width=\linewidth]{/Users/jim.maloney/Book/images/improvedPatentFig10.png}
\caption{Four sample fragmented aperture designs produced using the skewed-lattice (First Approach) pixel geometry.  None of the designs exhibit diagonal touching.}
\label{fig:IFAF10}
\end{figure}

Figure~\ref{fig:IFAF11} shows the broadside realized gain of each design as a function of frequency.  The gains are compared with the aperture gain limit (black line), which for these ground-plane-free apertures is $2\pi A/\lambda^2$.  All four designs approach the aperture gain limit within their respective design bands.

\begin{figure}
\includegraphics[angle=0,width=\linewidth]{/Users/jim.maloney/Book/images/improvedPatentFig11.png}
\caption{Broadside realized gain for the four skewed-lattice designs shown in Figure~\ref{fig:IFAF10}.  The black line indicates the aperture gain limit $2\pi A/\lambda^2$.}
\label{fig:IFAF11}
\end{figure}

Figure~\ref{fig:IFAF12} shows the VSWR of each design.  The VSWR remains below 1.5 across the respective design bands, consistent with the fitness function requirement of return loss better than 15~dB.

\begin{figure}
\includegraphics[angle=0,width=\linewidth]{/Users/jim.maloney/Book/images/improvedPatentFig12.png}
\caption{VSWR for the four skewed-lattice designs shown in Figure~\ref{fig:IFAF10}.}
\label{fig:IFAF12}
\end{figure}

\subsection{Second Approach Designs}

The alternating-shape approach of Figure~\ref{fig:IFAF7} was also used to design fragmented aperture antennas.  This pixel geometry naturally supports both left-right and top-bottom symmetry when required.  The aperture area was again 25.4~cm $\times$ 25.4~cm, excited at the center with a $100~\Omega$ feed.  The aperture contained 841 shaped pixels, and with both symmetries enforced, the number of independent degrees of freedom was 221.

\textcolor{red}{\textbf{[Note: The Second Approach currently has two sample designs.  Four designs are planned to match the First Approach presentation.  Figures and text to be updated.]}}

Figure~\ref{fig:IFAF13} shows two sample designs for the 0.5--0.8~GHz and 0.8--1.2~GHz bands.  As with the first approach designs, no diagonal touching is present in the physical structures.

\begin{figure}
\includegraphics[angle=0,width=\linewidth]{/Users/jim.maloney/Book/images/improvedPatentFig13.png}
\caption{Two sample fragmented aperture designs produced using the alternating-shape (Second Approach) pixel geometry for the 0.5--0.8~GHz and 0.8--1.2~GHz bands.}
\label{fig:IFAF13}
\end{figure}

Figures~\ref{fig:IFAF14} and~\ref{fig:IFAF15} summarize the broadside gain and VSWR for these two designs.  The performance is consistent with the design objectives, with gain approaching the aperture limit within each design band and VSWR remaining below 1.5.

\begin{figure}
\begin{center}
\includegraphics[angle=0,scale=0.2]{/Users/jim.maloney/Book/images/improvedPatentFig14.png}
\caption{Broadside realized gain for the two alternating-shape designs shown in Figure~\ref{fig:IFAF13}.}
\label{fig:IFAF14}
\end{center}
\end{figure}

\begin{figure}
\begin{center}
\includegraphics[angle=0,scale=0.2]{/Users/jim.maloney/Book/images/improvedPatentFig15.png}
\caption{VSWR for the two alternating-shape designs shown in Figure~\ref{fig:IFAF13}.}
\label{fig:IFAF15}
\end{center}
\end{figure}

\subsection{Third Approach Designs}

The self-tessellating shape approach of Figure~\ref{fig:IFAF8} can also be used to design fragmented aperture antennas.  As noted earlier, for vertically or horizontally polarized elements with a broadside beam, the lack of inherent left-right and top-bottom symmetry in many self-tessellating shapes is a drawback.  However, for applications where the desired beam direction is not broadside or the desired polarization is circular or slant-linear, symmetry constraints are not needed and the third approach is fully competitive with the first and second.

\textcolor{red}{\textbf{[Missing content: Third Approach sample designs are needed.  Planned designs include: (a) broadside vertically polarized, (b) broadside horizontally polarized, (c) broadside slant-linear polarized, and (d) vertically polarized beam steered $45^\circ$ from broadside.  Corresponding gain, VSWR, and azimuth pattern figures are also needed.]}}

%\begin{figure}
%\includegraphics[angle=0,width=\linewidth]{/Users/jim.maloney/Book/images/improvedFragFig16.png}
%\caption{Four sample designs from the self-tessellating shape (Third Approach): (a) broadside vertically polarized, (b) broadside horizontally polarized, (c) broadside slant-linear polarized, (d) vertically polarized beam steered $45^\circ$ from broadside.}
%\label{fig:IFAF16}
%\end{figure}

%\begin{figure}
%\includegraphics[angle=0,width=\linewidth]{/Users/jim.maloney/Book/images/improvedFragFig17.png}
%\caption{Broadside realized gain for the four self-tessellating shape designs.}
%\label{fig:IFAF17}
%\end{figure}

%\begin{figure}
%\includegraphics[angle=0,width=\linewidth]{/Users/jim.maloney/Book/images/improvedFragFig18.png}
%\caption{VSWR for the four self-tessellating shape designs.}
%\label{fig:IFAF18}
%\end{figure}

%\begin{figure}
%\includegraphics[angle=0,width=\linewidth]{/Users/jim.maloney/Book/images/improvedFragFig19.png}
%\caption{Azimuth gain pattern at midband for the four self-tessellating shape designs.}
%\label{fig:IFAF19}
%\end{figure}


\FloatBarrier

% Bibliography for Chapter 3
% Uses chapter-specific .bib files organized by topic
\bibliography{../Literature/master_bibliography,%
              ../Literature/fragmented_aperture_core,%
              ../Literature/ml_ai_optimization_methods}
\bibliographystyle{IEEEtran}
			% ch3
\chapter{Sample Antenna Design}
%\authortoc{James G. Maloney}
%\chapterauthor{James G. Maloney}

\section{feed strategies}

\begin{itemize}
\item{feed strategies}
\item{in-plane self-balun}
\item{in-Plane twin feed}
\item{unbalanced fed thru ground plane}
\item{balanced fed thru ground plane}
\item{tailor bandwidth}
\item{fixed steering}
\item{broadside}
\item{forward and backward 45} 
\item{end fire}
\item{Polarization}
\subitem{linear}
\subitem{circular}
\subitem{non-broadside cp}
\item{tailor beamwidth}
\item{out-band-rejection}
\end{itemize}


\section{First Success}
\label{sec:first}	% specifying a section label to be used by a \ref
\cite{Maloney1}.

\begin{figure}
\includegraphics[angle=0,width=\linewidth]{/Users/jim.maloney/Book/images/origPatentFig3.png}
\caption{Comparison of fragmented design to uniform aperture limit $2\pi A/\lambda^2$}
\label{fig:IFAF1}		% figure references have to be below the \caption line
\end{figure}

Figure \ref{fig:ACF1} is a schematic drawing showing a typical volume in which Maxwell's equations are to be solved. The volume is divided into unit cells each of volume. 

\begin{thebibliography}{99}

 \bibitem{Yee66} K. S. Yee, ``Numerical Solution of Initial Boundary Value Problems Involving Maxwell's Equations in Isotropic Media,'' IEEE Trans. Antennas Propagat., Vol. AP-14, pp. 302-307, May 1966.

\bibitem{Maloney1} J. G. Maloney, G. S. Smith, and W. R. Scott, Jr., ``Accurate Computation of the Radiation from Simple Antennas Using the Finite-Difference Time-Domain Method,'' IEEE Trans. Antennas Propagat., Vol. AP-38, pp. 1059-1068, July 1990.

\bibitem{BoonPist} J. J. Boonzaaier and C. W. Pistorius, ``Thin Wire Dipoles ? A Finite-Difference Time-Domain Approach,'' Electronics Lett., Vol. 26, pp. 1891-1892, 25 October, 1990.

\bibitem{KatzHorn} D. S. Katz, M. J. Picket-May, A. Taflove, and K. R. Umashankar, ``FDTD Analysis of Electromagnetic Wave Radiation from Systems Containing Horn Antennas,'' IEEE Trans. Antennas Propagat., Vol. AP-39, pp. 1203-1212, August 1991.

\end{thebibliography}


		% ch4
\chapter{Reconfigurable Fragmented Aperture Antennas}

\section{Introduction}

In the preceding chapters, we have shown how the fragmented aperture concept can be used to design antennas that meet particular performance specifications.  A planar sheet of conductor is divided into many sub-wavelength pixels, and a genetic algorithm working with an FDTD simulation determines which pixels should be conducting and which should not.  Different designs can be obtained to meet different specifications: for example, one design might produce an antenna with a broadside beam optimized for a particular bandwidth, while a second design might produce an antenna with the beam steered to $45^o$ from broadside.

Of course, once a fragmented aperture antenna is fabricated, it can only meet one set of specifications---either the broadside design or the steered design, but not both.  It would be enormously useful if a single fragmented aperture could be electronically switched between different configurations to meet different requirements on demand.  This would require a mechanism for dynamically changing individual pixels from conducting to non-conducting and vice versa.  Recent work has demonstrated various reconfigurable pixelated antenna implementations using MEMS switches \cite{Wright2018Mems,Ali2014Mems}, phase transition materials \cite{Lou2025Frequency}, and magnetically actuated mechanisms \cite{Pal2018Magnetically}.

One can imagine, for example, making the cladding on a circuit board from a photoconductive material and using a laser to selectively illuminate the pixels that need to be conducting for a particular design.  Changing the illumination pattern would reconfigure the antenna.  While this particular approach remains impractical with current technology, the underlying idea---a reconfigurable fragmented aperture---motivated the development of the Agile Aperture Antenna described in this chapter.

\section{The Agile Aperture Antenna Concept}

The concept of a reconfigurable fragmented aperture antenna was first published in 2000 under the name ``switched fragmented aperture antenna'' \cite{MaloneyRECAP}.  Subsequently, DARPA funded a solicitation for ``reconfigurable aperture'' antennas and coined the acronym RECAP.  To distinguish the fragmented aperture approach from other reconfigurable antenna concepts, we later adopted the term ``Agile Aperture Antenna'' (A3), emphasizing that the purpose of reconfiguration is to make the antenna \emph{agile}---able to dynamically change its frequency of operation, beam direction, polarization, or other characteristics.  The seminal IEEE paper on this work \cite{Pringle2004Reconfigurable} and a comprehensive book chapter \cite{Balanis2007ReconfigChapter} provide detailed treatments of reconfigurable aperture antenna technology.

The Agile Aperture Antenna implementation that was successfully demonstrated is shown schematically in Figure~\ref{fig:RECAP1} \cite{RECAP}.  A thin dielectric substrate supports an array of square metallic pads.  The pads are electrically small, with side length $l$ satisfying $l/\lambda_o \ll 1$, where $\lambda_o$ is the free-space wavelength at the operating frequency.  Each pad is connected to its neighboring pads by switched links, indicated by the arrows in the figure.  Each switch may be independently set to open or closed depending on the desired antenna configuration.  A single feed point (pair of terminals) is located near the center of the antenna.

\begin{figure}\begin{center}
\includegraphics[angle=0,scale=0.7]{/Users/jim.maloney/Book/images/RECAP1.png}
\caption{Schematic drawing of the Agile Aperture Antenna in dipole form.  Square metallic pads are connected by switched links (arrows).  The state of each switch (open or closed) determines the antenna configuration \cite{RECAP}.}
\label{fig:RECAP1}
\end{center}\end{figure}

The Agile Aperture Antenna can be understood as a variant of the fragmented aperture antenna in which the fundamental unit is not a single pixel but a metallic pad composed of a group of pixels.  The pads are not contiguous; they are separated by narrow dielectric gaps.  The antenna structure for any given configuration consists of the conducting pads that are connected by closed switches, together with all of the unconnected pads that remain present on the substrate.  This is an important distinction from a conventional fragmented aperture: in the Agile Aperture Antenna, the unconnected pads are always physically present and contribute to the electromagnetic behavior of the antenna through scattering, even when they are not part of the connected conducting structure.

\section{Static Proof of Concept}

\textcolor{red}{INSERT: Description of the first two static (hard-wired) pixelated designs that demonstrated the unconnected pads did not prevent good antenna performance.  Include figures showing the two designs and their measured performance.}

\section{Reconfigurable Proof of Concept}

To prove the validity of the Agile Aperture Antenna concept, a detailed study was conducted using a prototype antenna with hard-wired switches---gaps that were either closed by a soldered wire or left open.  This study not only validated the design approach but also identified areas where future research would be needed to extend the concept to practical, electronically reconfigurable antennas.

\subsection{Prototype Description}

For all of the antennas discussed in this section, the aperture was formed from a printed circuit board 22.5~cm $\times$ 22~cm in size, with the pads etched from the copper cladding on one side of the board.  The pad side length and spacing were both $l = s = 1.0$~cm (see Figure~\ref{fig:RECAP1}).  The board contained a total of 120 pads and 208 switches.  The dielectric substrate was 1.7~mm thick FR4 circuit board with measured electrical properties $\varepsilon_r = 4.27$ and $\tan\delta = 0.07$ \textcolor{red}{(verify loss tangent value)}.

The frequency of operation was in the range $0.85~\text{GHz} \leq f \leq 1.45~\text{GHz}$, so the pads were electrically small: $0.028 \leq l/\lambda_o \leq 0.048$.  All of the antenna designs described in this section have mirror symmetry about the horizontal line through the feed point, including the states of the switches.  This symmetry allows the antennas to be analyzed and measured in either the ``dipole form'' shown in Figure~\ref{fig:RECAP1} or the ``monopole form'' shown in Figure~\ref{fig:RECAP2}.

\subsection{Measurement Setup}

\begin{figure}\begin{center}
\includegraphics[angle=0,scale=0.7]{/Users/jim.maloney/Book/images/RECAP2.png}
\caption{Experimental arrangement for measuring the Agile Aperture Antenna in monopole form.  The antenna is mounted vertically on a rotatable disc centered in a large metallic image plane \cite{RECAP}.}
\label{fig:RECAP2}
\end{center}\end{figure}

Figure~\ref{fig:RECAP2} shows the experimental setup used for all of the measurements reported in this section.  The monopole version of the Agile Aperture Antenna was mounted vertically on a rotatable disc centered in a large metallic image plane \textcolor{red}{(insert image plane dimensions)}.  The antenna was fed from below the image plane by a 50~$\Omega$ coaxial line, with the center conductor connected to the bottom pad in the center column of the antenna.  A calibrated TEM horn antenna was located at a distance of \textcolor{red}{(insert distance)} from the antenna \cite{RECAP}.  The scattering parameters for the two-port network formed by the Agile Aperture Antenna and the TEM horn were measured with a network analyzer and used to determine the absolute gain of the Agile Aperture Antenna \cite{LeeSmith2004}.  The horizontal radiation pattern ($|E|$ versus azimuth angle $\phi$) was obtained by rotating the disc while recording the output signal from the horn.

\subsection{Design Procedure}

The procedure for designing a switch configuration for the Agile Aperture Antenna is conceptually the same as for a conventional fragmented aperture: a rigorous FDTD simulation of the antenna is run in conjunction with a genetic algorithm optimizer (see Appendix~A for an introduction to the FDTD method).  In all of the FDTD simulations reported here, cubical Yee cells with a side length of 2.5~mm were used.  More recent optimization approaches for reconfigurable pixelated antennas include beam steering techniques \cite{Towfiq2018Reconfigurable} and quantum genetic algorithms \cite{Bichara2021Miniaturized}.

A performance goal is first established---for example, maximum broadside realized gain over a specified bandwidth.  The GA then searches for the switch configuration (which switches should be open and which should be closed) that best meets this goal.  Taking into account the mirror symmetry of the antenna, there are $2^{104} \approx 2 \times 10^{31}$ possible switch configurations---far too many to evaluate exhaustively.  The GA provides an efficient, though approximate, method for searching this enormous design space.

\subsection{Broadside Design}

The design goal for the first example was to maximize the broadside realized gain over the frequency range $0.85~\text{GHz} \leq f \leq 1.25~\text{GHz}$ (a fractional bandwidth of 38\%).  The target was that the realized gain should equal or exceed the directivity of a uniform sheet of vertically directed current occupying the same aperture area.

Figure~\ref{fig:RECAP3}(a) shows the switch configuration obtained by the GA for this broadband, bidirectional, broadside design.  Notice that this configuration has right-left symmetry in addition to the imposed top-bottom symmetry; all of the broadside designs discussed in this chapter are constrained to have this additional symmetry.

\begin{figure}\begin{center}
\includegraphics[angle=0,scale=0.7]{/Users/jim.maloney/Book/images/RECAP3.png}
\caption{Switch configurations for the Agile Aperture Antenna (monopole form) with hard-wired switches.  (a) Broadband, bidirectional, broadside design.  (b) Narrowband, unidirectional, end-fire design.  The two configurations are strikingly different, yet both are realized on the same physical antenna \cite{RECAP}.}
\label{fig:RECAP3}
\end{center}\end{figure}

Figure~\ref{fig:RECAP4}(a) shows the realized gain versus frequency for this design.  The dashed blue line is the design goal (uniform aperture directivity), the solid black line is the FDTD simulation, and the red line with markers is the measured result.  All realized gain values are for the antenna in the dipole configuration.  The simulated and measured realized gains are in good agreement over the design bandwidth, with a maximum difference of approximately 1~dB.  The realized gain falls approximately 0.5--1.5~dB below the goal; a portion of this difference is attributable to impedance mismatch at the antenna feed.

Figure~\ref{fig:RECAP4}(b) shows the mismatch factor $(1 - |\Gamma_A|^2)$ as a function of frequency, where $\Gamma_A$ is the voltage reflection coefficient at the antenna terminals.  Within the design bandwidth, this factor ranges from 0.0~dB to $-1.5$~dB, confirming that mismatch accounts for a significant portion of the difference between the realized gain and the goal.

\begin{figure}\begin{center}
\includegraphics[angle=0,scale=0.7]{/Users/jim.maloney/Book/images/RECAP4.png}
\caption{Results for the broadband, bidirectional, broadside design with hard-wired switches.  (a) Realized gain versus frequency.  (b) Gain reduction due to impedance mismatch: $(1-|\Gamma_A|^2)$ \cite{RECAP}.}
\label{fig:RECAP4}
\end{center}\end{figure}

Figure~\ref{fig:RECAP5} shows the horizontal radiation pattern at the center frequency $f = 1.05$~GHz.  The simulated and measured patterns are nearly identical, with both normalized to a maximum of 0~dB.  The heavy line at the center of the pattern indicates the orientation of the dielectric substrate.

\begin{figure}\begin{center}
\includegraphics[angle=0,scale=0.5]{/Users/jim.maloney/Book/images/RECAP5.png}
\caption{Horizontal radiation pattern at $f = 1.05$~GHz for the broadband, bidirectional, broadside design with hard-wired switches.  Both patterns are normalized to 0~dB \cite{RECAP}.}
\label{fig:RECAP5}
\end{center}\end{figure}

\subsection{End-Fire Design}

To demonstrate the versatility of the Agile Aperture Antenna concept, a second design was produced for a completely different objective: a narrowband, unidirectional, end-fire beam over the frequency range $1.0~\text{GHz} \leq f \leq 1.1~\text{GHz}$ (a fractional bandwidth of 9.5\%).  The goal was again that the realized gain should equal or exceed the directivity of a uniform sheet of current, but now with the current phased to produce end-fire radiation.

Figure~\ref{fig:RECAP3}(b) shows the switch configuration for this end-fire design.  It is immediately apparent that this configuration is strikingly different from the broadside configuration in Figure~\ref{fig:RECAP3}(a)---the end-fire design does not have right-left symmetry and produces a fundamentally different current distribution on the aperture.  Yet both configurations are realized on the same physical hardware simply by changing which switches are open and which are closed.

Figure~\ref{fig:RECAP6}(a) shows the realized gain versus frequency for the end-fire design.  The simulated and measured results are again in good agreement over the design bandwidth, with a maximum difference of approximately 1~dB.  The realized gain falls approximately 1.0--2.0~dB below the goal.  The mismatch factor shown in Figure~\ref{fig:RECAP6}(b) is within the range 0.0~dB to $-0.8$~dB over the design bandwidth.

\begin{figure}\begin{center}
\includegraphics[angle=0,scale=0.6]{/Users/jim.maloney/Book/images/RECAP6.png}
\caption{Results for the narrowband, unidirectional, end-fire design with hard-wired switches.  (a) Realized gain versus frequency.  (b) Gain reduction due to impedance mismatch: $(1-|\Gamma_A|^2)$ \cite{RECAP}.}
\label{fig:RECAP6}
\end{center}\end{figure}

Figure~\ref{fig:RECAP7} shows the horizontal radiation pattern at $f = 1.05$~GHz.  The simulated and measured patterns are in excellent agreement, and both clearly show the characteristic end-fire beam directed to one side of the antenna.

\begin{figure}\begin{center}
\includegraphics[angle=0,scale=0.6]{/Users/jim.maloney/Book/images/RECAP7.png}
\caption{Horizontal radiation pattern at $f = 1.05$~GHz for the narrowband, unidirectional, end-fire design with hard-wired switches.  Both patterns are normalized to 0~dB \cite{RECAP}.}
\label{fig:RECAP7}
\end{center}\end{figure}

\subsection{Observations on the Designed Configurations}

In the configurations studied, approximately 30\% to 60\% of the switches were closed.  One might expect that examination of the switch states for a particular design would reveal recognizable antenna structures---for example, the end-fire design might show strings of pads forming linear elements arranged like the driven element, reflector, and director of a Yagi-Uda array.  However, as seen in Figure~\ref{fig:RECAP3}(b), this is not the case.  In general, there is no simple, discernible relationship between the switch states and the design goal.

This lack of an obvious physical interpretation is consistent with the experience from conventional fragmented aperture design (Chapter~2): the GA discovers complex, non-intuitive structures that exploit the full electromagnetic physics of the problem.  In the case of the Agile Aperture Antenna, the optimization is further complicated by the presence of the unconnected pads, which scatter electromagnetic energy and must be accounted for in the design.  It is clear, however, that the switch connections near the feed point are often arranged to improve the impedance match between the antenna and the transmission line.

\section{Discussion}

The broadside and end-fire examples presented above, along with several other designs not shown, demonstrate that the Agile Aperture Antenna concept is viable: a single physical antenna can be reconfigured via its switch states to meet fundamentally different performance specifications.  The excellent agreement between FDTD simulations and measurements further validates the design procedure.

However, for the Agile Aperture Antenna to transition from a laboratory concept to a practical technology, several challenges must be addressed.  Most critically, a switch technology must be developed that can be electronically controlled without interfering with the electromagnetic performance of the antenna.  The switches must introduce minimal insertion loss when closed, provide high isolation when open, and the control circuitry (bias lines, drivers) must not distort the antenna's radiation characteristics.

Subsequent research has explored several promising switch technologies for reconfigurable pixelated antennas.  MEMS-based switches have been successfully demonstrated for reconfigurable patch antennas \cite{Wright2016Effect,Ali2014Mems,Wright2018Mems}, offering excellent RF performance with low insertion loss and high isolation.  Phase transition materials such as vanadium dioxide (VO2) provide another approach, enabling frequency reconfiguration through voltage-controlled phase changes \cite{Lou2025Frequency}.  Other techniques include magnetically actuated pixels \cite{Pal2018Magnetically} and tunable designs using varactors \cite{Ouedraogo2014Tunable}.  More recent work has focused on dual-port mmWave reconfigurable designs with optimized pixel configurations \cite{Tang2023Dual-port} and novel frequency reconfigurable implementations \cite{Qiao2023Novel,Bichara2021Miniaturized}.

\section{Acknowledgement}
The author would like to thank Professor Glenn Smith for his dedication in writing the original IEEE paper \cite{RECAP} on which much of this chapter is based.  The author also acknowledges the contributions of the full research team as described in \cite{MaloneySmithAntChapters}.

\addcontentsline{toc}{section}{References}
\FloatBarrier

% Bibliography for Chapter 5
% Uses chapter-specific .bib files organized by topic
\bibliography{../Literature/master_bibliography,%
              ../Literature/fragmented_aperture_core,%
              ../Literature/reconfigurable_ch5,%
              ../Literature/ml_ai_optimization_methods}
\bibliographystyle{IEEEtran}


		% ch5
\include{FragArrays}					% ch6
\include{WidebandArrays}				% ch7
\chapter{Designing Wide Scan Fragmented Array Antennas}
%\authortoc{James G. Maloney}
%\chapterauthor{James G. Maloney}

\section{Introduction}

The previous chapters have demonstrated that fragmented aperture arrays can achieve remarkable bandwidths---up to 33:1 and beyond---while maintaining broadside gain that closely tracks the uniform aperture limit.  However, one persistent challenge has been maintaining wide scan volume across the full operating bandwidth.  This chapter addresses that challenge, describing design techniques that achieve scan volumes exceeding $\pm 60^{\circ}$ across the entire design band.

Chapter~7 presented the development of ultra-wideband fragmented arrays, culminating in a 33:1 bandwidth proof-of-concept array.  Figure~\ref{fig:WSA1} shows the measured and predicted embedded element realized gain for that array, confirming excellent agreement between model and measurement across three independent measurement facilities.  The broadside gain tracks the uniform aperture area limit well across the 0.3 to 10~GHz design band, and the inset photograph shows the array on the outdoor measurement range.

\begin{figure}
\begin{center}
\includegraphics[angle=-90,width=\linewidth]{/Users/jim.maloney/Book/images/5996889-fig-1-hires.pdf}
\caption{Measured and predicted embedded element realized gain (EERG) for the 33:1 bandwidth array.  The broadside gain (red line) tracks the uniform aperture area limit (blue line) across the 0.3 to 10~GHz design band.  Measurements from three independent facilities (anechoic chamber, compact range, and outdoor range) confirm the model accuracy.  Inset: the array on the outdoor measurement range \cite{MaloneyWideScan}.}
\label{fig:WSA1}
\end{center}
\end{figure}

Despite the excellent broadside performance, this array exhibited a shortcoming in its scan volume.  As discussed in Chapter~7 (see Figure~\ref{fig:WBA21}), the principal-plane antenna patterns for the 33:1 array show wide scan volume ($> \pm 60^{\circ}$) across most of the operating band.  However, near the top of the band (approximately 8.5 to 10~GHz), a significant narrowing of the scan volume is evident, reducing the usable scan range to roughly $\pm 40^{\circ}$.  In the years since the design of the 33:1 antenna, preventing this upper-band scan volume narrowing has been an active area of research.

A second shortcoming of the 33:1 array is related to aperture efficiency.  As visible in Figure~\ref{fig:WSA1}, the array was within 3~dB of the uniform aperture gain limit (better than 50\% aperture efficiency), but the majority of the loss was attributable to the resistive backplane required to suppress the ground-plane nulls over such a large bandwidth.  This loss limits the array to receive-only or low-power transmit applications.

This chapter describes progress in overcoming these two limitations.  First, an improved design process using spectral-domain FDTD simulation enables the genetic algorithm to optimize scan performance directly, alleviating the upper-band scan volume narrowing without requiring tighter inter-element spacing.  Second, for applications that do not require the extreme bandwidth of the 33:1 design, a simpler antenna structure can be used that avoids the introduction of resistive loss, enabling high-efficiency operation suitable for transmit applications.  Third, a laminated printed circuit board (PCB) fabrication approach replaces the traditional machined aluminum construction, yielding a more integrated, mass-producible antenna.  These advances are illustrated through the design of a wide-scanning ($\pm 60^{\circ}$), whole X-band (8--12~GHz) phased array element.

\section{Fabrication Approaches}

\subsection{Traditional Construction}

Early fragmented aperture array antennas were constructed using the approach depicted in Figure~\ref{fig:WSA2}.  This design consists of an array of elements printed on circuit board material, suspended over a conducting ground plane by machined aluminum feed towers.  The feed towers enclose differential coaxial lines that provide the dual-polarized excitation to each antenna element.  Combining networks and beamforming electronics connect to these coaxial lines in the space behind the ground plane.

\begin{figure}
\begin{center}
\includegraphics[angle=-90,width=0.85\linewidth]{/Users/jim.maloney/Book/images/5996889-fig-3-hires.pdf}
\caption{Traditional fragmented aperture array construction.  Fragmented layers are printed on circuit board material and separated by foam spacers above a machined aluminum ground plane.  Feed towers enclose differential coaxial lines that connect to the dual-polarized elements \cite{MaloneyWideScan}.}
\label{fig:WSA2}
\end{center}
\end{figure}

This construction method can produce lightweight structures; for example, a 0.6~m $\times$ 0.6~m array built in this manner weighed only 4~kg for the entire antenna assembly.  This basic construction has been used to build many successful wideband fragmented aperture designs over the past two decades, including the 33:1 bandwidth array and the various designs summarized in the thickness-versus-frequency chart in Chapter~7 (Figure~\ref{fig:WBA22}).

\subsection{Laminated Printed Circuit Board Construction}

While the traditional machined construction produces excellent antenna performance, it is not ideal for mass production, particularly at higher frequencies where the mechanical tolerances become demanding.  An alternative approach, shown in Figure~\ref{fig:WSA3}, builds the entire antenna from laminated printed circuit board layers.

\begin{figure}
\begin{center}
\includegraphics[angle=-90,width=0.85\linewidth]{/Users/jim.maloney/Book/images/5996889-fig-4-hires.pdf}
\caption{Laminated printed circuit board (PCB) fabrication approach.  The entire antenna---fragmented layers, dielectric spacers, and ground plane---is built up as a multi-layer PCB stack.  Element feeds are plated vias near the center of each unit cell, and surface wave suppression vias are placed near the perimeter \cite{MaloneyWideScan}.}
\label{fig:WSA3}
\end{center}
\end{figure}

In this approach, the fragmented conducting layers, the dielectric spacers between them, and the ground plane are all integrated into a single laminated PCB stack-up.  The element feeds, which were previously machined coaxial lines enclosed in aluminum towers, are now implemented as closely spaced plated vias manufactured using standard multi-layer PCB techniques.  Feed networks can also be implemented as additional PCB layers, producing a highly integrated antenna panel.

The PCB approach offers several practical advantages.  It leverages mature, high-volume PCB manufacturing processes, resulting in antennas that are more repeatable, more easily mass-produced, and potentially less expensive than machined alternatives.  These benefits are especially significant at X-band and higher frequencies, where the small feature sizes make machined construction increasingly difficult.

However, the laminated PCB approach also introduces new design challenges:

\begin{itemize}

\item \textbf{Surface wave suppression.}  Thick dielectric substrates can support surface waves that cause scan blindness---a condition where the array becomes poorly matched at specific combinations of frequency and scan angle.  To prevent this, surface wave suppression vias must be incorporated into the laminated stack-up as an integral part of the design.  These vias are visible near the perimeter of the unit cell in Figure~\ref{fig:WSA3}.

\item \textbf{Dielectric loading.}  The presence of dielectric material throughout the cavity between the radiating layers and the ground plane changes the effective wavelength and alters the impedance environment seen by the antenna elements.  The design process must account for the specific dielectric properties of the PCB substrate materials.

\item \textbf{Modified feed structures.}  The feeds are no longer simple 50~$\Omega$ coaxial lines in air; they are closely spaced vias in a dielectric environment.  The characteristic impedance and coupling behavior of these via-based feeds must be accurately modeled during the design process.

\end{itemize}

All of these effects are captured naturally by the FDTD simulation used in the genetic algorithm design process, so no special analytical treatment is required.  The full-wave simulation simply includes the dielectric layers, the vias, and the complete PCB stack-up geometry, and the optimizer works with the true electromagnetic behavior of the structure.

\section{Spectral-Domain FDTD for Wide Scan Optimization}

\subsection{Limitations of Standard Periodic Boundary Conditions}

The design process for fragmented aperture array elements uses a genetic algorithm to optimize a single unit cell of the array, terminated by periodic boundary conditions (PBC) that simulate an infinite array environment.  The FDTD method is ideal for this purpose because a single time-domain simulation produces the full frequency-domain response across the entire design bandwidth.

In the standard FDTD implementation of periodic boundary conditions, the PBC is applied as a wrap-around boundary at the edges of the unit cell.  This approach naturally models the infinite array at broadside (zero scan angle) and provides the complete broadband response in a single simulation.  This is the approach that was used to design the earlier fragmented aperture arrays, including the 33:1 bandwidth design.

However, standard broadside-only PBC does not provide information about the array's scan performance.  The scan volume narrowing observed in the 33:1 array (Figure~\ref{fig:WBA21}) was a consequence of optimizing only for broadside performance: the genetic algorithm had no information about off-broadside behavior, so it had no mechanism to prevent scan volume degradation.

\subsection{The Spectral-Domain FDTD Approach}

To incorporate scan performance into the design process, a spectral-domain FDTD approach to periodic boundary conditions \cite{Aminian} was integrated into the design suite.  In the spectral-domain formulation, the FDTD simulation is performed at a fixed transverse wavenumber $(k_x, k_y)$ rather than at a fixed scan angle $(\theta, \phi)$.  The PBC is still implemented as a wrap-around boundary, but the constant transverse wavenumber assumption means that the effective scan angle is frequency dependent.

Figure~\ref{fig:WSA4} illustrates this relationship.  Each curve shows the elevation angle as a function of frequency for a specific normalized transverse wavenumber $K_z$.  The dashed box indicates the target design space: the 8--12~GHz X-band with a scan volume of $\pm 60^{\circ}$.  Because the contours are not horizontal (i.e., constant angle versus frequency), a single spectral-domain simulation does not map to a single scan angle.  Instead, each simulation sweeps through different scan angles as it sweeps through frequency.

\begin{figure}
\begin{center}
\includegraphics[angle=-90,width=0.85\linewidth]{/Users/jim.maloney/Book/images/5996889-fig-5-hires.pdf}
\caption{Elevation angle versus frequency contours for several normalized transverse wavenumbers $K_z$ used in the spectral-domain FDTD design process.  The dashed box indicates the target 8--12~GHz, $\pm 60^{\circ}$ scan volume.  Because the contours are not constant versus frequency, the transverse wavenumber values are chosen to provide denser sampling at higher frequencies and larger scan angles, where scan problems are most likely to occur \cite{MaloneyWideScan}.}
\label{fig:WSA4}
\end{center}
\end{figure}

\subsection{Sampling the Scan Volume}

The key to successful wide-scan design is to strike the right balance between sampling the scan volume sufficiently and not performing too many simulations (since each simulation adds to the computational cost of every fitness evaluation in the genetic algorithm).  The exact number of spectral-domain simulations needed is problem specific, but a useful observation guides the selection of wavenumber samples: poor scan performance, when it occurs, typically manifests at higher frequencies and larger scan angles.

As shown in Figure~\ref{fig:WSA4}, the transverse wavenumber values are chosen to concentrate the sampling in the region of the scan volume where problems are most likely---the upper-right portion of the frequency-angle space.  The $K_z = 0$ contour corresponds to broadside at all frequencies, while progressively larger values of $K_z$ sweep through progressively larger scan angles.  By including several such simulations in the fitness evaluation, the genetic algorithm receives information about both broadside and off-broadside performance and can optimize accordingly.

This approach adds computational cost to each fitness evaluation, since multiple spectral-domain FDTD simulations must be run for each candidate element design.  However, the cost is manageable because the number of required wavenumber samples is typically small (on the order of five to ten), and each individual simulation is no more expensive than the single broadside simulation used in the previous design approach.

\section{Example: Whole X-Band Array Element}

\subsection{Design Parameters}

To illustrate the improved design process and the PCB fabrication approach, a whole X-band (8--12~GHz) phased array element was designed with a target scan volume of $\pm 60^{\circ}$ in all azimuth planes.  The X-band was chosen because it represents a practical frequency range for military radar and communications applications, and because typical printed circuit board elements (e.g., microstrip patches) are not broadband enough to cover the full 8--12~GHz band in a low-profile PCB form factor.

The design begins by selecting the array lattice constant to prevent grating lobes at the maximum scan angle and highest operating frequency:
\begin{equation}
\label{eq:WSAlattice}
\frac{s}{\lambda_{\text{high}}} = \frac{1}{1 + \sin\theta_{\max}} = 0.536 \quad \text{for } \theta_{\max} = 60^{\circ}
\end{equation}
where $s$ is the element spacing and $\lambda_{\text{high}}$ is the free-space wavelength at the highest operating frequency (12~GHz).

Next, the number of fragmented pixels across the unit cell is selected using the rules of thumb described in Chapter~3: typically 20--30 pixels across the unique quadrant of the element (as illustrated in Figure~\ref{fig:WSA3}).  The element is then designed using the genetic algorithm approach with the spectral-domain FDTD providing the fitness evaluation at the normalized $K_z$ and $K_y$ values shown in Figure~\ref{fig:WSA4}.

\subsection{Broadside Performance}

After the genetic algorithm completes the design, the element performance is verified by simulating a large finite array---in this case, 21 $\times$ 21 elements.  The embedded element realized gain is obtained by computing the gain of the central element while all surrounding elements are terminated in matched loads.

Figure~\ref{fig:WSA5} shows the broadside embedded element realized gain as a function of frequency.  The gain is within approximately 0.2~dB of the uniform aperture area limit across the entire 8--12~GHz design band.  This near-ideal aperture efficiency demonstrates that the PCB fabrication approach, with its dielectric loading and via-based feeds, does not compromise the ability of the fragmented aperture design to efficiently utilize the available aperture area.

\begin{figure}
\begin{center}
\includegraphics[angle=0,width=0.85\linewidth]{/Users/jim.maloney/Book/images/5996889-fig-6-hires.pdf}
\caption{Broadside embedded element realized gain for the whole X-band fragmented array element (blue line) compared to the uniform aperture area limit (red dots).  The realized gain is within approximately 0.2~dB of the theoretical limit across the 8--12~GHz design band \cite{MaloneyWideScan}.}
\label{fig:WSA5}
\end{center}
\end{figure}

\subsection{Impedance Match}

Figure~\ref{fig:WSA6} compares the VSWR for the embedded element with the VSWR for the infinite array scanned at broadside.  Both are well below 2:1 across the design band, but they are not identical.  The difference arises because the infinite-array VSWR includes the mutual coupling from all neighboring elements, whereas the embedded element VSWR reflects the impedance seen at the terminals of a single element in the finite array environment.

\begin{figure}
\begin{center}
\includegraphics[angle=-90,width=0.85\linewidth]{/Users/jim.maloney/Book/images/5996889-fig-7-hires.pdf}
\caption{VSWR comparison for the X-band fragmented array element.  The embedded element VSWR (blue line) and the broadside infinite-array scan VSWR (green line) are both below 2:1 across the design band.  The difference between the two reflects the influence of mutual coupling from neighboring elements \cite{MaloneyWideScan}.}
\label{fig:WSA6}
\end{center}
\end{figure}

An important point for phased array design is that it is the \emph{scanned} VSWR (the infinite-array value) that must be kept small, because this is the impedance that the element presents to the beamforming network during scanning.  The embedded element VSWR may actually be higher than the scanned VSWR when significant mutual coupling is being exploited to improve the scanned impedance match.  This is another example of how the fragmented aperture design approach embraces mutual coupling rather than attempting to minimize it.

\subsection{Scan Performance}

Figures~\ref{fig:WSA7} through \ref{fig:WSA9} show the embedded element realized gain as a function of both azimuth and elevation angle at three frequencies spanning the X-band: 8~GHz (bottom), 10~GHz (middle), and 12~GHz (top).  These plots are obtained by simulating the 21 $\times$ 21 finite array with the central element excited and all others terminated, then computing the realized gain pattern of the central element.

\begin{figure}
\begin{center}
\includegraphics[angle=-90,width=0.85\linewidth]{/Users/jim.maloney/Book/images/5996889-fig-8-hires.pdf}
\caption{Embedded element realized gain at 8~GHz as a function of azimuth and elevation angle for the V-pol feed.  The wide scan volume is evident, with usable gain extending well beyond $60^{\circ}$ in both planes \cite{MaloneyWideScan}.}
\label{fig:WSA7}
\end{center}
\end{figure}

\begin{figure}
\begin{center}
\includegraphics[angle=-90,width=0.85\linewidth]{/Users/jim.maloney/Book/images/5996889-fig-9-hires.pdf}
\caption{Embedded element realized gain at 10~GHz.  The wide scan volume ($> \pm 60^{\circ}$) is maintained at the center of the X-band \cite{MaloneyWideScan}.}
\label{fig:WSA8}
\end{center}
\end{figure}

\begin{figure}
\begin{center}
\includegraphics[angle=-90,width=0.85\linewidth]{/Users/jim.maloney/Book/images/5996889-fig-10-hires.pdf}
\caption{Embedded element realized gain at 12~GHz.  Some slight reduction in the scan volume is visible in the azimuth direction, but the overall scan volume still substantially exceeds $\pm 60^{\circ}$ \cite{MaloneyWideScan}.}
\label{fig:WSA9}
\end{center}
\end{figure}

At 8~GHz (Figure~\ref{fig:WSA7}), the scan volume is excellent, with strong realized gain extending well beyond $60^{\circ}$ in both azimuth and elevation.  At 10~GHz (Figure~\ref{fig:WSA8}), the wide scan volume is maintained.  At 12~GHz (Figure~\ref{fig:WSA9}), there is some slight degradation in the scan volume in the azimuth direction, but the overall scan performance still substantially exceeds $\pm 60^{\circ}$.

This is a dramatic improvement over the scan performance of the 33:1 array element (Figure~\ref{fig:WBA21}), which showed significant scan volume narrowing to only $\pm 40^{\circ}$ in the upper portion of its operating band.  The key difference is that the spectral-domain FDTD design process gives the genetic algorithm direct information about the scan performance, allowing it to optimize for wide scan and wide bandwidth simultaneously.

\section{Discussion}

\subsection{Bandwidth--Efficiency--Scan Trade Space}

The results of this chapter highlight an important trade space in fragmented aperture array design.  The 33:1 bandwidth array described in Chapter~7 achieved extraordinary bandwidth but required a lossy broadband screen backplane to suppress ground-plane nulls, limiting the array to receive-only applications.  The X-band element described in this chapter achieves a more modest bandwidth (approximately 1.5:1) but does so without the need for a lossy backplane, enabling high-efficiency, transmit-capable operation.

This trade-off is a fundamental consequence of the physics.  As discussed in Chapter~7, the broadband screen backplane uses resistive card (r-card) layers to prevent the half-wave nulls that occur when the aperture-to-ground-plane spacing is an integer multiple of $\lambda/2$.  For very large bandwidths, these nulls cannot be avoided without some form of loss.  For moderate bandwidths (on the order of 2:1 to 3:1), the aperture thickness can be chosen so that no half-wave nulls fall within the operating band, eliminating the need for lossy backplane layers entirely.

\subsection{PCB Design Rules}

The design rules and scaling relationships discussed in Chapter~7 (Section~7.6) apply to the PCB approach as well, with one important modification.  The $\lambda/12$ rule of thumb for cavity thickness (Figure~\ref{fig:WBA22}) was derived from air-filled cavity designs.  For the PCB approach, the cavity is dielectrically loaded, so the relevant wavelength is the wavelength in the substrate material.  The modified rule of thumb is:
\begin{equation}
\label{eq:WSAthickness}
T \approx \frac{\lambda_{\text{substrate}}}{12} = \frac{\lambda_0}{12\sqrt{\epsilon_r}}
\end{equation}
where $\lambda_0$ is the free-space wavelength at the lowest operating frequency and $\epsilon_r$ is the relative permittivity of the substrate.  This dielectric loading results in a physically thinner antenna, which is one of the practical benefits of the PCB approach.

\textcolor{red}{[TODO: Include a thickness vs.\ frequency chart for PCB designs analogous to the air-filled cavity chart in Chapter~7, once enough PCB designs have been completed to establish the trend.]}

\section{Summary and Conclusions}

This chapter described three advances that address key limitations of earlier fragmented aperture array designs:

\begin{enumerate}

\item \textbf{Spectral-domain FDTD for wide scan optimization.}  By incorporating a spectral-domain FDTD approach to periodic boundary conditions into the genetic algorithm design process, the optimizer gains direct information about the array's scan performance at multiple angles.  This enables the design of elements with scan volumes exceeding $\pm 60^{\circ}$ across the full operating bandwidth, a significant improvement over designs optimized only at broadside.

\item \textbf{Laminated PCB fabrication.}  Replacing the traditional machined aluminum construction with a laminated printed circuit board approach yields a more integrated, more easily mass-produced, and potentially lower-cost antenna, with particular advantages at X-band and higher frequencies where machining tolerances are demanding.

\item \textbf{High-efficiency, transmit-capable designs.}  For applications with moderate bandwidth requirements (on the order of 1.5:1 to 3:1), the antenna can be designed without a lossy broadband screen backplane, resulting in high aperture efficiency suitable for transmit applications.

\end{enumerate}

These advances were demonstrated through the design of a whole X-band (8--12~GHz) phased array element that achieves broadside gain within 0.2~dB of the uniform aperture area limit and a scan volume exceeding $\pm 60^{\circ}$ across the design band, all in a laminated PCB form factor.

\FloatBarrier
\begin{thebibliography}{99}

\bibitem{MaloneyWideScan} J.~G.~Maloney, B.~N.~Baker, R.~T.~Lee, G.~N.~Kiesel, and J.~J.~Acree, ``Wide Scan, Integrated Printed Circuit Board, Fragmented Aperture Array Antennas,'' in Proc.\ 2011 IEEE International Symposium on Antennas and Propagation, Spokane, WA, July 2011, pp.\ 1965--1968.

\bibitem{MaloneyFragPatent} J.~G.~Maloney, M.~P.~Kesler, P.~H.~Harms, and G.~S.~Smith, Fragmented Aperture Antennas and Broadband Ground Planes, U.S.\ Patent No.\ 6,323,809~B1, November 27, 2001.

\bibitem{Aminian} A.~Aminian and Y.~Rahmat-Samii, ``Spectral FDTD: A novel technique for the analysis of oblique incident plane wave on periodic structures,'' IEEE Trans.\ Antennas Propag., vol.~54, no.~6, pp.~1818--1825, June 2006.

\bibitem{BalanisHB} C.~A.~Balanis, Modern Antenna Handbook, Chapter~12, Wiley, 2008.

\end{thebibliography}


	% ch8
\chapter{Reconfigurable Fragmented Aperture Arrays}
%\authortoc{James G. Maloney}
%\chapterauthor{James G. Maloney}

\section{Introduction}

The preceding chapters have treated the fragmented aperture antenna as a fixed structure: once the genetic algorithm determines the optimal pixel configuration and the antenna is fabricated, the conducting pattern is permanent.  Chapter~5 demonstrated that reconfigurable single-element fragmented apertures can dynamically change their frequency, beam direction, and polarization by switching pixel connections.  Chapter~8 showed that fixed fragmented array elements, when designed using spectral-domain FDTD, can achieve scan volumes exceeding $\pm 60^{\circ}$ across the operating bandwidth.

This chapter describes the convergence of these two ideas: a phased array in which the fragmented aperture element is electronically reconfigured as the beam is steered.  The motivation is to overcome a fundamental limitation of any phased array with fixed element geometry---the scan-dependent degradation of the embedded element pattern, which causes the array gain to fall off faster than the physical aperture projection as the beam is steered away from broadside.

\section{The Scan Loss Problem}

\subsection{Embedded Element Pattern and Scan Rolloff}

In a phased array, the gain in any direction is determined by the product of the array factor and the embedded element pattern (EEP).  The array factor can be steered to any angle within the visible region by adjusting the element phase weights, and in the absence of grating lobes, the array factor peak tracks the commanded steer direction with negligible loss.  The embedded element pattern, however, is a property of the fixed element geometry and its electromagnetic environment---the mutual coupling to neighboring elements, the ground plane, and the feed structure.  For a well-designed array element, the EEP typically peaks at or near broadside and rolls off with increasing scan angle.

The physical area of the aperture projected onto the scan direction decreases as $\cos\theta$, where $\theta$ is the scan angle measured from broadside.  This projected-area loss is a fundamental geometric consequence that no antenna design can overcome: a planar aperture of area $A$ presents an effective collecting area of $A\cos\theta$ when viewed from angle $\theta$.  For an ideal element whose EEP exactly followed the projected-area limit, the array gain at scan angle $\theta$ would be
%
\begin{equation}
G(\theta) = G_0 \cos\theta
\label{eq:RCA_costheta}
\end{equation}
%
where $G_0$ is the broadside gain.  This represents the best possible performance for a planar phased array; no fixed-element design can exceed this limit.

In practice, however, the embedded element pattern of a fixed-geometry array element rolls off faster than $\cos\theta$.  The element pattern exhibits frequency-dependent features---ripples, nulls, and asymmetries---that arise from the complex electromagnetic interactions within and between elements.  The scan loss is compounded by impedance mismatch: as the beam is steered, the active impedance at each element changes due to the angle-dependent mutual coupling, degrading the impedance match and further reducing the realized gain.

\subsection{The $\cos^n\theta$ Model}

In radar system engineering, the scan loss of a phased array is often modeled as a power of $\cos\theta$:
%
\begin{equation}
G(\theta) = G_0 \cos^n\theta
\label{eq:RCA_cosn}
\end{equation}
%
where the exponent $n$ captures the combined effects of projected area ($n = 1$), element pattern rolloff, and scan impedance mismatch.  For an ideal element, $n = 1$.  In practice, $n$ ranges from 1.3 to 2.5 or higher, depending on the element type, bandwidth, and frequency within the operating band.  A value of $n = 1.5$ is a common first approximation for well-designed elements.  Radar textbooks often use $n = 2$ as a representative model.

The consequences of $n > 1$ are significant.  At $\theta = 60^{\circ}$, the difference between $\cos^1(60^{\circ}) = 0.5$ ($-3.0$~dB) and $\cos^2(60^{\circ}) = 0.25$ ($-6.0$~dB) is 3~dB---a factor of two in power.  For a radar system, this 3~dB of additional scan loss translates directly into reduced detection range: the fourth-root dependence of radar range on power means that each additional dB of scan loss reduces the detection range by approximately 6\%.

\begin{table}
\begin{center}
\caption{Scan loss at selected angles for various values of the scan loss exponent $n$.  The additional loss beyond the ideal $\cos\theta$ projection is shown in the rightmost column.}
\label{tab:RCA_scanloss}
\begin{tabular}{cccc}
\hline
$\theta$ & $\cos^1\theta$ (dB) & $\cos^2\theta$ (dB) & Additional loss (dB) \\
\hline
$0^{\circ}$  & 0.0 & 0.0 & 0.0 \\
$30^{\circ}$ & $-1.2$ & $-2.5$ & 1.2 \\
$45^{\circ}$ & $-3.0$ & $-6.0$ & 3.0 \\
$60^{\circ}$ & $-6.0$ & $-12.0$ & 6.0 \\
$70^{\circ}$ & $-9.3$ & $-18.7$ & 9.3 \\
\hline
\end{tabular}
\end{center}
\end{table}

Table~\ref{tab:RCA_scanloss} quantifies the impact for $n = 1$ and $n = 2$.  At moderate scan angles ($30^{\circ}$--$45^{\circ}$), the additional loss from $n = 2$ versus $n = 1$ is 1--3~dB.  At $60^{\circ}$, the penalty doubles to 6~dB.  At very wide scan angles approaching $70^{\circ}$, the $n = 2$ element pattern has rolled off to $-18.7$~dB---nearly 10~dB worse than the projected-area limit.  For systems that require wide-angle coverage, reducing the scan loss exponent from $n \approx 2$ to $n \approx 1$ is a transformative improvement.

\subsection{Why Fixed Elements Cannot Achieve $n = 1$}

The $\cos\theta$ projected-area limit is the best that any fixed-geometry element can achieve, and in practice no fixed element achieves it uniformly across frequency and angle.  The fundamental reason is that a fixed conducting pattern supports a fixed set of current modes.  These modes produce a radiation pattern that is optimized (by the GA) for a particular set of conditions---typically broadside or a weighted combination of broadside and a few scan angles (as in Chapter~8).  When the beam is steered to an angle that was not explicitly optimized, the fixed current modes radiate with a pattern whose peak does not track the scan direction.

The spectral-domain FDTD approach described in Chapter~8 mitigates this problem by sampling the scan volume during the design process, but it does not eliminate it.  The optimizer must balance broadside performance against scanned performance, and the fixed element geometry represents a single compromise.  At any given frequency and scan angle, a design specifically optimized for that condition would outperform the compromise design.

\section{Reconfigurable Array Elements for Scan Management}

\subsection{Concept}

The solution suggested by this analysis is straightforward: do not require the element to be a single compromise design.  Instead, use a reconfigurable element---a fragmented aperture array element whose pixel configuration can be electronically changed---and re-optimize the configuration for each combination of operating frequency and beam steer direction.

When the array is commanded to steer to angle $\theta_s$, the element is reconfigured to a pixel pattern that has been pre-optimized for that scan angle.  The embedded element pattern of the reconfigured element has its peak near $\theta_s$ rather than at broadside, so the element pattern does not fight the array factor.  The scan loss approaches the $\cos\theta$ projected-area limit because the only remaining loss mechanism is the geometric reduction of the effective aperture area.

This concept can be understood by analogy with the Agile Aperture Antenna of Chapter~5.  In that chapter, different switch configurations were designed for different objectives: one for a broadside beam, another for an end-fire beam.  The reconfigurable array concept is the same principle applied to every element of a phased array, with the objective for each configuration being maximum embedded element gain in the commanded steer direction.

\subsection{Design Methodology}

The design procedure for a reconfigurable array element follows naturally from the tools developed in earlier chapters.  For each desired scan angle $\theta_s$ (and, if needed, each frequency sub-band), the genetic algorithm is run in the infinite-array environment with periodic boundary conditions appropriate for that scan angle.  The fitness function rewards embedded element realized gain in the direction $\theta_s$, impedance match at the scanned active impedance, and bandwidth coverage.

For a set of $M$ discrete scan angles $\{\theta_1, \theta_2, \ldots, \theta_M\}$, the design process produces $M$ corresponding pixel configurations $\{P_1, P_2, \ldots, P_M\}$.  These configurations are stored in a look-up table.  When the array controller commands a beam steer to angle $\theta_s$, the appropriate pixel configuration $P_s$ is loaded onto every element simultaneously with the phase weights that steer the array factor.

The spectral-domain FDTD approach of Chapter~8 is used for each design, but with an important distinction.  In Chapter~8, the optimizer sampled multiple scan angles simultaneously and sought a single design that performed acceptably across the full scan volume.  Here, each optimization focuses on a \emph{single} target scan angle (or a narrow range around it), freeing the optimizer to find the best possible design for that specific condition.  The result is a family of designs, each of which is individually superior at its target angle to any single fixed design.

\subsection{Pre-Computed Design Library}

The practical implementation requires a library of pre-computed element configurations spanning the desired scan volume.  The library is indexed by scan angle (and possibly frequency), and the array controller selects the appropriate configuration in real time as the beam is steered.

The number of entries in the library depends on the angular resolution required.  If the element performance varies slowly with scan angle---as is typical for electrically small pixels---then a coarse angular grid may suffice, with interpolation or nearest-neighbor selection for intermediate angles.  Preliminary design studies suggest that scan angle steps of $5^{\circ}$ to $10^{\circ}$ provide adequate coverage, resulting in a library of approximately 20--50 configurations for hemispherical scan coverage.

\textcolor{red}{\textbf{[INSERT FIGURE: Conceptual diagram showing a library of pre-computed pixel configurations indexed by scan angle, with the array controller selecting the appropriate configuration for the commanded steer direction.]}}

For a two-dimensional scan volume, the library must cover both azimuth and elevation, but the symmetry of a square lattice element reduces the required entries.  An element with four-fold symmetry requires configurations covering only one octant of the upper hemisphere; the remaining scan directions are obtained by $90^{\circ}$ rotations of the element pattern.

\section{Design Examples}

\textcolor{red}{\textbf{[NOTE: The design examples in this section are based on work performed for the DARPA Arrays at Commercial Timescales (ACT) program.  The original design data is being located and, if necessary, will be regenerated using the current design tools.  The results described below represent the scope and character of the designs; specific numerical values and figures will be updated when the data is confirmed.]}}

\subsection{Frequency Reconfiguration}

A first set of design examples demonstrated the ability of a reconfigurable fragmented array element to tune across multiple frequency bands.  The element geometry and lattice were held constant, and the pixel configuration was optimized separately for operation in several frequency bands spanning a wide range.  In each case, the optimizer was able to find a pixel configuration that produced a well-matched, broadside element with gain near the aperture area limit for the target band.

This result is not surprising given the experience from Chapter~5, where very different switch configurations produced antennas operating in different frequency ranges.  What is significant is that the same behavior extends to array elements operating in the infinite-array mutual-coupling environment.  The pixel configuration adjusts not only the resonant behavior of the individual element but also the inter-element coupling, effectively re-tuning the array's electromagnetic environment for each frequency band.

\textcolor{red}{\textbf{[INSERT FIGURE: Several pixel configurations for the same array element, each optimized for a different frequency band.  The designs should look visually distinct, illustrating that different frequencies require fundamentally different conducting patterns.]}}

\textcolor{red}{\textbf{[INSERT FIGURE: Embedded element realized gain versus frequency for each configuration, showing that each design provides near-aperture-limited gain within its target band.]}}

\subsection{Scan Angle Reconfiguration}

The more compelling set of design examples demonstrated scan angle reconfiguration.  The element was optimized for a fixed frequency band at several discrete scan angles: broadside ($0^{\circ}$), $30^{\circ}$, $45^{\circ}$, $60^{\circ}$, and beyond.

At each scan angle, the optimizer produced a pixel configuration that placed the embedded element pattern peak near the target angle.  Figure~\ref{fig:RCA_scanconfigs} illustrates the concept: the conducting pattern changes as the target scan angle increases, reflecting the need for different current distributions to radiate efficiently in different directions.

\textcolor{red}{\textbf{[INSERT FIGURE: Pixel configurations for $\theta = 0^{\circ}, 30^{\circ}, 45^{\circ}, 60^{\circ}$, showing the evolution of the conducting pattern with scan angle.]}}
\begin{figure}
\caption{Pixel configurations for the reconfigurable array element at four scan angles.  The conducting pattern changes significantly with scan angle, reflecting the different current distributions required to maximize radiation in the target direction.}
\label{fig:RCA_scanconfigs}
\end{figure}

The key result was that the embedded element pattern of each reconfigured design tracked the commanded scan angle, maintaining gain near the $\cos\theta$ projected-area limit out to $60^{\circ}$ and beyond.  This is in contrast to a fixed-element design, which would show progressively degraded gain beyond $45^{\circ}$.

\textcolor{red}{\textbf{[INSERT FIGURE: Overlay of embedded element patterns for each reconfigured design, showing the pattern peak tracking the scan angle.  A fixed-element pattern should be included for comparison.]}}

\textcolor{red}{\textbf{[INSERT FIGURE: Scan loss versus angle, comparing the reconfigurable element ($n \approx 1$) to a fixed element ($n \approx 1.5$--$2$) and the ideal $\cos\theta$ limit.  This is the central result of the chapter.]}}

It was particularly noteworthy that the reconfigured elements maintained good performance at scan angles beyond $60^{\circ}$.  At these extreme angles, fixed elements typically exhibit severe pattern degradation and impedance mismatch.  The reconfigurable element avoids these problems because its pixel pattern is specifically tailored for the wide-angle electromagnetic environment, including the dramatically different mutual coupling that occurs at extreme scan angles.

\section{Scan Loss Reduction: Physical Interpretation}

The improvement from reconfigurable scan management can be understood through several complementary perspectives.

\subsection{Current Distribution Perspective}

When a fixed-element array is scanned to a large angle, the current distribution on each element does not change---only the relative phases between elements change.  The element still supports the same modes it was designed for at broadside, and these modes do not radiate efficiently in the scanned direction.

In the reconfigurable element, changing the pixel configuration changes the conducting pattern, which changes the allowed current modes.  The optimizer finds a pattern that supports current distributions that radiate efficiently toward the target angle.  In essence, the element is not merely being phased to a new direction; it is being \emph{redesigned} for that direction.

\subsection{Impedance Perspective}

As a phased array is scanned, the active impedance at each element port changes due to the angle-dependent mutual coupling between elements.  For fixed elements, this impedance change is an uncontrollable consequence of the element geometry and the scan angle.  At certain combinations of frequency and scan angle, the active impedance can deviate dramatically from the matched condition, causing significant mismatch loss.

Reconfiguring the element pixel pattern changes the mutual coupling between elements, because the electromagnetic fields in the inter-element region depend on the conducting pattern.  The optimizer can therefore adjust the pixel pattern to maintain a good impedance match at each scan angle, compensating for the scan-dependent impedance variation that afflicts fixed designs.

\subsection{Scan Blindness Avoidance}

Scan blindness occurs when the active impedance of the array element becomes purely reactive at a specific combination of frequency and scan angle, typically due to the excitation of a surface wave on the array aperture.  For fixed-geometry elements, scan blindness can be avoided through careful design (as discussed in Chapter~8), but the available design space is limited by the requirement that the single fixed geometry must also perform well at all other scan angles.

Reconfigurable elements provide additional degrees of freedom to avoid scan blindness.  If a particular pixel configuration excites a surface wave at the current operating frequency and scan angle, the configuration can be changed to one that does not.  The pixel pattern optimized for each scan angle is inherently free of scan blindness at that angle, because the optimizer would reject any configuration that exhibited it.

\section{System Considerations}

\subsection{Switching Speed}

The practical utility of reconfigurable scan management depends on the switching speed of the pixel reconfiguration mechanism.  For electronically steered arrays that change beam direction on a pulse-to-pulse basis (as in modern radar systems), the pixel configuration must be updated within the beam dwell time---typically on the order of microseconds.

The MEMS-based switches discussed in Chapter~5 have switching times of 1--10~$\mu$s, which is marginal for pulse-to-pulse beam steering.  Solid-state switches (PIN diodes, FETs) are significantly faster, with switching times in the nanosecond range, but they introduce higher insertion loss and nonlinearity.  Phase-change materials such as VO2 offer a potential path to fast, low-loss switching, as discussed in Chapter~5.

For applications that do not require pulse-to-pulse agility---for example, communications systems that maintain a fixed beam direction for extended periods---the switching speed requirement is relaxed, and even slower reconfiguration mechanisms may be acceptable.

\subsection{Calibration}

In a conventional phased array, calibration accounts for element-to-element variations in gain and phase.  In a reconfigurable array, the calibration must also account for variations between pixel configurations.  Each entry in the design library corresponds to a different electromagnetic state of the array, and the calibration coefficients may differ for each state.

A practical calibration approach would characterize the array at each configuration in the design library, storing a separate set of calibration weights for each entry.  When the array switches configurations, the corresponding calibration weights are applied.

\subsection{Design Library Computation}

The computational cost of generating the design library is a one-time expense.  For a library of $M$ scan angles, $M$ independent genetic algorithm optimizations must be performed.  Each optimization has the same computational cost as a single fixed-element design.  For the spectral-domain FDTD approach, each design requires on the order of five to ten FDTD simulations per fitness evaluation, and the GA typically converges in 500--2000 generations with a population size of 100--500.

This amounts to a total computational investment of $M$ times the cost of a single fixed design.  For $M = 50$ (hemispherical coverage at $5^{\circ}$ steps, exploiting symmetry), this is a factor of 50 increase---substantial but well within the capabilities of modern computing clusters, particularly since the optimizations for different scan angles are completely independent and can be run in parallel.

The machine learning methods discussed in Chapter~10 offer a path to dramatically reducing this computational cost.  A surrogate model trained on the initial designs could predict the performance of candidate pixel configurations without running full FDTD simulations, accelerating the optimization by orders of magnitude.

\section{The DARPA ACT Program}

\subsection{Motivation and Program Goals}

Today's electromagnetic systems rely on antenna arrays for capabilities that are critical to a wide variety of military applications: multiple beam forming and electronic steering for communications, signal intelligence (SIGINT), radar, and electronic warfare.  However, wider adoption of phased arrays has been limited by two persistent problems: lengthy system development times and the inability to upgrade already-fielded capabilities.  These problems are exacerbated by the fact that military electronics have evolved at a far slower cadence than commercial electronics, and the performance gap between the RF capabilities of fielded military systems and the continuously improving digital electronics surrounding them continues to widen.

The traditional approach to phased array development---in which each new system is a monolithic, custom design---has resulted in 10-year development cycles, 20- to 30-year static life cycles, and costly service-life extension programs.  The DARPA Arrays at Commercial Timescales (ACT) program was established to push past these traditional barriers by developing a fundamentally different architecture: a digitally interconnected building block from which larger systems can be formed, scalable and customizable for each application without requiring a full redesign.

The ACT program was organized around two thrusts, each focused on a specific enabling technology for rapidly upgradable and widely deployable array architectures:

\begin{enumerate}
\item \textbf{Digitally-influenced common module:} A standardized building block comprising 80 to 90 percent of an array's core functionality, designed for insertion into a wide range of applications.  By leveraging commercial off-the-shelf (COTS) components and commercial CMOS semiconductor technology, the common module would dramatically reduce both development time and per-unit manufacturing cost.

\item \textbf{Reconfigurable and tunable RF apertures:} Aperture technology capable of spanning S-band to X-band frequencies (and points between) for a wide variety of antenna characteristics.  This thrust aligns directly with the reconfigurable fragmented aperture concept described in this chapter: a pixel-based aperture that can be electronically reconfigured for different frequency bands, scan angles, and operating modes.
\end{enumerate}

A central goal of the ACT program was to enable next-generation, multifunctional RF systems capable of simultaneous radar, electronic warfare, and communications functions from a single aperture.  This multifunctional requirement aligns naturally with the reconfigurable fragmented aperture concept: the same pixel array can be reconfigured not only for different scan angles (as described in this chapter) but also for different operating modes and frequency bands, providing the agility needed for multifunctional operation.

\subsection{GTRI's Advanced Phased Array Antenna Technology}

The Georgia Tech Research Institute (GTRI) developed the Advanced Phased Array Antenna Technology (APAT) system under the ACT program.  The APAT system is an all-digital, modular antenna designed to process signals directly on the antenna elements using radio frequency system-on-chip (RFSoC) technology.  This element-level digital architecture enables the full flexibility of digital beamforming---arbitrary beam patterns, simultaneous multiple beams, and adaptive interference cancellation---at each element.

The APAT represented the largest all-digital antenna system developed for tracking applications, enabling simultaneous, multi-stream data capture for hypersonic flight testing.  The all-digital architecture eliminates the analog beamforming network that constrains conventional arrays, allowing the beam patterns and signal processing to be defined entirely in software and updated without hardware modification.  This approach directly addresses the ACT program's goal of breaking the decades-long development and upgrade cycles of traditional array systems.

The system achieved extremely low latency---on the order of milliseconds---in edge processing for phased-array antenna applications, enabling real-time adaptive beamforming and signal classification.  GTRI's work also contributed to the broader effort in 3D heterogeneous integration (3DHI) of microelectronics, partnering with the Texas Institute for Electronics on a large-scale Department of Defense initiative to advance the packaging and integration technologies that are critical for next-generation common-module array architectures.

\subsection{Connection to Reconfigurable Fragmented Apertures}

The reconfigurable fragmented aperture array concept was proposed as the radiating element technology for the ACT architecture.  The combination of a reconfigurable pixel aperture (providing scan-optimized embedded element patterns) with element-level digital processing (providing arbitrary amplitude and phase control) would yield a phased array with performance that cannot be achieved by either technology alone: $\cos\theta$ scan loss across the full scan volume, multi-octave bandwidth, and real-time multifunctional capability.

The ACT program's Thrust~2---reconfigurable and tunable RF apertures spanning S-band to X-band---is precisely the application for which the fragmented aperture approach is designed.  A single fragmented aperture element, with sufficient pixel resolution and switching capability, can be reconfigured to operate across this entire frequency range.  Combined with the common-module digital architecture of Thrust~1, this provides the flexible, upgradable RF system that the DoD seeks: one that can be updated as quickly as commercial electronics evolve.

\textcolor{red}{\textbf{[PLACEHOLDER: Specific design results from the ACT proposal---including element configurations optimized for multiple frequency bands and scan angles, demonstrating scan beyond $60^{\circ}$---will be included when the original data is located or the designs are regenerated.]}}

\section{Summary and Conclusions}

This chapter has described a reconfigurable fragmented aperture array concept that addresses the scan loss problem inherent in all phased arrays with fixed element geometry.  The key findings are:

\begin{enumerate}

\item \textbf{Scan loss in fixed-element arrays exceeds the projected-area limit.}  The embedded element pattern of a fixed-geometry element rolls off faster than $\cos\theta$, resulting in scan loss that is well modeled by $\cos^n\theta$ with $n = 1.3$--$2.5$.  At $60^{\circ}$, this represents 2--6~dB of excess loss beyond the unavoidable $3$~dB of projected-area loss.

\item \textbf{Reconfigurable elements reduce the scan loss exponent to near unity.}  By re-optimizing the pixel configuration for each scan angle, the embedded element pattern peak is made to track the beam steer direction, and the scan loss approaches the $\cos\theta$ projected-area limit ($n \approx 1$).

\item \textbf{Reconfigurable elements provide wide-angle scanning beyond $60^{\circ}$.}  Fixed elements typically exhibit severe performance degradation at extreme scan angles.  Reconfigurable elements, optimized specifically for each angle, maintain good impedance match and pattern quality to $60^{\circ}$ and beyond.

\item \textbf{Reconfigurable elements naturally avoid scan blindness.}  The pixel configuration optimized for each scan angle is inherently free of scan blindness at that angle, providing additional robustness compared to fixed designs that must avoid scan blindness across the entire scan volume simultaneously.

\item \textbf{A pre-computed design library makes the concept practical.}  The computational cost of optimizing a library of 20--50 scan-angle configurations is a one-time investment that scales linearly with the number of configurations and can be parallelized.

\end{enumerate}

The combination of this reconfigurable scan management with the wide-bandwidth capabilities demonstrated in Chapters~7 and~8 suggests the possibility of a phased array that achieves multi-octave bandwidth with $\cos\theta$ scan loss across the entire scan volume---performance that no fixed-element array can match.

\FloatBarrier

\begin{thebibliography}{99}

\bibitem{BalanisHB12} W.~Croswell, T.~Durham, M.~Jones, D.~Schaubert, P.~Friederich, and J.~G.~Maloney, ``Wideband Arrays,'' in \emph{Modern Antenna Handbook}, C.~A.~Balanis, Ed., Hoboken, NJ: Wiley, 2008, ch.~12.

\bibitem{MaloneyWideScan} J.~G.~Maloney, B.~N.~Baker, R.~T.~Lee, G.~N.~Kiesel, and J.~J.~Acree, ``Wide scan, integrated printed circuit board, fragmented aperture array antennas,'' in \emph{Proc.\ 2011 IEEE International Symposium on Antennas and Propagation}, Spokane, WA, Jul.\ 2011, pp.\ 1965--1968.

\bibitem{MaloneyFragPatent} J.~G.~Maloney, M.~P.~Kesler, P.~H.~Harms, and G.~S.~Smith, ``Fragmented aperture antennas and broadband ground planes,'' U.S.\ Patent 6,323,809~B1, Nov.\ 27, 2001.

\end{thebibliography}

			% ch9	show improved scan loss, dynamic polarization, etc.

\appendix
\chapter{Computational Modeling of Antennas}

\section{Acknowledgement}
The author would like to personally thank Professor Glenn Smith, Georgia Tech Regents Professor Emeritus, for his tremendous help in compiling this appendix on the computational modeling of antennas.  Much of this material was published in the Balanis Antenna Engineering Handbook \cite{BalanisHB} and earlier in the Taflove book on computational electrodynamics \cite{MaloneySmithAntChapters}.

\section{Introduction}
The finite-difference time-domain (FDTD) method is a computational procedure for solving Maxwell's equations that is based on a clever algorithm first proposed by Kane S. Yee in 1966 \cite{Yee66}.  When Yee proposed his algorithm, the method was computationally intensive in terms of both storage and run time, and only problems of very modest size could be solved using the best computational facilities (mainframe computers).  Since then the power of computers has steadily increased, as has the popularity of the FDTD method.  The first comprehensive analyses of practical antennas using the method were performed during the early 1990s, and today such computations are routinely performed on personal computers \cite{Maloney1}--\cite{LuebbersGain}.

The purpose of this appendix is to introduce the reader to the fundamentals of the FDTD method as applied to practical antennas.  After studying this material, the reader should understand both the power and the limitations of the method and be in a position to decide whether the FDTD method is suitable for analyzing a particular antenna problem.  Because of the limited space, we cannot provide the details for implementing the method in a computer program.  Readers interested in writing their own program are referred to \cite{MaloneySmithAntChapters} for the details; others may wish to use one of the commercially available FDTD computer codes.

\section{The Basic FDTD Algorithm}
In the Yee algorithm, both space and time are discretized, with the increments in space for rectangular coordinates being $\Delta x$, $\Delta y$, $\Delta z$ and the increment in time being $\Delta t$ \cite{SmithBook}, \cite{Taf2005}.  Figure~\ref{fig:ACF1} is a schematic drawing showing a typical volume in which Maxwell's equations are to be solved.  The volume is divided into unit cells, and the electromagnetic constitutive parameters ($\epsilon=\epsilon_r \epsilon_o$, $\mu=\mu_r\mu_o$, $\sigma$) can vary from cell to cell to define different objects within the volume.\footnote{Here we mention only simple materials with constant permittivity, permeability, and electrical conductivity.  In the FDTD method there are techniques to handle more complicated materials, such as those with dispersive and anisotropic properties \cite{Taf2005}.}  The six components of the electromagnetic field $(E_x, E_y, E_z; H_x, H_y, H_z)$ are distributed over a unit cell (Yee cell) as shown in the inset of Figure~\ref{fig:ACF1}.  Notice that all of the components are located at different points within the cell, and the components of $H$ are displaced from those of $E$ by one half of a spatial increment, e.g., $\Delta x/2$.  Although not shown in the figure, the components of $H$ are also evaluated at points displaced by one half of a time increment, $\Delta t/2$, from those of $E$.

\begin{figure} \begin{center}
\includegraphics[angle=0,width=\linewidth]{/Users/jim.maloney/Book/images/AntChapOrigFig1.png}
\caption{Schematic drawing showing the computational volume, FDTD spatial lattice, and unit cell.}
\label{fig:ACF1}
\end{center}\end{figure}

The partial derivatives in Maxwell's equations are approximated by ratios of differences, for example,

\begin{equation}
\frac{\partial E_x}{\partial z} \approx \frac{\Delta E_x}{\Delta z}, \quad
\frac{\partial H_y}{\partial t} \approx \frac{\Delta H_y}{\Delta t}
\label{eqn:ACE1}
\end{equation}

For the spatial derivatives, the increment in the numerator is formed by differencing corresponding field components from adjacent unit cells, and for the temporal derivatives, it is formed by differencing field components from two adjacent time steps, e.g., $t$ and $t+\Delta t$.  The discretized Maxwell's equations are arranged to form two sets of difference equations known collectively as ``update equations.''  The first set, which we will call~A, determines the change in the magnetic field, $H(t+\Delta t/2) - H(t-\Delta t/2)$, from the electric field at an intermediate time step, $E(t)$.  The second set, which we will call~B, determines the change in the electric field, $E(t+\Delta t) - E(t)$, from the magnetic field at an intermediate time step, $H(t+\Delta t/2)$.

At the start of the computation, we have the initial conditions: throughout the computational volume, the electric field is known at time $t=0$, and the magnetic field is known at the earlier time $t = -\Delta t/2$.  The update equations~A are then used with the initial conditions to obtain the magnetic field at time $t=\Delta t/2$.  Next, the update equations~B are used with the magnetic field just obtained and the electric field at time $t=0$ to obtain the electric field at time $t=\Delta t$.  This procedure of alternately applying update equations~A and~B to advance the solution in time is known as ``marching-in-time'' or ``stepping-in-time.''  It is repeated until the electromagnetic field is known throughout the computational volume at the desired time $t=t_\text{max} = N_t \Delta t$.

The choice for the increments of space and time ($\Delta x, \Delta y, \Delta z,$ and $\Delta t$) is critical to the success of the algorithm, because their size determines how well the solution to the difference equations approximates the solution to Maxwell's equations.  The spatial and temporal increments cannot be chosen independently; for convergence (as $\Delta x \rightarrow 0$, $\Delta t \rightarrow 0$, etc.) and stability of the algorithm, the increments must satisfy the Courant-Friedrichs-Lewy condition, which for free space is

\begin{equation}
c \Delta t \sqrt{ \frac{1}{\Delta x^2} +  \frac{1}{\Delta y^2} + \frac{1}{\Delta z^2} } \leq 1.
\label{eqn:ACE2}
\end{equation}

\noindent For cubical cells, $\Delta x = \Delta y = \Delta z$, Equation~(\ref{eqn:ACE2}) becomes $S=c \Delta t / \Delta x \leq \sqrt{1/3}$, where $S$ is referred to as the ``Courant number,'' and a reasonable choice is $S=1/2$.

Additional restrictions on the spatial and temporal increments can only be obtained from knowledge of the variation of the field (the solution) in space and time.  We must make $\Delta z$ and $\Delta t$ in Equation~(\ref{eqn:ACE1}) small enough that the errors incurred by replacing the derivatives by the ratios of differences are acceptable.  One obvious requirement is that the size of the spatial cells must be small enough to resolve all of the important structural features and the local field surrounding these features.  Another requirement is that the error introduced by a phenomenon known as ``numerical dispersion'' must be negligible.

When there is numerical dispersion, a pulse that starts out with one shape ends up with a different shape after propagating through the FDTD lattice.  Numerical dispersion is caused by the different frequency components of the pulse propagating through the lattice with different phase velocities.  It can be quantified by considering a time-harmonic plane wave of angular frequency $\omega$ propagating in free space along one of the axes of the FDTD lattice, say the $x$ axis.  Assuming cubical cells, the numerical phase velocity, $\bar{v}_p$, for the wave, normalized to the speed of light in free space $c$, is

\begin{equation}
\frac{\bar{v}_p}{c} = \pi \left \{ N_\lambda \sin^{-1} \left[ \frac{1}{S} \sin\left( \frac{\pi S}{N_\lambda} \right) \right] \right \} ^{-1},
\label{eqn:ACE3}
\end{equation}

\noindent in which $N_\lambda = \lambda / \Delta x$ is the number of cells per wavelength \cite{Schneider99}.  Figure~\ref{fig:ACF2} is a graph of this equation showing the relative error in the phase velocity in percent (solid line) and a related quantity, the error in the phase per cell in degrees (dashed line).  Notice that the phase velocity is less than the speed of light, and that the error decreases monotonically with an increase in $N_\lambda$.  For large $N_\lambda$ (say $N_\lambda > 10$), the error in the phase velocity is approximately $(\pi^2 / 6)(1-S^2)/N^2_\lambda$, so halving the cell size reduces the error by a factor of four.  In theory, any desired accuracy can be obtained by increasing $N_\lambda$.

\begin{figure}
\begin{center}
\includegraphics[angle=0,scale=0.6]{/Users/jim.maloney/Book/images/AntChapOrigFig2.png}
\end{center}
\caption{Numerical dispersion as a function of the number of cells per wavelength, $N_\lambda$, for a time-harmonic plane wave propagating along one of the axes of an FDTD lattice of cubical cells. Solid line: the relative error in the phase velocity in percent. Dashed line: the error in the phase per cell in degrees. $S=0.5$.}
\label{fig:ACF2}
\end{figure}

Ideally, given an electromagnetics problem, we would like to estimate accurately the computational resources (computer memory and execution time) required to solve the problem using the FDTD method.  This estimate is highly dependent on the problem and the computer being used.  In practice, the estimate is usually made by comparing the requirements for the problem under consideration with those of a ``benchmark problem'' that has been run using a particular FDTD code on a particular computer.  Even though the specific requirements are computer dependent, general rules for the scaling of the required memory and execution time with cell size are easily obtained.

Consider a computational volume that is a cube composed of cubical FDTD cells; then the total number of cells is $N=N^3_x$.  Because only the most recent values of the electric and magnetic fields are needed at each step of the algorithm, the total storage required scales as $N$ or $N^3_x$, i.e., as the third power of the number of cells along the edge of the cubical volume.  The simulation must be run for a time roughly proportional to that required for light to cross the volume, $t_\text{max} \propto N_x \Delta x / c$.  Thus, the number of time steps required is $N_t = t_\text{max} / \Delta t \propto N_x / S \propto N_x$.  The execution time is proportional to the product of the number of cells with the number of times the cells must be updated: $N \times N_t \propto N^4_x$.  The execution time therefore scales as the fourth power of the number of cells along the edge of the cubical volume.  If we halve the dimensions of the cells, the storage will increase by a factor of~8, and the execution time will increase by a factor of~16.

\section{Formulation of the Antenna Problem in the FDTD Method}

Antennas are customarily used in two states: transmission and reception.  While the two states are related through the reciprocity inherent in Maxwell's equations, not all quantities for one state can be obtained from the other.  Thus, we must have two separate FDTD formulations for the antenna problem: one for the transmitting antenna and the other for the receiving antenna.

\subsection{Transmitting Antenna}

Figure~\ref{fig:ACF3}(a) is a schematic drawing showing the basic elements involved in the FDTD analysis of a transmitting antenna.  The figure is for a cross section through the computational volume, and the antenna is located near the center of the volume.  The arrangement used to excite the antenna is shown in Figure~\ref{fig:ACF4}(a).  The antenna is connected to the source by a transmission line (waveguide) of characteristic impedance $R_o$, and the source is matched to the characteristic impedance (there is no reflection for a wave entering the source).\footnote{Throughout this appendix we will assume that the characteristic impedance of a transmission line is real, a resistance.}  The specified excitation is the outward-propagating (incident) voltage wave $V^\text{+}_t(t)$ for a single mode at the reference plane in the line.  At this reference plane there is also a voltage $V^\text{-}_t(t)$ associated with an inward-propagating (reflected) wave.

The finite computational volume in Figure~\ref{fig:ACF3}(a) is surrounded by an absorbing boundary.  The objective for this boundary is to reproduce at its interior surface the same conditions for the electromagnetic field that would exist if the volume were infinite.  Stated differently, if we consider the electromagnetic field within the volume to be composed of a spectrum of plane waves, both outward propagating and evanescent, all of these waves should be absorbed without reflection by the boundary.  The most effective absorbing boundaries in use today are the perfectly matched layers (PMLs).  Their implementation is discussed in the literature \cite{Gedney96}, \cite{Gedney2005}.

\begin{figure}
\includegraphics[angle=0,width=\linewidth]{/Users/jim.maloney/Book/images/AntChapOrigFig3.png}
\caption{(a) Schematic drawing showing the basic elements involved in the FDTD analysis of a transmitting antenna. (b) Details for the near-field to far-field transformation.}
\label{fig:ACF3}
\end{figure}

\begin{figure}
\includegraphics[angle=0,width=\linewidth]{/Users/jim.maloney/Book/images/AntChapOrigFig4.png}
\caption{The details for the feed region of (a) the transmitting antenna and (b) the receiving antenna. The characteristic impedance of the transmission line is $R_o$, and the source and termination are matched to this impedance.}
\label{fig:ACF4}
\end{figure}

The FDTD method provides the electromagnetic field for all lattice points within the finite computational volume.  However, for many antenna applications, we need the radiated or far-zone field, which is the field in the limit as the radial distance from the antenna becomes infinite ($r\rightarrow\infty$).  This field can be obtained by applying a near-field to far-field (NFFF) transformation.  For this transformation, a closed surface $S$ is placed around the antenna and inside the absorbing boundary, as shown by the dashed line in Figure~\ref{fig:ACF3}.  The field ($E^t$ and $H^t$) on this surface is obtained for the time period of interest and used to calculate the following electric and magnetic surface current densities:

\begin{equation}
J_s(r^\prime,t) = \hat{n}\times H^t(r^\prime,t),
\label{eqn:ACE4}
\end{equation}
\begin{equation}
M_s(r^\prime,t) = -\hat{n}\times E^t(r^\prime,t).
\label{eqn:ACE5}
\end{equation}

\noindent Here, as shown in Figure~\ref{fig:ACF3}(b), $r^\prime$ locates a point on the surface, and $\hat{n}$ is the outward-pointing unit vector normal to the surface at that point.  Outside the surface $S$, these currents produce the same electromagnetic field as the transmitting antenna ($E^t$, $H^t$), and inside the surface they produce a null field ($E=0$, $H=0$).

At the position $r$, the radiated or far-zone field (indicated by the additional superscript $r$) is obtained using these currents with a version of Huygens' principle for electromagnetic fields \cite{SmithBook}:

\begin{equation}
E^{tr}(r,t) = \frac{\mu_o}{4\pi r} \iint \limits_S  \left\{  \hat{r} \times \hat{r} \times \frac{\partial}{\partial t^\prime} \left[ J_s(r^\prime,t^\prime) \right]  + \frac{1}{\eta_o}\hat{r} \times \frac{\partial}{\partial t^\prime} \left[ M_s(r^\prime,t^\prime) \right] \right\}_{t^\prime = t_r} dS^\prime ,
\label{eqn:ACE6}
\end{equation}

\begin{equation}
H^{tr}(r,t) =  \frac{1}{\eta_o}\hat{r} \times E^{tr}(r,t) ,
\label{eqn:ACE7}
\end{equation}

\noindent in which the retarded time is

\begin{equation}
t_r = t-(r-\hat{r}\cdot\vec{r}^{\text{ }\prime}) / c
\label{eqn:ACE8}
\end{equation}

\noindent and $\eta_o = \sqrt{\mu_o /  \epsilon_o}$ is the wave impedance of free space.

In some situations, we may require the near field at points that are so far from the antenna that it is impractical to extend the computational volume to include them.  In such cases, a near-field to near-field (NFNF) transformation can be used: the FDTD analysis is performed for a volume such as that shown in Figure~\ref{fig:ACF3}(a), and the field on the surface of the volume is transformed to obtain the near field outside the volume.  Details for the NFNF transformation can be found in \cite{Shlager94}, \cite{Shlager95}.

\subsection{Receiving Antenna}

Figure~\ref{fig:ACF5}(a) is a schematic drawing showing the basic elements involved in the FDTD analysis of a receiving antenna.  As for the transmitting antenna, the figure is for a cross section through the computational volume, and the finite computational volume is surrounded by an absorbing boundary.  The excitation for the antenna is an incident, transverse electromagnetic (TEM) plane wave propagating in the direction $\hat{k}_i$ with the field

\begin{equation}
E^i(r,t), \quad H^i(r,t) = \frac{1}{\eta_o}\hat{k}_i \times E^i(r,t) .
\label{eqn:ACE9}
\end{equation}

\noindent Here, the vector $E^i$ is transverse to $\hat{k}_i$, viz, $\hat{k}_i \cdot E^i = 0$.

The closed surface $S$ with outward-pointing unit normal vector $\hat{n}$ is placed around the antenna and inside the absorbing boundary.  As shown in Figure~\ref{fig:ACF5}(b), the following electric and magnetic surface current densities are placed on this surface to produce the incident field ($E^i$, $H^i$) inside the surface and a null field ($E=0$, $H=0$) outside the surface:

\begin{equation}
J_s(r,t) =-\hat{n}\times H^i(r,t),  \quad M_s(r,t)=\hat{n}\times E^i(r,t) .
\label{eqn:ACE10}
\end{equation}

For the receiving antenna, we generally want to know the voltage produced in the antenna by the incident wave.  The arrangement used to accomplish this is shown in Figure~\ref{fig:ACF4}(b).  The antenna is connected to the termination by a transmission line (waveguide) of characteristic impedance $R_o$, and the termination is matched to the characteristic impedance (there is no reflection for a wave entering the termination).  The desired response is the inward-propagating voltage wave $V^\text{-}_r(t)$ for a single mode at the reference plane in this line.

The scattered field is the field produced by the currents induced in the antenna by the incident field.  Notice from Figure~\ref{fig:ACF5}(a) that the field inside the surface $S$ is the total field, i.e., the sum of the incident and scattered fields ($E^r=E^i+E^s$, $H^r=H^i+H^s$).  However, the field outside the surface, in the region between the surface and the absorbing boundary, is only the scattered field ($E^s$, $H^s$).  If we are interested in the scattering properties of the antenna, we can obtain them from knowledge of the field in this region.  For example, the far-zone scattered field can be determined using a near-field to far-field transformation, as in the case for the transmitting antenna.  The surface for the transformation must be placed between the surface for the plane-wave source and the absorbing boundary.

\begin{figure}
\includegraphics[angle=0,width=\linewidth]{/Users/jim.maloney/Book/images/AntChapOrigFig5.png}
\caption{(a) Schematic drawing showing the basic elements involved in the FDTD analysis of a receiving antenna. (b) Details for the plane-wave source.}
\label{fig:ACF5}
\end{figure}

\subsection{Reciprocity}

As mentioned earlier, some quantities for the states of transmission and reception are related through reciprocity.  For example, when the arrangements shown in Figure~\ref{fig:ACF4} are used for the source and termination, the following relationship applies \cite{Smith2004}:

\begin{equation}
V^\text{+}_t(t) \ast V^\text{-}_r(t) = \frac{2\pi R_o}{\eta_o} \left[ c \int \limits_{t^\prime = -\infty}^{t} E^i(0,t^\prime)dt^\prime \right] \cdot \ast \left[ r E^{tr}(-r\hat{k}_i,t+r/c)\right] ,
\label{eqn:ACE11}
\end{equation}

\noindent in which $\ast$ indicates time convolution, and $\cdot\ast$ indicates the scalar product with time convolution.  Here, the origin for the spherical coordinate system is centered on the antenna, as in Figure~\ref{fig:ACF4}(a), and the incident electric field $E^i$ is evaluated at the origin ($r=0$) of this system.  The radiated electric field $E^{tr}$ is evaluated at the radial distance $r$ in the direction ($-\hat{k}_i$) from which the incident field arrives and at the time $t+r/c$.  This relationship can sometimes be used to eliminate the need for analyzing one of the two states (transmission or reception) when the other is known, or it can be used for verifying results from one state with results from the other.

\subsection{Frequency Domain}

The FDTD method is inherently a time-domain technique.  When quantities are needed in the frequency domain (angular frequency $\omega$), they are obtained using the discrete Fourier transformation, which is indicated by $V(t) \leftrightarrow V(\omega)$.  The quantities customarily used for evaluating the performance of an antenna in the frequency domain are determined from the transformed variables.  For the transmitting antenna, the voltage reflection coefficient $\Gamma_A$ and input impedance $Z_A$ are

\begin{equation}
\Gamma_A(\omega) = \frac{V^\text{-}_t(\omega)}{V^\text{+}_t(\omega)} ,
\label{eqn:ACE12}
\end{equation}

\begin{equation}
Z_A(\omega) = R_o \left[ \frac{1+\Gamma_A(\omega)}{1-\Gamma_A(\omega)} \right],
\label{eqn:ACE13}
\end{equation}

\noindent and the realized gain $G_\text{Rel}$ (gain including mismatch) and gain $G$ in the direction $\hat{r}$ are

\begin{equation}
G_\text{Rel}(\hat{r},\omega) = \frac{4\pi r^2\hat{r} \cdot \text{Re} [ S^{tr}_c(r,\omega) ] }{\text{Power available from source}} = \frac{4\pi R_o r^2 | E^{tr}(r,\omega) |^2}{\eta_o | V^\text{+}_t(\omega) |^2} ,
\label{eqn:ACE14}
\end{equation}

\begin{equation}
G(\hat{r},\omega) = \frac{4\pi r^2\hat{r} \cdot \text{Re} [ S^{tr}_c(r,\omega) ] }{\text{Power accepted by antenna}} = \frac{1}{1 - | \Gamma_A(\omega) |^2} G_\text{Rel}(\hat{r},\omega) ,
\label{eqn:ACE15}
\end{equation}

\noindent in which $S_c$ is the complex Poynting vector.

For the receiving antenna, the realized effective area $A_\text{Rel}(\hat{k}_i, \omega)$ and the effective area $A_e(\hat{k}_i,\omega)$ for an incident plane wave propagating in the direction $\hat{k}_i$ are

\begin{equation}
A_\text{Rel}(\hat{k}_i,\omega) = \frac{\text{Power accepted by termination}}{\hat{k}_i \cdot \text{Re} [ S^{i}_c(r,\omega) ] } =  \frac{\eta_o}{R_o}  \frac{| V^\text{-}_r(\omega) |^2}{| E^i(\omega) |^2} ,
\label{eqn:ACE16}
\end{equation}

\noindent and

\begin{equation}
A_e(\hat{k}_i,\omega) = \frac{\text{Power available from antenna}}{\hat{k}_i \cdot \text{Re} [ S^{i}_c(r,\omega) ] } = \frac{1}{1 - | \Gamma_A(\omega) |^2} A_\text{Rel}(\hat{k}_i,\omega) .
\label{eqn:ACE17}
\end{equation}

The gain and the effective area are related through reciprocity (Equation~\ref{eqn:ACE11}); for a polarization match we have\footnote{For a polarization match, the state of polarization for the incident plane wave in a particular direction (reception) is matched to the state of polarization for the radiated field in the same direction (transmission).  For example, if the radiated electric field is linearly polarized, the electric field of the incident plane wave is linearly polarized and points in the same direction.  If the radiated electric field is right-handed circularly polarized, the electric field of the incident plane wave is right-handed circularly polarized.}

\begin{equation}
G(\hat{r},\omega) = \frac{4\pi}{\lambda^2} A_e(-\hat{r},\omega) .
\label{eqn:ACE18}
\end{equation}

\subsection{Input Signals}

When we are interested in the performance of an antenna over a band of frequencies, a pulsed input signal is useful, followed by the Fourier transform to obtain the desired frequency-domain response.  A natural choice for the pulse shape is the Gaussian pulse shown as a solid line in Figure~\ref{fig:ACF6}(a),

\begin{align}
\begin{split}
f(t) & =\text{exp} \left[ -(t/\tau_p)^2 / 2 \right], \\
F(\omega) & =\sqrt{2\pi}\,\tau_p\,\text{exp}\left[-(\omega \tau_p)^2/2 \right],
\end{split}
\label{eqn:ACE19}
\end{align}

\noindent in which $\tau_p$ is the characteristic time.  However, the spectrum for the Gaussian pulse contains significant low-frequency content (including dc), and this usually is not radiated by the antenna (the dc component never is).  Thus, the field near the antenna may take an unacceptably long time to settle when a Gaussian pulse is used.

A better choice is the differentiated Gaussian pulse shown as a dashed line in Figure~\ref{fig:ACF6}(a),

\begin{align}
\begin{split}
f(t) & =-\left( \frac{t}{\tau_p} \right) \text{exp} \left\{ -\left[ (t/\tau_p)^2-1 \right] / 2 \right\}, \\
F(\omega) & =j\sqrt{2\pi}\,\omega \tau^2_p\,\text{exp}\left\{-\left[(\omega \tau_p)^2-1\right]/2 \right\},
\end{split}
\label{eqn:ACE20}
\end{align}

\noindent or the sinusoid of frequency $\omega_o$ amplitude modulated by a Gaussian pulse shown in Figure~\ref{fig:ACF6}(b),

\begin{align}
\begin{split}
f(t) & =\text{exp} \left[ -(t/\tau_p)^2 / 2 \right] \sin(\omega_o t), \\
F(\omega) & =j\sqrt{\pi/2}\,\tau_p \left\{ \exp\left[ -(\omega+\omega_o)^2\tau^2_p / 2 \right] - \exp\left[ -(\omega-\omega_o)^2\tau^2_p / 2 \right] \right\}.
\end{split}
\label{eqn:ACE21}
\end{align}

\noindent \textcolor{red}{\textbf{[Note: Verify the coefficient in Equation~\ref{eqn:ACE21}; the Fourier transform expression was incomplete in the source material.]}}

\begin{table}
\caption{Characteristics for Various Input Signals.}
\label{tab:ACT1}
\textcolor{red}{\textbf{[Missing content: Table data for characteristics of various input signals was not provided in the source material.]}}
\end{table}

The differentiated Gaussian pulse has a rather large fractional bandwidth that is fixed; for example, the bandwidth associated with the points at which the spectrum is 10\% ($-20$~dB) of the maximum is \textcolor{red}{\textbf{[missing value]}}, where $\omega_\text{pk}=1/\tau_p$ is the frequency at the peak.  The modulated sinusoid has a variable fractional bandwidth that is controlled by the relative width of the modulating pulse, $\omega_o\tau_p$; for example, the bandwidth associated with the points at which the spectrum is 10\% of the maximum is $\Delta\omega / \omega_o \approx 4.29/\omega_o\tau_p$ (when $\omega_o\tau_p \gg 1$).  For the case shown in Figure~\ref{fig:ACF6}(b), $\omega_o\tau_p=15$, so the fractional bandwidth is $\Delta\omega/\omega_o \approx 0.29$, which is much narrower than the fractional bandwidth for the differentiated Gaussian pulse shown in Figure~\ref{fig:ACF6}(a).

\begin{figure}
\includegraphics[angle=0,width=\linewidth]{/Users/jim.maloney/Book/images/AntChapOrigFig6.png}
\caption{(a) The Gaussian pulse (solid line) and the differentiated Gaussian pulse (dashed line) and the magnitude of their Fourier transforms. (b) The sinusoid of frequency $\omega_o$ amplitude modulated by a Gaussian pulse and the magnitude of its Fourier transform. All waveforms are normalized to have a maximum value of 1.0.}
\label{fig:ACF6}
\end{figure}

\section{Examples of the Use of the Method for Antenna Analysis}

In the previous sections, we presented the fundamentals of the FDTD method and described in general how the method is used to analyze an antenna for both transmission and reception.  In this section, we show results obtained by applying the method to analyze particular antennas.  These examples were chosen to illustrate specific issues that arise and must be dealt with when applying the method.

\subsection{Cylindrical Monopole: Theoretical Model Versus Experimental Model}

The ultimate test for any physical theory is how well its predictions agree with experimental measurements, and this is certainly the case for electromagnetic theory when applied to antennas.  One of the most important factors that affect the agreement is how closely the theoretical model for the antenna matches the experimental model.  To examine this issue we consider the FDTD analysis of the cylindrical monopole, the image equivalent of the cylindrical dipole, which is arguably the most fundamental antenna.

The monopole antenna, shown in Figure~\ref{fig:ACF7}(a), is formed by extending the metallic center conductor of a coaxial line the distance $h$ above an infinite metallic image plane \cite{Maloney1}, \cite{SmithBook}.  The dimensions of the transmission line, inner conductor radius $a$ and outer conductor radius $b$, are chosen so that only the TEM mode propagates in the line for the signals of interest.  The FDTD model for the transmitting monopole is shown in Figure~\ref{fig:ACF7}(b).  All of the conductors in the model are perfect (perfect electric conductors, PECs), and the structure is surrounded by a PML \cite{Maloney97}.  Because of the rotational symmetry of the structure and the excitation, a two-dimensional cylindrical lattice $(\rho,z)$ with the spatial increments $\Delta\rho$ and $\Delta z$ is used in the FDTD analysis.  A ``one-way source'' excites the coaxial line, consisting of the electric and magnetic surface currents

\begin{figure}
\includegraphics[angle=0,width=\linewidth]{/Users/jim.maloney/Book/images/AntChapOrigFig7.png}
\caption{(a) Cylindrical monopole antenna fed through an image plane from a coaxial transmission line. (b) FDTD model for the cylindrical monopole antenna. The PML that surrounds the computational space is not shown.}
\label{fig:ACF7}
\end{figure}

\begin{equation}
J_s(\rho,t) = - \frac{V^\text{+}_t(t)}{2\pi R_o\rho} \hat{\rho}, \quad M_s(\rho,t)=-\frac{\eta_o V^\text{+}_t(t)}{2\pi R_o \rho} \hat{\phi}
\label{eqn:ACE22}
\end{equation}

\noindent on the plane $z=-l$ that produce the incident TEM voltage wave, $V^\text{+}_t$, above the source and a null field below the source.  An absorbing boundary is placed at the bottom of the line.  With this configuration, only the reflected TEM voltage wave, $V^\text{-}_t$, appears below the source, so it is easily determined.  Notice the similarity of this arrangement to the plane-wave source used with the receiving/scattering antenna in Figure~\ref{fig:ACF5}.

Figure~\ref{fig:ACF8} is a comparison of results from the FDTD simulation (solid line) with measurements (dots) made on an experimental model corresponding to the geometry in Figure~\ref{fig:ACF7}(a).  The height of the monopole is $h = 5.0$~cm, and the dimensions of the coaxial line (precision line with APC-7 connector) are $a = 1.52$~mm, $b = 3.5$~mm, which gives a characteristic impedance of $R_o = (\eta_o/2\pi) \ln(b/a) = 50~\Omega$.  The excitation $V^\text{+}_t$ is a unit-amplitude Gaussian pulse in time (Equation~\ref{eqn:ACE19}), with the characteristic time $\tau_p = 0.161\tau_a$, where $\tau_a=h/c$ is the time for light to travel the length of the monopole.

Figure~\ref{fig:ACF8}(a) is for the reflected voltage, $V^\text{-}_t$, in the transmission line, and Figure~\ref{fig:ACF8}(b) is for the electric field on the image plane at the radial distance $\rho/h=12.7$; both are shown as a function of the normalized time $t/\tau_a$.  In Figure~\ref{fig:ACF8}(a), we see the initial reflection of the incident pulse from the drive point~(A), followed by its initial reflection from the open end of the monopole~(B).  As expected, these events are separated by roughly the time for light to make a round trip on the monopole, $(t_B-t_A)/\tau_a \approx 2$.  Additional reflections of decreased amplitude occur each time the pulse encounters the drive point and the open end.  In Figure~\ref{fig:ACF8}(b), we see that radiation occurs each time the pulse encounters the drive point or the open end of the monopole.  As expected, the initial radiation from the drive point~(A) is separated from the initial radiation from the open end of the monopole~(B) by roughly the time for light to travel the length of the monopole, $(t_B-t_A)/\tau_a \approx 1$.  The agreement between the theoretical and measured results is very good.

\begin{figure}
\includegraphics[angle=0,width=\linewidth]{/Users/jim.maloney/Book/images/AntChapOrigFig8.png}
\caption{Comparison of theoretical and measured results for the cylindrical monopole antenna. (a) Reflected voltage in the coaxial line. (b) Electric field on the image plane at $\rho/h = 12.7$.}
\label{fig:ACF8}
\end{figure}

The FDTD method inherently provides information about the electromagnetic field within the computational volume over the entire period of the simulation.  Only a small fraction of this information is used when investigating conventional antenna parameters, such as the results shown in Figure~\ref{fig:ACF8}.  Sometimes this additional information can be used to perform ``numerical experiments'' that improve our understanding of the radiation process for the antenna.  This is illustrated in Figure~\ref{fig:ACF9}, which shows the instantaneous Poynting vector in the region surrounding the monopole \cite{Smith2001}.  On the right-hand side of these figures, the logarithm of the magnitude of the Poynting vector, $|S|$, is plotted on a color scale.  The intensity of the field increases as the hue goes from blue to red, and the range for the values of $|S|$ displayed is $10^4:1$.  On the left-hand side, the arrows indicate the direction of the Poynting vector, and the length of an arrow is proportional to the logarithm of $|S|$.  The excitation is a Gaussian voltage pulse with $\tau_p = 0.0537\tau_a$.  For this value of $\tau_p$, about three non-overlapping pulses fit along the length of the monopole, so the reflections associated with different points are separated and easily identified.

\begin{figure}
\includegraphics[angle=0,width=\linewidth]{/Users/jim.maloney/Book/images/AntChapOrigFig9.png}
\caption{Three snapshots in time showing the magnitude (right) and direction (left) of the Poynting vector surrounding the cylindrical monopole antenna. Logarithmic scaling is used for both plots. Notice that (a) and (b) only show a portion of the monopole. (After Smith and Hertel \cite{Smith2001}, \copyright~2001 IEEE.)}
\label{fig:ACF9}
\end{figure}

In Figure~\ref{fig:ACF9}(a), the pulse has just left the drive point and is traveling up the monopole.  A spherical wavefront $W_1$ centered on the drive point has formed, and it is attached to the outward-propagating pulses of charge/current on the monopole and image plane.  In Figure~\ref{fig:ACF9}(b), the pulse has encountered the open end of the monopole, and it is traveling back down the monopole.  A second spherical wavefront $W_2$ centered on the open end has formed, and it connects the inward-propagating pulse of charge/current on the monopole with the wavefront $W_1$.  Additional wavefronts, $W^\prime_2$, $W_3$, etc., shown in Figure~\ref{fig:ACF9}(c), are produced each time the pulse encounters the drive point and the open end.  All of these spherical wavefronts travel outward at the speed of light.  The Poynting vectors are seen to be predominantly normal to the wavefronts, which indicates that energy is being transported away from both the drive point and the open end.

The input impedance or admittance $Y_A(\omega) = 1/Z_A(\omega) = G_A(\omega) + jB_A(\omega)$ of the monopole antenna is a useful parameter for practical applications, and it is also a sensitive measure of the accuracy of any theoretical model.  It is easily calculated from the FDTD time-domain results using Equations~(\ref{eqn:ACE12}) and~(\ref{eqn:ACE13}).  In Figure~\ref{fig:ACF10} the input admittance is graphed as a function of frequency for a monopole with the same dimensions as used for Figure~\ref{fig:ACF8} \cite{Smith2003}.  FDTD results (lines) for three different levels of discretization~(A, B, C) are compared with measurements (dots).  The parameters for the three levels of discretization are given in Table~\ref{tab:ACT2}.

\begin{table}
\caption{FDTD Discretization Parameters for the Cylindrical Monopole Antenna.}
\label{tab:ACT2}
\textcolor{red}{\textbf{[Missing content: Table data for discretization parameters (A, B, C) was not provided in the source material.  Should include cell size, number of cells across the gap, number of cells along monopole, and cells per wavelength at the highest frequency.]}}
\end{table}

In this graph we observe the convergence of the FDTD method.  Consider the input susceptance, $B_A$; the result for discretization~A is slightly displaced from the measured values, while the results for discretizations~B and~C are essentially the same as the measured values.  Hence, we can conclude that, for practical purposes, the FDTD results for the input admittance have converged to the measured values at discretization~B, which corresponds to four FDTD cells across the gap in the coaxial line or 101 cells along the length of the monopole.  Note that the dimensions of the FDTD cell for this example had to be chosen so that an integral number of cells fit along the dimensions of the antenna, so the cells are not perfectly square.  Discretization~B corresponds to 135 cells per wavelength at the highest frequency ($f=4.5$~GHz) and a relative error in the phase velocity (Equation~\ref{eqn:ACE3} and Figure~\ref{fig:ACF2} for $S=0.5$) of only $6.77\times10^{-3}\%$.  For this example, it is not the error in the phase velocity that determines the accuracy of the solution.  The fine details of the structure must be accurately modeled, and this requires cells that are much smaller than needed for a small error in the phase velocity.

\begin{figure}
\begin{center}
\includegraphics[angle=0,scale=0.5]{/Users/jim.maloney/Book/images/AntChapOrigFig10.png}
\end{center}
\caption{Comparison of theoretical and measured results for the input admittance of the cylindrical monopole antenna. Results are shown for three levels of discretization (A, B, C) in the FDTD method. (After Hertel and Smith \cite{Smith2003}, \copyright~2003 IEEE.)}
\label{fig:ACF10}
\end{figure}


The very good agreement between the theoretical results and the measurements evident in Figures~\ref{fig:ACF8} and~\ref{fig:ACF10} is a consequence of the close match of the theoretical model for the monopole, Figure~\ref{fig:ACF7}(b), to the experimental model, Figure~\ref{fig:ACF7}(a).  In some cases, additional constraints on the analysis require a reduction in the fidelity of the FDTD model, and such good agreement cannot be expected.  To illustrate the effect a reduction in the fidelity of the model can have on the accuracy of the results, we examine some common simplifications used for the FDTD model of the monopole.

For the models shown in Figure~\ref{fig:ACF11}, the cylindrical conductor of the monopole has been replaced by an equivalent square conductor \cite{King56}.  Thus, the monopole can now be analyzed using the conventional three-dimensional rectangular FDTD lattice rather than the two-dimensional cylindrical lattice of Figure~\ref{fig:ACF7}(b).

\begin{figure}
\includegraphics[angle=0,width=\linewidth]{/Users/jim.maloney/Book/images/AntChapOrigFig11.png}
\caption{Simplified models for the cylindrical monopole antenna. (a) Model incorporating a ``hard source.'' (b) Model incorporating a virtual one-dimensional transmission line. The monopole conductor has a square cross section in both models.}
\label{fig:ACF11}
\end{figure}

The excitation for the monopole has also been changed from that in Figure~\ref{fig:ACF7}(b).  For the model in Figure~\ref{fig:ACF11}(a), the so-called ``hard source'' is used.  This specifies the total voltage $V_t=V^\text{+}_t + V^\text{-}_t$ across the gap of length $l_g$ at the base of the monopole.  For the model in Figure~\ref{fig:ACF11}(b), a virtual one-dimensional transmission line is connected across the gap at the base of the monopole \cite{Maloney94}.  This transmission line contains the same elements as the transmission line in Figure~\ref{fig:ACF7}(b), in particular, a one-way source that specifies the incident voltage $V^\text{+}_t$.  We refer to this line as virtual because it does not appear in the FDTD lattice surrounding the monopole.  It is in a different location and coupled to the monopole through the voltage and current at its terminals.  The hard source, while simple to implement, suffers from two drawbacks not present with the transmission line feed.  There is no damping in the hard source, unless resistance is added, so the currents on the antenna can ring for a long period of time.  Also, the total voltage is specified, so the reflected voltage, a quantity often of interest in time-domain simulations, is not readily available.

In Figure~\ref{fig:ACF12}, FDTD results for the input admittance for both models in Figure~\ref{fig:ACF11} are compared with measurements made with the configuration shown in Figure~\ref{fig:ACF7}(b) \cite{Smith2003}.  The level of discretization used is such that the simulations have converged for practical purposes.  The theoretical results for the input conductance, $G_A$, for both models are in very good agreement with the measurements; however, those for the input susceptance, $B_A$, differ from the measurements, particularly for the hard source (dashed line).  The difference in susceptance is a consequence of the geometry for the simplified models not accurately representing the experimental model, Figure~\ref{fig:ACF7}(a), in the vicinity of the drive point (the aperture of the coaxial line).  The susceptance for the simplified models can be brought into better agreement with the measured results by adding a small capacitance in parallel with the terminals of the monopole \cite{Smith2003}.

\begin{figure}
\begin{center}
\includegraphics[angle=0,scale=0.5]{/Users/jim.maloney/Book/images/AntChapOrigFig12.png}
\end{center}
\caption{Comparison of theoretical and measured results for the input admittance of the cylindrical monopole antenna. Results are shown for the two simplified FDTD models. (After Hertel and Smith \cite{Smith2003}, \copyright~2003 IEEE.)}
\label{fig:ACF12}
\end{figure}

\subsection{Metallic Horns and Spirals: Stair-Stepped Surfaces}

For the monopole antennas discussed in the previous section, the boundaries of the FDTD cells as well as the boundaries of all material regions (PECs) coincided with surfaces of constant coordinate.  Thus, the boundaries of material regions never passed obliquely through an FDTD cell.  This is a special case that is not encountered for most antennas.

Figure~\ref{fig:ACF13}(a) illustrates the more general case.  It shows the cross section of a PEC object with the rectangular FDTD lattice superimposed.  The curved surface of the object does not coincide with any of the lattice lines.  For the computation we only need to know the field in the FDTD cells that are exterior to the PEC, because both $E$ and $H$ are zero inside the PEC.  There are different approaches that can be used for this case.  One approach is to introduce non-rectangular FDTD cells that conform to the surface of the object; these cells could be used throughout the computational volume or just adjacent to the object \cite{DeyMittra}--\cite{Gedney05}.  Another much simpler approach, shown in Figure~\ref{fig:ACF13}(b), is to deform the curved surface of the object so that it conforms to the rectangular FDTD lattice.  The surface of the object is said to be replaced by a ``stair-stepped'' or ``staircase'' approximation.  The stair-stepped approximation will introduce an error, but often the error can be made negligible by choosing the size of the staircase to be small compared to the physical dimensions of the object \cite{Cangellaris}, \cite{Holland}.  The stair-stepped approximation is commonly used, and it is the only approach we will consider in this introductory treatment.

\begin{figure}
\begin{center}
\includegraphics[angle=0,scale=0.8]{/Users/jim.maloney/Book/images/AntChapOrigFig13.png}
\end{center}
\caption{(a) Rectangular FDTD lattice superimposed on the cross section of an object that is a perfect electric conductor (PEC). (b) The surface of the object has been deformed to conform to the rectangular lattice; the surface of the object has been replaced by a stair-stepped approximation.}
\label{fig:ACF13}
\end{figure}

We now consider two practical antennas for which the stair-stepped approximation was used in modeling the structure in the FDTD analysis.  As these examples will show, when properly used, the approximation can yield results that are in good agreement with experimental measurements.  The first example is the metallic, pyramidal horn shown in Figure~\ref{fig:ACF14} (Flann Microwave Instruments Ltd.\ Model 1624-20).  Antennas like this are used in many microwave applications, and sometimes they serve as gain standards (standard gain horns).  The small drawings at the bottom of the figure show the lengths and angles that describe this particular horn antenna: $a = 10.95$~cm, $b = 7.85$~cm, $D = 2.284$~cm, $l_w = 5.08$~cm, $\alpha = 10.74^\circ$, and $\beta = 8.508^\circ$.  The waveguide feeding the horn is type WR-90 (X-Band, with the operational bandwidth 8.2--12.4~GHz).

\begin{figure}
\includegraphics[angle=0,width=\linewidth]{/Users/jim.maloney/Book/images/AntChapOrigFig14.png}
\caption{Schematic drawing for the pyramidal horn antenna. The inset shows the FDTD cells used to model the bottom of the horn.}
\label{fig:ACF14}
\end{figure}

In the FDTD model, the cubical cells have the side length $\Delta x = 0.635$~mm, and the perfectly-conducting walls are plates two cells thick.  The inset shows the faces of the individual cells that model the bottom wall of the horn; the cells are shown seven times actual size.  The slanted sides of the horn are stair stepped, as indicated in the figure, with a ``tread length-to-rise'' of approximately six cells to one.  The horn is fed by a probe inserted into the section of rectangular waveguide, and the incident and reflected voltages in a one-dimensional transmission line ($R_o=50~\Omega$) connected to the probe are used in the analysis.

The structure is symmetrical about the $x$-$z$ plane, and this symmetry was used in the analysis to reduce the size of the computational volume, which was $519 \times 116 \times 183$ cells.  The sides of the antenna were 20 cells from the PML absorbing boundary (10 cells thick), except the front side (radiating aperture), which was 40 cells from the absorbing boundary.

The pyramidal horn was first analyzed as a transmitting antenna.  The excitation in the transmission line, $V^\text{+}_t(t)$, was a differentiated Gaussian pulse (Equation~\ref{eqn:ACE20}) with the characteristic time $\tau_p = 1.59\times 10^{-11}$~s.  This pulse has significant energy over the operational bandwidth of the horn: 8.2--12.4~GHz.  The peak of the spectrum for the pulse is at 10.0~GHz, and the spectrum drops to 10\% of the peak at 600~MHz and 27.6~GHz.

At the highest frequency (shortest wavelength) within the operational bandwidth of the horn we have $\Delta x=0.0226\lambda$, which corresponds roughly to 38 cells per wavelength.  From this result, we can estimate the numerical dispersion using Figure~\ref{fig:ACF2} or Equation~(\ref{eqn:ACE3}).  The relative error in the phase velocity is about 0.1\%, which is equivalent to $8.1\times 10^{-3}$ degrees of phase error per cell, or a total error of 4.2 degrees of phase error for propagation across the longest side of the computational volume.

Figure~\ref{fig:ACF15} is a comparison of the FDTD results (solid line) for this antenna with measurements (dots).  The measured data were kindly supplied by Dr.\ David G.\ Gentle of the National Physical Laboratory, Teddington, Middlesex, U.K.  Figures~\ref{fig:ACF15}(a) and~\ref{fig:ACF15}(b) show the E- and H-plane field patterns at the frequency 10~GHz, and Figure~\ref{fig:ACF15}(c) shows the gain on boresight as a function of frequency.  The results from the FDTD calculations are in very good agreement with the measurements.  The small differences that do exist in the H-plane field pattern are for angles at which the field is very weak, 50~dB below the peak.  We note that the precise details of the probe feeding the waveguide in the FDTD model do not affect the calculation of the gain (Equation~\ref{eqn:ACE15}) of the horn.  This would not be the case if the realized gain (Equation~\ref{eqn:ACE14}) were determined.

\begin{figure}
\includegraphics[angle=0,width=\linewidth]{/Users/jim.maloney/Book/images/AntChapOrigFig15.png}
\caption{Comparison of theoretical and measured results for the pyramidal horn antenna. (a) E-plane pattern and (b) H-plane pattern at 10~GHz. (c) Boresight gain versus frequency.}
\label{fig:ACF15}
\end{figure}

The pyramidal horn was also analyzed as a receiving antenna.  For this case, a plane wave was incident from the boresight direction ($\hat{k}_i = -\hat{x}$) with the electric field pointing in the $z$ direction.  The incident electric field was a differentiated Gaussian pulse in time (Equation~\ref{eqn:ACE20}) with the same characteristic time as used for the transmitting case, $\tau_p = 1.59 \times 10^{-11}$~s.  The effective area obtained from the receiving analysis was converted to a gain using Equation~(\ref{eqn:ACE18}), and the result is shown as a dashed line in Figure~\ref{fig:ACF15}(c).  As expected from reciprocity, the results from the two FDTD calculations (transmitting and receiving) are nearly identical.

The FDTD method provides the field throughout the computational volume, and it can be used to construct graphical results that illustrate the process of radiation for the transmitting horn antenna.  For such illustrations, we want an excitation whose spectrum lies within the operational bandwidth of the antenna.  Frequencies outside of this band will either be cutoff in the waveguide or overmode the waveguide.  A good choice for the voltage $V^{\text{+}}_t(t)$ is the sinusoid of frequency $\omega_o$ amplitude modulated by a Gaussian pulse, i.e., Equation~(\ref{eqn:ACE21}) shown in Figure~\ref{fig:ACF6}(b).  With $f_o = \omega_o/2\pi = 10.0$~GHz and $\tau_p = 7.96 \times 10^{-11}$~s, the spectrum for this signal is 10\% of its peak at $f = 5.7$~GHz and $f = 14.3$~GHz.

Figure~\ref{fig:ACF16} shows three gray-scale plots for the magnitude of the electric field on the $x$-$z$ plane of the transmitting antenna.  In Figure~\ref{fig:ACF16}(a) the pulse has entered the horn from the waveguide, but it has not reached the aperture.  The spacing between the white lines (nulls) roughly corresponds to one half of a guide wavelength.  Notice that this spacing decreases on going from the throat of the horn towards the aperture.  In the rectangular waveguide, the guide wavelength is about 1.3 times the free-space wavelength, whereas at the aperture of the horn it is closer to the free-space wavelength.  Figure~\ref{fig:ACF16}(b) is for a time when the pulse has reached the aperture.  Notice that the white lines in the horn near the aperture are distorted; there is a small segment that is concave to the right.  This is caused by the reflection from the aperture that is traveling back toward the throat of the horn.  Directly in front of the aperture, the radiated wave is roughly planar.  In Figure~\ref{fig:ACF16}(c), the field has propagated away from the horn, and a spherical wavefront has formed that is approximately centered on the aperture.  The change in the shade of gray in going around the antenna (dark in front to light in back) clearly shows a large ``front-to-back ratio'' for the horn.  In the forward direction, minima appear along the wavefront, and these minima will define the main beam in the far zone.  Back in the horn, the field has several minima and maxima across its width, indicating the presence of higher-order modes that were excited when the initial pulse encountered the aperture.

\begin{figure}\begin{center}
\includegraphics[angle=0,width=\linewidth]{/Users/jim.maloney/Book/images/AntChapOrigFig16.png}
\caption{Gray-scale plots for the magnitude of the electric field on the vertical symmetry plane of the transmitting horn antenna. The excitation is a sinusoid amplitude modulated by a Gaussian pulse.}
\label{fig:ACF16}
\end{center}\end{figure}

The second example is the two-arm, conical spiral antenna shown in Figure~\ref{fig:ACF17} \cite{Hertel02}.  It is used in applications that require an antenna to radiate circular polarization over a broad bandwidth.  This antenna is formed by winding two metallic strips around the surface of a truncated cone.  The angles and dimensions for the particular antenna we consider are $d=1.9$~cm, $D=15.2$~cm, $\theta_o = 7.5^\circ$, $\alpha=75^\circ$, and $\delta = 90^\circ$.  It is designed to have constant gain and input impedance ($Z_A \approx 100~\Omega$) over an operational bandwidth extending from $f_\text{min} = 0.5$~GHz to $f_\text{max}=3.3$~GHz.

\begin{figure}\begin{center}
\includegraphics[angle=0,width=\linewidth]{/Users/jim.maloney/Book/images/AntChapOrigFig17.png}
\caption{Geometry for the two-arm conical spiral antenna. (After Hertel and Smith \cite{Hertel02}, \copyright~2002 IEEE.)}
\label{fig:ACF17}
\end{center}\end{figure}

In the FDTD model, the arms of the spiral are formed by making selected faces of the cubical cells ($\Delta x=0.8$~mm) PEC.  The result is the stair-stepped approximation in Figure~\ref{fig:ACF18}.  For clarity, only the lower 10\% of the antenna is shown in the figure.  The spiral is fed by a one-dimensional transmission line ($R_o = 100~\Omega$) connected at the bottom of the antenna, the same arrangement as used with the monopole antenna in Figure~\ref{fig:ACF11}(b).  The excitation in the transmission line, $V^\text{+}_t(t)$, is a differentiated Gaussian pulse (Equation~\ref{eqn:ACE20}), whose spectrum is centered on the operational bandwidth of the antenna.

\begin{figure}\begin{center}
\includegraphics[angle=0,scale=0.75]{/Users/jim.maloney/Book/images/AntChapOrigFig18.png}
\caption{Schematic drawing showing the arrangement of FDTD cells used to model the conical spiral antenna. For clarity, only the lower 10\% of the antenna is shown. (After Hertel and Smith \cite{Hertel02}, \copyright~2002 IEEE.)}
\label{fig:ACF18}
\end{center}\end{figure}

The computational volume was $691 \times 240 \times 240$ cells, with the sides of the antenna 15 cells from the PML absorbing boundary (10 cells thick), except the bottom side (main direction for radiation), which was 30 cells from the absorbing boundary.  At the highest frequency (shortest wavelength) within the operational bandwidth of the antenna we have $\Delta x = 0.0093\lambda$, which corresponds roughly to 107 cells per wavelength.  From this result, we can estimate the numerical dispersion using Figure~\ref{fig:ACF2} or Equation~(\ref{eqn:ACE3}).  The relative error in the phase velocity is about 0.01\%, which is equivalent to $3.6\times 10^{-4}$ degrees of phase error per cell, or a total error of 0.25 degrees of phase error for propagation across the longest side of the computational volume.  As with the earlier case of the monopole antenna, it is not the error in the phase velocity that determines the accuracy of the solution but the degree to which the fine details of the structure are modeled.

Figure~\ref{fig:ACF19} is a comparison of the FDTD results (solid line) for this antenna with measurements (dashed line).  Figure~\ref{fig:ACF19}(a) shows the magnitude of the reflection coefficient at the terminals of the antenna, and Figure~\ref{fig:ACF19}(b) shows the realized gain (Equation~\ref{eqn:ACE14}) at boresight ($-\hat{z}$ direction) as a function of frequency.  The results from the FDTD calculations are in fairly good agreement with the measurements.  The differences that do exist are most likely caused by elements in the experimental model that were not included in the theoretical model.  In the experimental model, the metallic arms were on a very thin dielectric substrate (Kapton, thickness 0.051~mm), which was not included in the theoretical model.  In addition, the terminal measurements were made through a balun, and the imperfections in the balun were not taken into account.

\begin{figure}\begin{center}
\includegraphics[angle=0,scale=0.8]{/Users/jim.maloney/Book/images/AntChapOrigFig19.png}
\caption{Comparison of theoretical and measured results for the conical spiral antenna. (a) Magnitude of reflection coefficient versus frequency. (b) Realized gain in the boresight direction versus frequency. (After Hertel and Smith \cite{Hertel02}, \copyright~2002 IEEE.)}
\label{fig:ACF19}
\end{center}\end{figure}

The FDTD method provides detailed information about the electromagnetic field surrounding the spiral, and it can be used to graphically illustrate how energy is radiated from this structure \cite{Hertel03}.  Figure~\ref{fig:ACF20} shows three gray-scale plots of the magnitude of the $x$ component of the electric field on the $x$-$z$ plane.  Each plot is for a different normalized time $t/\tau_L$, where $\tau_L$ is the time for light to travel the length of the spiral arm.  We can see that the radiation is roughly periodic with the spacing between the nulls (white lines) being $\lambda/2$.  The frequency corresponding to this wavelength is indicated on each plot.  These plots clearly show that the region from which radiation leaves the antenna changes with the wavelength, moving from the small end (diameter $d$) for the shortest wavelengths (highest frequencies) to the large end (diameter $D$) for the longest wavelengths (lowest frequencies).  This is in keeping with the ``active-region concept,'' which states that the radiation originates at the cross section of the spiral that is approximately one wavelength in circumference \cite{Dyson65}.

\begin{figure}\begin{center}
\includegraphics[angle=0,width=\linewidth]{/Users/jim.maloney/Book/images/AntChapOrigFig20.png}
\caption{Gray-scale plots for the magnitude of the electric field near the conical spiral antenna for three instants in time: (a) $t/\tau_L=0.1$, (b) $t/\tau_L=0.6$, and (c) $t/\tau_L=1.1$, where $\tau_L$ is the time for light to travel the length of the spiral arm. (After Hertel and Smith \cite{Hertel03}, \copyright~2003 IEEE.)}
\label{fig:ACF20}
\end{center}\end{figure}

In the previous two examples, the stair-stepped approximations used for the geometry of the antennas in the FDTD models were adequate for obtaining theoretical results in good agreement with the measurements.  This is a consequence of choosing the size of the steps to be small compared to the dimensions defining the geometry of the antennas.  For example, for the pyramidal horn, the height of the stair step is only about 10\% of the smallest dimension of the antenna (the height of the rectangular waveguide).  We will now consider a case in which the stair-stepped approximation leads to significant errors in the calculated results.

The transverse electromagnetic (TEM) horn is a simple antenna used for applications that require broad bandwidth.  The FDTD model for the monopole version of this antenna is shown in Figure~\ref{fig:ACF21}(a).  It is formed from a PEC plate that is an isosceles triangle of side length $s$ and angle at the apex $\alpha$.  The plate is inclined at the angle $\beta/2$ to the PEC image plane, and the antenna is fed by a transmission line connected between the apex of the plate and the image plane.  The plate/image plane forms a TEM transmission line, and for the example to be discussed ($\alpha=25.4^\circ$, $\beta=11.2^\circ$), the characteristic impedance of this line is $R_o \approx 50~\Omega$ \cite{Shlager96}--\cite{LeeHornTEM}.  The transmission line feeding the antenna has the same characteristic impedance.

The plate for this antenna is stair stepped in the FDTD model in the manner shown in Figure~\ref{fig:ACF21}(b).  Two different sizes for the staircase will be examined: Case~A for which the rise is $\Delta z=1$~mm and the tread length is $\Delta s=1$~cm, and Case~B for which $\Delta z=2$~mm and $\Delta s = 2$~cm.  Notice that the level of discretization for Case~B is twice as coarse as that for Case~A.  The smallest dimensions for the horn are at the drive point, where the initial tread for both cases is 4~mm above the image plane.  So for Case~A, the rise of the staircase, $\Delta z$, is about 25\% of the smallest dimension of the horn; whereas, for Case~B it is about 50\% of the smallest dimension.

\begin{figure}\begin{center}
\includegraphics[angle=0,width=\linewidth]{/Users/jim.maloney/Book/images/AntChapOrigFig21.png}
\caption{(a) Schematic drawing for the TEM horn antenna (monopole configuration). (b) Cross sections showing the stair-stepped approximation to the plate for two different cases, A and B.}
\label{fig:ACF21}
\end{center}\end{figure}

Figure~\ref{fig:ACF22}(a) shows the reflected voltage, $V^\text{-}_t(t)$, in the feeding transmission line of the horn when the incident voltage, $V^\text{+}_t(t)$, is a unit-amplitude, differentiated Gaussian pulse (Equation~\ref{eqn:ACE20}) with the characteristic time $\tau_p=5.31\times 10^{-11}$~s.  The peak of the spectrum for the pulse is at 3.0~GHz.  The solid line is for Case~A and the dashed line is for Case~B.  The initial reflection from the drive point is evident and is similar for both cases, and the reflection from the open end of the horn has been windowed out.  There is a pronounced ripple in the result for the coarser staircase, Case~B.  The ripple is clearly due to the staircase, because its period roughly corresponds to the round-trip time on a tread, which is $\Delta t = 2\Delta s_B/c \approx 2.5\tau_p$.  Notice that the amplitude of the ripple decreases with time.  This is because the reflections that occur later in time are from stair steps further out along the antenna, where the rise of the staircase, $\Delta z$, is a smaller fraction of the separation between the plate and the image plane.

\begin{figure}\begin{center}
\includegraphics[angle=0,scale=0.7]{/Users/jim.maloney/Book/images/AntChapOrigFig22.png}
\caption{Results for two different stair-stepped approximations (A and B) applied to the TEM horn antenna. (a) The reflected voltage in the feeding transmission line; the reflection from the open end of the horn has been windowed out. (b) The magnitude of the Fourier transform of the reflection coefficient for the antenna.}
\label{fig:ACF22}
\end{center}\end{figure}

Figure~\ref{fig:ACF22}(b) shows the magnitude of the Fourier transform (spectrum) of the reflection coefficient for the antenna.  Notice that the results for the two cases, A and B, are quite different.  Specifically, for Case~B there is a distinct dip in the reflection coefficient near $2f\Delta s_B/c = 1$ ($f=7.5$~GHz).  At this frequency, $\Delta s_B/\lambda = 1/2$, so the small reflections from all of the steps in the staircase add in phase.

To avoid the problem described above, we must use a finer staircase, such as in Case~A.  For TEM horns with low characteristic impedance (generally small $\beta$), this can require a very fine level of discretization.  A similar problem is encountered with bow-tie antennas with low characteristic impedance \cite{ShlagerBowtie94}.

\subsection{Microstrip Patches: Excessive Ringing for Narrow-Band Antennas}

The antennas we examined in the previous section, a conical spiral and horns, are fairly wideband antennas.  Now we consider the other extreme, namely, narrowband antennas.  For our example, we use the basic, rectangular microstrip patch antenna shown in Figure~\ref{fig:ACF23}.

\begin{figure}\begin{center}
\includegraphics[angle=0,scale=0.7]{/Users/jim.maloney/Book/images/AntChapOrigFig23.png}
\caption{Rectangular microstrip patch antenna fed by a coaxial line probe.}
\label{fig:ACF23}
\end{center}\end{figure}

In the mid-1980s, Chang et al.\ made extensive measurements of this antenna, and first we compare our FDTD results with their measurements \cite{Change}.  The dimensions for a patch designed for frequencies around $f = 7.0$~GHz are $s = 1.1$~cm, $w = 1.7$~cm, and $h = 3.175$~mm.  As shown in the figure, the probe of the feeding coaxial line ($R_o=50~\Omega$) is displaced from the broad side of the patch by $l_p = 1.5$~mm.  In the model, the dielectric substrate is $10~\text{cm} \times 10$~cm with the electrical properties $\epsilon_r=2.33$ and $\sigma=2.1\times 10^{-3}$~S/m, and the ground plane is infinite.  The incident voltage, $V^\text{+}_t(t)$, in the feeding transmission line is a unit-amplitude, differentiated Gaussian pulse (Equation~\ref{eqn:ACE20}) with the characteristic time $\tau_p=2.65\times 10^{-11}$~s.  The peak of the spectrum for this pulse is at 6.0~GHz.

The dimensions of the FDTD rectangular cells ($\Delta x=0.529$~mm, $\Delta y=0.500$~mm, $\Delta z=0.500$~mm) were chosen so that all of the details of the coaxial feed line could be included in the model, and the time step was $\Delta t=9.44\times 10^{-13}$~s.  The number of time steps, $N_t$, required for the simulation was determined by observing the magnitude of the reflected voltage $|V^\text{-}_t|$ in the feeding transmission line versus the normalized time $t/\Delta t$; this is shown in Figure~\ref{fig:ACF24}(a).  Notice that the vertical scale is logarithmic.  When $t/\Delta t=3000$, the reflected voltage has dropped by six orders of magnitude from its peak, and it is at the noise level for the computation.  So any number of time steps greater than three thousand was deemed adequate for the simulation ($N_t=4000$ was actually used).

\begin{figure}\begin{center}
\includegraphics[angle=0,scale=0.7]{/Users/jim.maloney/Book/images/AntChapOrigFig24.png}
\caption{The magnitude of the reflected voltage in the feeding coaxial line versus the normalized time. (a) Rectangular microstrip patch. (b) Narrow-band, rectangular microstrip patch.}
\label{fig:ACF24}
\end{center}\end{figure}

Figure~\ref{fig:ACF25} is a comparison of the FDTD theoretical results with the measurements.  The graph in Figure~\ref{fig:ACF25}(a) shows the magnitude of the reflection coefficient versus frequency: theory (solid line) and measurement (dots).  The agreement is reasonably good, particularly when we consider that some of the geometrical detail for the measurement, such as the precise geometry at the feed, were not known for use in the FDTD model.

\begin{figure}\begin{center}
\includegraphics[angle=0,scale=0.7]{/Users/jim.maloney/Book/images/AntChapOrigFig25.png}
\caption{Comparison of theoretical and measured results for the rectangular microstrip patch antenna. (a) Magnitude of reflection coefficient versus frequency. (b) Field patterns for E and H planes at the frequency $f = 6.8$~GHz. Measured results from \cite{Change}.}
\label{fig:ACF25}
\end{center}\end{figure}

The field patterns were measured with the 10~cm $\times$ 10~cm substrate mounted at the center of a circular aluminum image plane of diameter 1~m.  We chose not to model this configuration with the same fine resolution used for the FDTD calculation of the reflection coefficient, because of the large amount of memory that would be required.  Instead, larger cells were used with the dimensions $\Delta x=1.59$~mm, $\Delta y=1.42$~mm, $\Delta z=1.57$~mm.  The use of the larger cells causes little error in the far-zone field patterns.  The FDTD and measured field patterns for the frequency $f = 6.8$~GHz are compared in Figure~\ref{fig:ACF25}(b).  These plots show the gain (Equation~\ref{eqn:ACE15}) versus angle, normalized to 0~dB at the peak.  Results are given for both the E~plane ($x$-$z$ plane, solid line and dots) and the H~plane ($y$-$z$ plane, dashed line and triangles).  Again the agreement is reasonably good.

For our second example, we chose a rectangular microstrip patch antenna designed to operate around $f = 1.9$~GHz that is similar to one reported in the literature \cite{Abdallah}.  The dimensions for the patch are $s = 5.12$~cm, $w = 6.0$~cm, and $h = 1.575$~mm, and the probe of the feeding coaxial line ($R_o=50~\Omega$) is displaced from the broad side of the patch by $l_p = 1.64$~cm.  The dielectric substrate ($\epsilon_r=2.2$ and $\sigma=1.1\times 10^{-3}$~S/m) and the ground plane are the same size: 11.5~cm $\times$ 11.5~cm.  The incident voltage, $V^\text{+}_t(t)$, in the feeding transmission line is a unit-amplitude, differentiated Gaussian pulse (Equation~\ref{eqn:ACE20}) with the characteristic time $\tau_p=1.061\times 10^{-10}$~s, and the peak of the spectrum for this pulse is at 1.5~GHz.  Again, the parameters for the FDTD simulation allow complete modeling of the details of the coaxial feed line ($\Delta x=0.529$~mm, $\Delta y=0.500$~mm, $\Delta z=0.500$~mm, $\Delta t=9.91\times 10^{-14}$~s).  The electrical thickness of the substrate for this example is about one eighth of that for the previous example, $h/\lambda = 0.010$ (for $f = 1.9$~GHz) versus $h/\lambda=0.077$ (for $f = 7.3$~GHz), so we expect this antenna to have a significantly narrower bandwidth \cite{Jackson}.

Figure~\ref{fig:ACF24}(b) shows the magnitude of the reflected voltage $|V^\text{-}_t|$ in the feeding transmission line (logarithmic scale) versus the normalized time $t/\Delta t$.  As a consequence of the narrower bandwidth, the reflected voltage decreases much more slowly with increasing $t/\Delta t$ than in the previous example, Figure~\ref{fig:ACF24}(a).  The reflected voltage has dropped by six orders of magnitude from its peak and is approaching the noise level for the computation when $t/\Delta t=50{,}000$.  So about fifty thousand time steps ($N_t=50{,}000$) are required for the simulation, as compared to three thousand for the previous example!  The inset in Figure~\ref{fig:ACF24}(b) shows the magnitude of the reflected voltage, plotted on a linear scale, for times around $t/\Delta t=40{,}000$.  The voltage is seen to be a slowly decaying sinusoid at the frequency $f\approx 1.89$~GHz.

In Figure~\ref{fig:ACF26} we show the magnitude of the reflection coefficient versus frequency for simulations with different numbers of time steps: $N_t=8{,}000$, $N_t=24{,}000$, and $N_t=50{,}000$.  For each case, a Hanning window is applied in time to eliminate truncation artifacts.  The antenna is seen to be matched at the frequency $f=1.89$~GHz, and the ``apparent'' bandwidth for the match is seen to depend on the number of time steps used for the simulation.  Thus, if one were to underestimate the number of time steps required for the simulation to converge, one would think that the antenna had a much wider bandwidth for the reflection coefficient than it actually has.  With the detailed analysis presented above, this point may appear to be obvious.  However, sometimes, particularly when a computation is automated, this degree of analysis may not be performed every time a parameter for the antenna, such as the thickness of the substrate, is changed.

In some cases, special techniques can be applied to shorten the computation for a narrow-bandwidth antenna.  For example, because of the well-defined, decaying sinusoidal waveform in the reflection coefficient for this antenna, a shorter computation time, say $N_t=20{,}000$, could be used with an extrapolation for the remainder of the waveform.  Such techniques are discussed in the literature \cite{Chebolu}.

\begin{figure}\begin{center}
\includegraphics[angle=0,scale=1.0]{/Users/jim.maloney/Book/images/AntChapOrigFig26.png}
\caption{Narrow-band, rectangular microstrip patch antenna. Magnitude of reflection coefficient versus frequency for three different numbers of time steps.}
\label{fig:ACF26}
\end{center}\end{figure}

\section{Summary and Conclusions}

In this appendix, we have presented an introduction to the finite-difference time-domain method, aimed at readers who have little or no experience with the method.  We have limited the presentation to the basics of the method, avoiding mention of many refinements that are generally restricted to particular applications.  To give the reader a sense of the breadth of application allowed by these refinements, we present a partial list below.
\begin{itemize}
  \item{Techniques for handling materials with dispersive properties (properties that are a function of the frequency), anisotropic properties (properties that depend on the direction of the field components), and nonlinear properties.}
  \item{Methods for incorporating impedance boundary conditions.}
  \item{Subcell methods for treating material sheets that are thinner than an FDTD cell.}
  \item{Methods for incorporating periodic boundary conditions, which are useful in treating antenna arrays.}
  \item{Higher-order FDTD schemes that have lower error (numerical dispersion) than the conventional Yee algorithm.}
  \item{Techniques for incorporating nonuniform and nonorthogonal grids.}
  \item{Special procedures for handling objects that are bodies of revolution.}
\end{itemize}

The brevity of this appendix precluded the derivation of the mathematical formulas associated with the method, e.g., FDTD update equations, equations for the perfectly matched layer, etc.  These formulas can be found in the comprehensive treatment of the method contained in the book edited by Taflove and Hagness \cite{Taf2005}.  A comprehensive web site on the method is maintained by J.~B.~Schneider at Washington State University: \texttt{www.fdtd.org}.  This site contains searchable lists of books, journal papers, conference papers, and dissertations.

To assess the popularity of the FDTD method, a search was done with INSPEC for documents that included either ``finite-difference time-domain'' or ``FDTD'' in the title.\footnote{A few of these documents apply the finite-difference time-domain method to problems other than electromagnetic, such as acoustic problems.}  The results of the search, presented in Figure~\ref{fig:ACF27}, clearly show the rapid growth in the popularity of the method over the last twenty-five years.

\begin{figure}\begin{center}
\includegraphics[angle=0,scale=1.0]{/Users/jim.maloney/Book/images/AntChapOrigFig27.png}
\caption{Number of documents published over a twenty-year span that include the words ``finite-difference time-domain'' or ``FDTD'' in the title. Each bar shows the total number of documents published during a five-year period.}
\label{fig:ACF27}
\end{center}\end{figure}

The emphasis throughout this appendix has been on the application of the FDTD method to the analysis of antennas.  After brief discussions of the special formulations associated with transmitting and receiving antennas, the details for the analysis of a few different types of antennas were presented.  Because of the brevity of this appendix, no attempt was made to mention all of the different antennas that have been analyzed with the method.  Many individuals have used the method to treat antennas; as an indication of the number, the INSPEC search mentioned above listed over 500 documents with FDTD and antenna(s) in the title.

All of the numerical results presented in the examples were obtained by the authors or their students.  Thus, we have very detailed knowledge for each example and can make fairly accurate statements about the results.  These examples were chosen not only to show the power of the FDTD method, particularly the good agreement with experimental measurements, but also to show that the method has some limitations---albeit, the limitations are sometimes due to the crudeness of the theoretical model for the antenna or the choice of the parameters for the simulation.  The refinement of the FDTD method and its application to practical problems is an ongoing story, and undoubtedly there will be some exciting accomplishments made in the future.

\emph{One area with great promise is the use of the method for antenna synthesis.}  Here we do not mean the conventional approach in which the method is coupled with an optimization routine and used to choose the parameters for a standard antenna (dipole, horn, etc.) so that certain criteria for the performance are met.  What we have in mind for antenna synthesis is quite different---a more modern approach.  In this approach, the structure of the antenna is not completely predetermined with only a few parameters to be chosen, but the structure of the antenna is actually developed as part of the synthesis.  The FDTD method is well suited for use in such schemes; because of the flexibility of the method, a new structure can easily be introduced.  The antenna structure is changed by simply changing the electromagnetic constitutive parameters associated with individual cells.  \emph{The earlier chapters in this book describe the ``fragmented aperture'' concept, which relied heavily on the FDTD method as described in this appendix} \cite{MaloneyURSI99}--\cite{RECAP}.

\begin{thebibliography}{99}

 \bibitem{Yee66} K. S. Yee, ``Numerical Solution of Initial Boundary Value Problems Involving Maxwell's Equations in Isotropic Media,'' IEEE Trans. Antennas Propagat., Vol. AP-14, pp. 302--307, May 1966.

\bibitem{Maloney1} J. G. Maloney, G. S. Smith, and W. R. Scott, Jr., ``Accurate Computation of the Radiation from Simple Antennas Using the Finite-Difference Time-Domain Method,'' IEEE Trans. Antennas Propagat., Vol. AP-38, pp. 1059--1068, July 1990.

\bibitem{BoonPist} J. J. Boonzaaier and C. W. Pistorius, ``Thin Wire Dipoles: A Finite-Difference Time-Domain Approach,'' Electronics Lett., Vol. 26, pp. 1891--1892, October 1990.

\bibitem{KatzHorn} D. S. Katz, M. J. Picket-May, A. Taflove, and K. R. Umashankar, ``FDTD Analysis of Electromagnetic Wave Radiation from Systems Containing Horn Antennas,'' IEEE Trans. Antennas Propagat., Vol. AP-39, pp. 1203--1212, August 1991.

\bibitem{TirkusRad} P. A. Tirkus and C. A. Balanis, ``Finite-Difference Time-Domain Method for Antenna Radiation,'' IEEE Trans. Antennas Propagat., Vol. AP-40, pp. 334--340, March 1992.

\bibitem{LuebbersGain} R. J. Luebbers and J. Beggs, ``FDTD Calculation of Wide-Band Antenna Gain and Efficiency,'' IEEE Trans. Antennas Propagat., Vol. AP-40, pp. 1403--1407, November 1992.

\bibitem{MaloneySmithAntChapters} J. G. Maloney and G. S. Smith, ``Modeling of Antennas,'' Chapter 7 in A. Taflove, Editor, Advances in Computational Electrodynamics, The Finite-Difference Time-Domain Method, pp. 409--460, Artech House, Boston, 1998. Also, J. G. Maloney, G. S. Smith, E. Thiele, O. Gandhi, N. Chavannes, and S. Hagness, Chapter 14 in A. Taflove and S. Hagness, Editors, Computational Electrodynamics: The Finite-Difference Time-Domain Method, 3rd Edition, pp. 607--676, Artech House, Boston, 2005.

\bibitem{SmithBook} G. S. Smith, An Introduction to Classical Electromagnetic Radiation, Cambridge University Press, Cambridge, UK, 1997.

\bibitem{Taf2005} A. Taflove and S. C. Hagness, Editors, Computational Electrodynamics: The Finite-Difference Time-Domain Method, 3rd Edition, Artech House, Boston, 2005.

\bibitem{Schneider99} J. B. Schneider and C. L. Wagner, ``FDTD Dispersion Revisited: Faster-Than-Light Propagation,'' IEEE Microwave and Guided Wave Lett., Vol. 9, pp. 54--56, February 1999.

\bibitem{Gedney96} S. Gedney, ``An Anisotropic Perfectly Matched Layer-Absorbing Medium for the Truncation of FDTD Lattices,'' IEEE Trans. Antennas Propagat., Vol. AP-44, pp. 1630--1639, December 1996.

\bibitem{Gedney2005} S. Gedney, ``Perfectly Matched Layer Absorbing Boundary Conditions,'' Chapter 7 in A. Taflove and S. C. Hagness, Editors, Computational Electrodynamics: The Finite-Difference Time-Domain Method, 3rd Edition, pp. 273--328, Artech House, Boston, 2005.

\bibitem{Shlager94} K. L. Shlager and G. S. Smith, ``Near-Field to Near-Field Transformation for Use With FDTD Method and Its Application to Pulsed Antenna Problems,'' Electronics Lett., Vol. 30, pp. 1262--1264, August 1994.

\bibitem{Shlager95} K. L. Shlager and G. S. Smith, ``Comparison of Two Near-Field to Near-Field Transformations Applied to Pulsed Antenna Problems,'' Electronics Lett., Vol. 31, pp. 936--938, June 1995.

\bibitem{Smith2004} G. S. Smith, ``A Direct Derivation of a Single-Antenna Reciprocity Relation for the Time Domain,'' IEEE Trans. Antennas Propagat., Vol. AP-52, pp. 1568--1577, June 2004.

\bibitem{Maloney97} J. G. Maloney, M. P. Kesler, and G. S. Smith, ``Generalization of PML to Cylindrical Geometries,'' 13th Annual Review of Progress in Applied Computational Electromagnetics, Monterey, CA, pp. 900--908, March 1997.

\bibitem{Smith2001} G. S. Smith and T. W. Hertel, ``On the Transient Radiation of Energy from Simple Current Distributions and Linear Antennas,'' IEEE Antennas Propagat. Magazine, Vol. 43, pp. 49--62, June 2001.

\bibitem{Smith2003} T. W. Hertel and G. S. Smith, ``On the Convergence of Common FDTD Feed Models for Antennas,'' IEEE Trans. Antennas Propagat., Vol. AP-51, pp. 1771--1779, August 2003.

\bibitem{King56} R. W. P. King, The Theory of Linear Antennas, p. 20, Harvard Univ. Press, Cambridge, MA, 1956.

\bibitem{Maloney94} J. G. Maloney, K. L. Shlager, and G. S. Smith, ``A Simple FDTD Model for Transient Excitation of Antennas by Transmission Lines,'' IEEE Trans. Antennas Propagat., Vol. AP-42, pp. 289--292, February 1994.

\bibitem{DeyMittra} S. Dey and R. Mittra, ``A Locally Conformal Finite-Difference Time-Domain (FDTD) Algorithm for Modeling Three-Dimensional Perfectly Conducting Objects,'' IEEE Microwave and Guided Wave Lett., Vol. 7, pp. 273--275, September 1997.

\bibitem{Taf05} A. Taflove, M. Celuch-Marcysiak, and S. Hagness, ``Local Subcell Models of Fine Geometrical Features,'' Chapter 10 in A. Taflove and S. C. Hagness, Editors, Computational Electrodynamics: The Finite-Difference Time-Domain Method, 3rd Edition, pp. 407--462, Artech House, Boston, 2005.

\bibitem{Gedney05} S. Gedney, F. Lansing, and N. Chavannes, ``Nonuniform Grids, Nonorthogonal Grids, Unstructured Grids, and Subgrids,'' Chapter 11 in A. Taflove and S. C. Hagness, Editors, Computational Electrodynamics: The Finite-Difference Time-Domain Method, 3rd Edition, pp. 463--516, Artech House, Boston, 2005.

\bibitem{Cangellaris} A. C. Cangellaris and D. B. Wright, ``Analysis of the Numerical Error Caused by the Stair-Stepped Approximation of a Conducting Boundary in FDTD Simulations of Electromagnetic Phenomena,'' IEEE Trans. Antennas Propagat., Vol. AP-39, pp. 1518--1525, October 1991.

\bibitem{Holland} R. Holland, ``Pitfalls of Staircase Meshing,'' IEEE Trans. Electromagnetic Compatibility, Vol. 35, pp. 434--439, November 1993.

\bibitem{Hertel02} T. W. Hertel and G. S. Smith, ``Analysis and Design of Two-Arm Conical Spiral Antennas,'' IEEE Trans. Electromagnetic Compatibility, Vol. 44, pp. 25--37, February 2002.

\bibitem{Hertel03} T. W. Hertel and G. S. Smith, ``On the Dispersive Properties of the Conical Spiral Antenna and Its Use for Pulsed Radiation,'' IEEE Trans. Antennas Propagat., Vol. AP-51, pp. 1426--1433, July 2003.

\bibitem{Dyson65} J. D. Dyson, ``The Characteristics and Design of the Conical Log-Spiral Antenna,'' IEEE Trans. Antennas Propagat., Vol. AP-13, pp. 488--499, July 1965.

\bibitem{Shlager96} K. L. Shlager, G. S. Smith, and J. G. Maloney, ``Accurate Analysis of TEM Horn Antennas for Pulse Radiation,'' IEEE Trans. Electromagnetic Compatibility, Vol. 38, pp. 414--423, August 1996.

\bibitem{LeeImpedTEM} R. T. Lee and G. S. Smith, ``On the Characteristic Impedance of the TEM Horn Antenna,'' IEEE Trans. Antennas Propagat., Vol. AP-52, pp. 315--318, January 2004.

\bibitem{LeeHornTEM} R. T. Lee and G. S. Smith, ``A Design Study for the Basic TEM Horn Antenna,'' IEEE Antennas Propagat. Magazine, Vol. 46, pp. 86--92, February 2004.

\bibitem{ShlagerBowtie94} K. L. Shlager, G. S. Smith, and J. G. Maloney, ``Optimization of Bow-Tie Antennas for Pulse Radiation,'' IEEE Trans. Antennas Propagat., Vol. AP-42, pp. 975--982, July 1994.

\bibitem{Change} E. Chang, S. A. Long, and W. F. Richards, ``An Experimental Investigation of Electrically Thick Rectangular Microstrip Antennas,'' IEEE Trans. Antennas Propagat., Vol. AP-34, pp. 767--772, June 1986.

\bibitem{Abdallah} H. Abdallah, W. Wasylkiwskyj, K. Parikh, and A. Zaghloul, ``Comparison of Return Loss Calculations with Measurements of Narrow-Band Microstrip Patch Antennas,'' ACES Journal, Vol. 19, pp. 184--186, November 2004.

\bibitem{Jackson} D. R. Jackson and N. G. Alexopoulos, ``Simple Approximate Formulas for the Input Resistance, Bandwidth, and Efficiency of a Resonant Rectangular Patch,'' IEEE Trans. Antennas Propagat., Vol. AP-39, pp. 407--410, March 1991.

\bibitem{Chebolu} S. Chebolu, R. Mittra, and W. D. Becker, ``The Analysis of Microwave Antennas Using the FDTD Method,'' Microwave Journal, Vol. 39, pp. 134--150, January 1996.

\bibitem{MaloneyURSI99} J. G. Maloney, P. H. Harms, M. P. Kesler, T. L. Fountain, and G. S. Smith, ``Novel, Planar Antennas Designed Using the Genetic Algorithm,'' 1999 USNC/URSI Radio Science Meeting, Orlando, FL, p. 237, July 1999.

\bibitem{MaloneyAP2000} J. G. Maloney, M. P. Kesler, P. H. Harms, T. L. Fountain, and G. S. Smith, ``The Fragmented Aperture Antenna: FDTD Analysis and Measurement,'' Millennium Conference on Antennas and Propagation (AP 2000), Davos, Switzerland, 4 pages, April 2000.

\bibitem{MaloneyFragPatent} J. G. Maloney, M. P. Kesler, P. H. Harms, and G. S. Smith, Fragmented Aperture Antennas and Broadband Ground Planes, U.S. Patent No. 6,323,809 B1, November 27, 2001.

\bibitem{RECAP} L. N. Pringle, P. H. Harms, S. P. Blalock, G. N. Kiesel, E. J. Kuster, P. G. Friederich, R. J. Prado, J. M. Morris, and G. S. Smith, ``A Reconfigurable Aperture Antenna Based on Switched Links Between Electrically Small Metallic Patches,'' IEEE Trans. Antennas Propagat., Vol. AP-52, pp. 1434--1445, June 2004.

\bibitem{BalanisHB} \textcolor{red}{\textbf{[Incomplete reference: Balanis Antenna Engineering Handbook chapter --- need full citation with chapter number, page range, edition, and year.]}}

\end{thebibliography}
			% For Zander


%\chapter{Computational Modeling of Array Antennas}

\section{Introduction}
Many structures of electromagnetic interest posses a periodicity in one or more dimensions.  For example, a frequency selective surface (FSS) is commonly used in a radome to control the energy that reaches an antenna.  A typical FSS consists of one or more layers of material, each of which is formed from an element periodically replicated in two dimensions.  Another periodic structure that has received considerable attention n recent years is the photonic bandgap (PBG) or electronic bandgap (EBG) structure.  A PBG is a type of periodic dielectric structure that has frequency regions in which electromagnetic propagation is forbidden.  A third type of structure that can be considered periodic is an antenna array.  If large enough, many of the important paramaters of an array can be analyzed by assuming that the structure is infinitely periodic.  

\section{Analysis of Periodic Structures}

The FDTD technique, introduced in Appendix A, can be applied to the analysis of periodic structures such as the ones mentioned above.  Such structures often have fine details at the element level that must be accurately modeled in order to predict the correct electromagnetic behavior.  Couple with the fact that the overall structure may consist of many replicas of the basic element, this level of detail leads to computational problems that are unmanagable.  One way to alleviate the computational burden is to model only the individual element and use boundary conditions to simulate the effect of periodic replication.  As an example of a periodic structure, consider the two-dimensional, electromagnetic screen shown in Figure \ref{fig:CMAAF1}.  The screen consists of infinitely long, periodic conducting bars of width $w$; thickness $d$; separated by a gap $g$.  The illumination is an electromagnetic plane wave with the electric field parallel to the bars (z polarized) and the direction of propagation is at angle $\phi_i$ with respect to the $x$ axis as shown in the figure. This is a two-dimensional electromagnetic problem; only the TE fields ($E_z$, $H_x$, and $H_y$) are nonzero.  Because of the periodicty inherent in the geometry. only fields in the ``unit cell'' are unique and need to be determined.  Dashed lines in the figure denote the edges of the unit cell.
\begin{figure} \begin{center}
\includegraphics[angle=0,width=\linewidth]{/Users/jim/Book/images/IAM1.pdf}
\caption{Two-dimensional electromagnetic screen (left) and a top view of the structure showing the field components at the edges of the unit cell. [Used with permission of Artech House]}
\label{fig:CAMMF1}
\end{center}\end{figure}

Now, consider discretizing the solution space within the unit cell using a lattice of traditional ``Yee'' cells \cite{Yee66}.  Field components that can be updated using the fields in the unit call are denoted by solid black symbols, while those that cannot be updated are denoted by white symbols.  The field components that cannot be updated can instead be determined using periodic boundary conditions (PBCs) that relate the fields on one side of the unit cell to those on the other side.  Of course, this relationship depends on the angle at which the incident field impinges on the structure.  For the geometry shown. the relationship is expressed in the frequency domain by
\begin{align}
\begin{split}
H_x(x,y_p+\Delta y/2, \omega) = & H_x(x,y_p+\Delta y/2,\omega) \exp(-j K_y y_p), \\
E_z(x,0,\omega) = & E_x(x,y_p,\omega) \exp(+j K_y y_p),
\label{eqn:CMAAE1}
\end{split}
\end{align}

\noindent where $k_y = k_o \sin(\phi_i)$ and $k_o=\omega/c$ is the free-space wavenumber and $c$ is the free-space speed of light.  FDTD requires the boundary conditions to be expressed in the time domain.  In the time domain, (\ref{eqn:CMAAE1}) becomes
\begin{align}
\begin{split}
H_x(x,y_p+\Delta y/2,t) = & H_x(x,y_p+\Delta y/2,t-y_p\sin\phi_t), \\
E_z(x,0,t) = & E_x(x,y_p,t+y_p\sin\phi_t).
\label{eqn:CMAAE2}
\end{split}
\end{align}

\noindent Equation \ref{eqn:CMAAE2} shows that the periodic boundary condition involves using both previous and well as future field components.  Specifically, the $H_x$ values at $y=y_p+\Delta y/2$ are equal to previous values of $H_x$ at $y=\Delta y/2$, while the $E_z$ values at $y=0$ are equal to future values of $E_z$ at $y=y_p$.  Previous values can simply be saved using additional computational storage.  However, there is no simple solution for this need for future values of $E_z$.



\begin{figure} \begin{center}
\includegraphics[angle=0,width=\linewidth]{/Users/jim/Book/images/IAM2.pdf}
\caption{A summary of techniques used to perform FDTD with periodic boundary conditions where the techniques are divided into two classes.  A complete discussion of all the techniques can be found in \cite{maloneyKeslerPBC} [Used with permission of Artech House]}
\label{fig:CAMMF2}
\end{center}\end{figure}

\section{Review of Scattering from Periodic Structure}



\begin{thebibliography}{99}

\bibitem{Yee66} K. S. Yee, ``Numerical Solution of Initial Boundary Value Problems Involving Maxwell's Equations in Isotropic Media,'' IEEE Trans. Antennas Propagat., Vol. AP-14, pp. 302-307, May 1966.

\bibitem{MaloneyKeslerPBC} J. G. Maloney and M. P. Kesler, ``Analysis of Periodic Structures,'' Chapter 6 in A. Taflove, Editor, Advances in Computational Electrodynamics, The Finite-Difference Time-Domain Method, pp. 345-407, Artech House, Boston, 1998. Also, J. G. Maloney and M. P. Kesler, ``Analysis of Periodic Structures,''  Chapter XX in A. Taflove, and S. Hagness, Editors, Computational Electrodynamics: The Finite-Difference Time-Domain Method, 3rd Edition, pp. YYY-ZZZ, Artech House, Boston, 2005.



\bibitem{Taf2005} A. Taflove and S. C. Hagness, Editors, Computational Electrodynamics: The Finite-Difference Time-Domain Method, Artech House, Boston, 2005.

\bibitem{Elsherbeni2009}  A Elsherbeni and V. Demir, The Finite-Difference Time-Domain Method for Electromagnetics with MATLAB Simulations, Scitech, Rayleigh, NC, 2009.

\bibitem{Gedney2011} S. D. Gedney, Introduction to the Finite-Difference Time-Domain Method for Electromagnetics, Morgan and Claypool, www.morganclaypool.com, 2011.

\bibitem{MaloneySmithAntChapters} J. G. Maloney and G. S. Smith, ``Modeling of Antennas,'' Chapter 7 in A. Taflove, Editor, Advances in Computational Electrodynamics, The Finite-Difference Time-Domain Method, pp. 409-460, Artech House, Boston, 1998. Also, J. G. Maloney, G. S. Smith, E. Thiele, O. Ghandi, N. Chavannes, and S. Hagness, Chapter 14 in A. Taflove, and S. Hagness, Editors, Computational Electrodynamics: The Finite-Difference Time-Domain Method, 3rd Edition, pp. 607-676, Artech House, Boston, 2005.

\bibitem{BalanisHB} G. S. Smith and J. G. Maloney, "Finite-Difference Time-Domain Method Applied to Antennas," Chapter 30 



\end{thebibliography}


			% Appendix B
%\chapter{Focused Beam Measurement of Antenna Gain}
%\authortoc{James G. Maloney}
%\chapterauthor{James G. Maloney}

\section{Introduction}

Over the last few years, the authors have been involved with developing the use of the focused beam measurement system to measure antenna properties such as gain and pattern [10].  A series of improved, fragmented aperture antenna designs will be fabricated and measured with the Compass Tech Focused Beam System and compared with the design predictions to validate the designs.  This comparison will be included in the poster presentation.

\section{Standard Focused Beam System Configurations}

Measuring antenna properties, such as gain and antenna pattern, traditionally require large anechoic chambers, expensive compact ranges or near-field scanning facilities.  The standard, two-foot (61 mm), focused beam system, shown in Figure~\ref{fig:EVF16}, has proven to be a reliable apparatus for measuring dielectric/magnetic materials, meta-materials, and FSS over the last several decades [11]. 

\begin{figure}
\begin{center}
\includegraphics[angle=0,width=\linewidth]{/Users/jim/Book/images/improvedAMTAFig16.png}
\caption{Standard, 2-foot (61 mm) Focused beam system.}
\label{fig:EVF16}		% figure references have to be below the \caption line
\end{center}
\end{figure}

The antenna under test (AUT) can be simply mounted in the center of the sample holder using a 2 foot x 2 foot foam sheet and the turntable enables the measurement of antenna pattern cuts as shown in Figure~\ref{fig:EVF16}.

The lenses in the standard focused beam system can be reconfigured to achieve different degrees of focusing. The three possibilities are termed ?standard beam?, ?expanded beam?, and ?collimated beam.?  For a complete description of the shape the standard lenses, see \cite{ReidThesis}(page 89). 

In Figure~\ref{fig:EVF17}, we compare the maximum antenna size that can be measured with each lens configuration.  In addition, we also look at the no-lens case to illustrate where the use of the lens provides increased capability over a simple two antenna method.  For all practical purposes Figure ~\ref{fig:EVF17} shows that only the collimated beam option allows measuring a larger antenna than simply having no lenses.  

In effect, the collimated beam configuration removes the quadratic phase across the test volume in a similar manner to how the reflector in large compact ranges mitigates the quadratic phase.

\section{Experimental Measurements}

\begin{figure}
\includegraphics[angle=0,width=\linewidth]{/Users/jim/Book/images/improvedAMTAFig17.png}
\caption{Maximum Antenna Size for 4 lens configurations}
\label{fig:EVF17}		% figure references have to be below the \caption line
\end{figure}

Figure~\ref{fig:EVF17} also shows that if we build experimental fragmented aperture prototypes that are no larger than 10? (~0.25 m) for use at frequencies below 10 GHz we should be able to safely use the collimated beam to measure the gain and pattern.



\section{Gain Measurements of Compass Technology SP2-18 Probe Antenna}


\section{Gain Measurements of Fragmented Aperture Antennas}

\textcolor{red}{We will include, measurements of  series of experimental fragmented prototypes.} 

\textcolor{red}{We will compare the gain measured using no-lens (i.e. two-antenna far-field method) with the gain measured using the collimated lens}

\textcolor{red}{NEED TO BUILD AND MEAS A FRAGMENTED SAMPLE FROM CHAPTER 4}
 

\begin{thebibliography}{99}

\bibitem{MaloneyAMTA2012} J. Maloney, J. Fraley, M. Habib, J. Schultz, K. C. Maloney, ``Focused Beam Measurement of Antenna Gain Patterns'', AMTA, 2012

\bibitem{SchultzBook} John W. Schultz, Focused Beam Methods, 2012.

\bibitem{ReidThesis} David Reid, A full electromagnetic analysis of fresnel zone plate antennas and the application to a free-space focused-beam measurement system, PhD. Thesis, Georgia Tech, Nov. 2.

\end{thebibliography}






%\include{WideScan}

% some other publication
%\include{ModelingAntennas}

\end{document}