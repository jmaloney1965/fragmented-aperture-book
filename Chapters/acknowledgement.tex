\chapter*{Acknowledgements}
Over the years, I have had the pleasure to work with many smart individuals who each contributed to the development of Fragmented Aperture Antennas.

For my whole career, Prof.\ Glenn Smith has always be a major supporter of my research efforts.  Starting with teaching me how to do research during my Ph.D. thesis, to discussing many research issues over coffee during my university lab years, to co-authoring many book chapters with me, to helping invent the Fragmented Aperture, Dr Smith became a treasured collaborator and friend.

In the early years, I had the good fortune to work with one of the brightest minds, Dr. Morris Kesler.  Morris was also one of the co-inventors of the Fragmented Aperture.  Morris also assisted me in writing book chapters on Modeling Periodic Structures that are used in the analysis and design of phased array antennas and other periodic structures.

I had many conversations with Dr. Eric Kuster when stuck on a research topic, and his non-engineer view point was very helpful to me over the years.

I would like to thank Mr. Paul Friederich, Dr. Lon Pringle, Mr. Jim Acree for their dedicated support as project directors for both the research on Fragmented Apertures and for helping advocate for customer solutions based on Fragmented Apertures.

In the later years, the development of Fragmented Aperture solutions for many diverse applications were impacted by the assistance of Mr. Brad Baker, Mr. Kevin Cook, Dr. Doug Denison, Ms. Lynn Fountain, Mr. James Fraley, Mr. David Landgren, and Dr. Todd Lee.

The fabrication of many of the Fragmented Apertures relied heavily on the machining skills of the best machinist I ever met, Mr. Kurt Weismayer.  Without Kurt's skills, many of the antennas certainly would not have been built exactly, and the measured RF performance would have been in worse agreement with the model predictions. 

I also have greatly enjoyed working closely with Dr. John Schultz over the last 10+ years.  Collaborating with John on many research projects has been rewarding.  Specifically, learning from John about the design of metamaterials and collaborating on their use with Fragmented antennas was always a pleasure. 

Lastly, I want to thank Ms. Rebecca Schultz and Dr. Kate Maloney for allowing me to assist Compass Technology Groups research efforts over the last few years.  Also, I would like to thank them for their support in further developing the Fragment Aperture antenna.