\chapter{Improved Approach to Design Fragmented Apertures}
%\authortoc{James G. Maloney}
%\chapterauthor{James G. Maloney}

\section{Overview}

In the late 1990s, Maloney et al.\ began investigating the design of highly pixelated apertures whose physical shape and structure are optimized using genetic algorithms (GA) and full-wave computational electromagnetic simulation (FDTD) to best meet a required antenna performance specification---gain, bandwidth, polarization, pattern, and so on \cite{MaloneyKeslerHarms}--\cite{BalanisHB12}.  Visual inspection of the optimized designs revealed that the metallic pixels formed many connected and disconnected fragments across the aperture, which led to the name \emph{Fragmented Aperture Antennas} for this new class of antennas.  A detailed description of the original design approach is disclosed in the original patent \cite{MaloneyFragPatent}.  Since the original publications, other research groups have successfully applied the fragmented aperture design approach to their own applications \cite{Herscovici}--\cite{EllgardtPersson}, including the use of genetic algorithms for broadband fragmented aperture phased array design \cite{Thors2005Broad-band} and the development of fragmented antennas based on coupled small radiating elements \cite{Barani2018Fragmented}.

However, the original fragmented design approach suffers from two significant deficiencies.  First, the placement of pixels on a generalized rectilinear grid leads to the problem of \emph{diagonal touching}---pixels that share only a corner vertex are electrically connected in the numerical model but may be disconnected when fabricated.  Other research groups have also encountered this diagonal touching problem \cite{EllgardtThesis}.  Second, the convergence of the GA becomes increasingly poor as the pixel count grows large ($\gg 100$).  Recent work has explored various optimization strategies to address convergence challenges, including the use of integrated baluns to simplify the design space \cite{Zang2019Optimum} and automated optimization techniques \cite{Li2019Automated}.

This chapter presents solutions to both of these shortcomings.  First, alternate approaches to the discretization of the aperture area that inherently avoid diagonal touching are presented.  Second, an improved mutation operator tailored for fragmented aperture design that significantly improves convergence for large pixel counts is introduced.

\section{The Diagonal Touching Problem}
\label{sec:diagtouch}

Originally, fragmented aperture antennas were envisioned as a planar surface partitioned into a grid of rectangular pixels, each either conducting or non-conducting, as shown in Figure~\ref{fig:IFAF1}.  A genetic algorithm and a computational electromagnetic model determined which pixels should be conducting and which should be non-conducting to form an antenna surface suitable for a given application.  This concept was generalized to parallelogram-shaped pixels, as shown in Figure~\ref{fig:IFAF2}, in the original fragmented aperture patent \cite{MaloneyFragPatent}.

\begin{figure}
\includegraphics[angle=0,width=\linewidth]{/Users/jim.maloney/Book/images/improvedPatentFig1.png}
\caption{Original fragmented aperture approach based on a lattice of rectangular pixels.  An example of diagonal touching is shown in the top right of the figure.}
\label{fig:IFAF1}
\end{figure}

As discussed in Chapter~2, the rectangular pixel approach was very successfully used to design novel antennas \cite{MaloneyKeslerHarms}--\cite{BalanisHB12}.  However, many of these designs proved troublesome to fabricate and measure.  The primary problem is \emph{diagonal touching} of pixels, as illustrated in the top right of Figures~\ref{fig:IFAF1} and~\ref{fig:IFAF2}.

\begin{figure}
\includegraphics[angle=0,width=\linewidth]{/Users/jim.maloney/Book/images/improvedPatentFig2.png}
\caption{Generalized fragmented aperture approach based on parallelogram-shaped pixels.  Again, an example of diagonal touching is shown in the top right of the figure.}
\label{fig:IFAF2}
\end{figure}

Diagonal touching is not a problem during the design phase because in the FDTD numerical model, diagonally adjacent conducting pixels are always electrically connected.  However, when fabricated using processes such as printed circuit board etching, the diagonal contact is often broken due to over-etching, as illustrated in Figure~\ref{fig:IFAF3}(a).  Disconnecting metal that should be connected can seriously degrade the antenna's impedance match and gain characteristics.

\begin{figure}
\includegraphics[angle=0,width=\linewidth]{/Users/jim.maloney/Book/images/improvedPatentFig3.png}
\caption{(a) Over-etching causing diagonal elements not to touch, as shown in the top center. (b) Close-up photograph of an etched copper fragmented antenna showing over-etch disconnecting diagonal fragments.}
\label{fig:IFAF3}
\end{figure}

Other researchers have also observed the difficulties caused by diagonal touching.  A close-up photograph of an etched copper fragmented surface is shown in Figure~\ref{fig:IFAF3}(b), where the disconnection of diagonally adjacent pixels is clearly visible \cite{Ellgardt2006Characteristics}.

In fact, nearly every antenna design included in the original fragmented aperture patent suffers from the diagonal touching problem.  Figure~\ref{fig:IFAF4} shows a few examples from U.S.\ Patent 6,323,809 \cite{MaloneyFragPatent}, with the most troublesome diagonal-touching locations near the antenna feed circled.

\begin{figure}
\includegraphics[angle=0,width=\linewidth]{/Users/jim.maloney/Book/images/improvedPatentFig4.png}
\caption{Two sample designs from the original fragmented aperture patent exhibiting diagonal touching \cite{MaloneyFragPatent}.  The most troublesome examples of diagonal touching near the antenna feed are circled.}
\label{fig:IFAF4}
\end{figure}

The root cause of the problem lies in the fundamental approach to area partitioning.  When the pixel edges are parallel to the lattice forming vectors, as in Figures~\ref{fig:IFAF1} and~\ref{fig:IFAF2}, the issue of diagonal touching is unavoidable.  The next two sections present solutions: Section~\ref{sec:mitigation} describes several practical mitigation strategies, while Section~\ref{sec:IFAS4} presents three improved pixel geometries that inherently eliminate diagonal touching.

\section{Mitigation Strategies for Diagonal Touching}
\label{sec:mitigation}

Before presenting the improved pixel geometries, it is instructive to review several practical strategies that have been used over the years to mitigate diagonal touching in designs based on the original rectangular pixel approach.  Four such strategies are illustrated in Figure~\ref{fig:IFAF5}.

\begin{figure}
\includegraphics[angle=0,width=\linewidth]{/Users/jim.maloney/Book/images/improvedPatentFig5.png}
\caption{Four approaches to mitigating diagonal touching: (a) super-cell approach using $3\times 3$ plus signs, (b) fabricating each pixel approximately 10\% larger than designed, (c) placing a small metal square at diagonal contact points to ensure connection, and (d) random coin-flip approach to resolve diagonal ambiguity.}
\label{fig:IFAF5}
\end{figure}

\subsection{Super-Cell Approach}

One approach that was successfully exploited is the super-cell strategy shown in Figure~\ref{fig:IFAF5}(a).  A super-cell is a collection of smaller sub-elements---for example, a $3 \times 3$ grid of sub-pixels.  To avoid diagonal touching, the conducting area is defined as the five sub-elements that form a plus sign, leaving the four corner sub-elements always non-conducting.  This guarantees that no diagonal touching can occur.

The super-cell approach successfully produced antennas with good correlation between measurement and model.  However, restricting the conducting area to plus-sign shapes forces electrical currents to flow only in directions aligned with the grid, which over-constrains the design and can lead to suboptimal antenna performance.

\subsection{Oversized Fabrication}

Another successful approach was to intentionally fabricate every pixel approximately 10\% larger than designed, as illustrated in Figure~\ref{fig:IFAF5}(b).  This enlargement ensures that diagonally adjacent pixels overlap slightly, guaranteeing electrical contact.  This strategy was found to yield a high percentage of successfully fabricated antennas.

However, oversized fabrication results in the antenna having roughly 10--20\% more conductor than originally designed, which can alter the antenna characteristics from the design predictions.  It is worth noting that fabricating pixels 10\% \emph{smaller} would guarantee that diagonal pixels never touch, but this would mean that conducting regions could never extend beyond a single pixel---a condition that would render the antenna virtually useless.  Moreover, smaller fabrication would be inconsistent with the FDTD model used during design, in which diagonally adjacent conducting pixels are always connected.

\subsection{Metal Bridge and Coin-Flip Approaches}

Other research groups have developed additional strategies.  Ellgardt and Persson \cite{EllgardtPersson} considered both the oversized pixel strategy of Figure~\ref{fig:IFAF5}(b) and a variant in which a small square of metal is placed at each diagonal contact point to ensure connection, as shown in Figure~\ref{fig:IFAF5}(c).  Rahmat-Samii and colleagues at UCLA have used a random coin-flip process to resolve diagonal ambiguity: when two non-conducting pixels are diagonally adjacent to two conducting pixels, a random selection determines which non-conducting pixel is made conducting to complete the connection, as illustrated in Figure~\ref{fig:IFAF5}(d).

While each of these mitigation strategies can be effective, they all represent workarounds for a fundamental problem.  The proper solution is to change the pixel geometry itself so that diagonal touching cannot occur, as described in the next section.

\section{Three Improved Pixel Geometries}
\label{sec:IFAS4}

The proper approach to eliminating diagonal touching is to break the dependence between pixel edges and lattice directions that is implicit in the rectangular and parallelogram geometries of Figures~\ref{fig:IFAF1} and~\ref{fig:IFAF2}.  The following three subsections present three different pixel geometries that accomplish this, each leading to improved fragmented aperture antennas.  Recent studies have examined important design considerations such as pixel size, symmetry constraints, and their effects on antenna performance \cite{Mair2024Design}.

\subsection{First Approach: Skewed Lattice}

In the first approach, the individual conducting/non-conducting elements are defined using lattice vectors that are not both parallel to the element edges, as illustrated in Figure~\ref{fig:IFAF6}.  The example shown uses square elements arranged on a skewed lattice with a skew angle of $\psi \approx 63.4^\circ$.

\begin{figure}
\includegraphics[angle=0,width=\linewidth]{/Users/jim.maloney/Book/images/improvedPatentFig6.png}
\caption{First approach to improved fragmented aperture antennas: square elements on a skewed lattice.  Skewing the lattice vector $\vec{V}_2$ eliminates the possibility of diagonal touching.}
\label{fig:IFAF6}
\end{figure}

Notice that skewing the lattice vector $\vec{V}_2$ completely eliminates the possibility of diagonal touching.  Every adjacent pixel pair shares a full edge, so the electrical connection between adjacent conducting pixels is always robust and unambiguous.

\subsection{Second Approach: Alternating Pixel Shapes}

In the second approach, two complementary pixel shapes are used that alternate across the aperture, as illustrated in Figure~\ref{fig:IFAF7}.  The shapes are chosen so that adjacent pixels always share a definite edge rather than merely a corner point.

\begin{figure}
\includegraphics[angle=0,width=\linewidth]{/Users/jim.maloney/Book/images/improvedPatentFig7.png}
\caption{Second approach to improved fragmented aperture antennas: alternating pixel shapes that tessellate the plane while ensuring definite edge contact between adjacent pixels.}
\label{fig:IFAF7}
\end{figure}

The key requirement is that the two shapes together tessellate the plane---that is, they tile the aperture surface without gaps or overlaps.  When this condition is met and the shapes are designed so that adjacency always involves a shared edge, diagonal touching is inherently impossible.

\subsection{Third Approach: Single Self-Tessellating Shape}

In the third approach, a single pixel shape is chosen that tessellates the plane by itself while ensuring that adjacent pixels share edges rather than corner points.  Figure~\ref{fig:IFAF8} shows one example of such a shape, although many other shapes satisfying these requirements exist.

\begin{figure}
\includegraphics[angle=0,width=\linewidth]{/Users/jim.maloney/Book/images/improvedPatentFig8.png}
\caption{Third approach to improved fragmented aperture antennas: a single self-tessellating pixel shape that tiles the aperture surface while ensuring definite edge contact between all adjacent pixels.}
\label{fig:IFAF8}
\end{figure}

The advantage of a single-shape tessellation is simplicity of implementation: only one pixel geometry needs to be defined and manufactured.  However, depending on the chosen shape, this approach may not naturally support the left-right and top-bottom symmetry constraints that are commonly imposed for broadside, linearly polarized antenna designs.  This makes the third approach particularly well suited for designs where symmetry is not required, such as beam-steered or circularly polarized antennas.

\section{Improved Mutation Algorithm for Better Convergence}
\label{sec:mutation}

In addition to the diagonal touching problem, the original fragmented aperture design approach suffered from poor convergence of the genetic algorithm when the number of pixels became large.  This section describes an improved mutation operator that addresses this limitation.  More recent work has explored alternative optimization approaches including topology optimization with level set methods \cite{Howard2024Topology}, machine learning and neural network techniques \cite{Li2025Inverse,Wang2024Machine,Jacobs2022Accurate}, and reinforcement learning \cite{Wang2024Bandwidth}; however, genetic algorithms remain popular due to their simplicity and effectiveness for pixelated antenna design \cite{Mair2022Evolutionary,Zeghdoud2025Accelerated}.

In a standard genetic algorithm, mutation is a random process in which a small number of genes are changed each generation to help the algorithm avoid convergence to suboptimal solutions.  For a fragmented aperture antenna, mutation randomly toggles a few pixels between conducting and non-conducting states.  However, many of these random mutations produce only an isolated metal pixel in a non-conducting region or a small hole in an otherwise solid metal area.  Such changes have negligible effect on the antenna performance, and the mutation step is effectively wasted.

The improved mutation algorithm is biased so that mutations preferentially extend the boundaries of existing conducting fragments into empty regions, or enlarge holes within large metal regions.  This is accomplished using an adjacency matrix that describes which pixels are in contact with each other.  By analyzing the local neighborhood of each pixel, the algorithm identifies pixels at the boundaries between conducting and non-conducting regions and preferentially selects these boundary pixels for mutation.

To demonstrate the effectiveness of this adjacency-based mutation strategy, three independent design trials were conducted with the traditional mutation algorithm and three with the improved mutation algorithm.  Figure~\ref{fig:IFAF9} shows the convergence of the fitness score as a function of generation count, averaged over the three trials for each algorithm.  The improved mutation algorithm (green line) converges to a better fitness score in fewer generations than the traditional approach (blue line).

\begin{figure}
\includegraphics[angle=0,width=\linewidth]{/Users/jim.maloney/Book/images/improvedPatentFig9.png}
\caption{Convergence comparison between the traditional mutation algorithm (blue) and the improved adjacency-based mutation algorithm (green), averaged over three independent design trials.  The improved algorithm converges to a better fitness score in fewer generations.}
\label{fig:IFAF9}
\end{figure}

As shown in Table~\ref{tab:IFAT1}, the improved mutation algorithm produced a better result than the traditional algorithm in every one of the three trials.  The best trial with the improved algorithm achieved a fitness score of $-1.684$, compared to $-2.238$ for the best trial with the traditional algorithm---an improvement of more than 0.5~dB.

\begin{table}
\caption{Fitness score comparison across three convergence trials for the traditional and improved mutation algorithms (higher score is better).}
\includegraphics[angle=0,width=\linewidth]{/Users/jim.maloney/Book/images/improvedPatentFig16.png}
\label{tab:IFAT1}
\end{table}

The results in Table~\ref{tab:IFAT1} also illustrate an important practical point: when using an evolutionary algorithm to design an antenna, more than one design trial should always be executed.  As these results show, the variation between trials can exceed 1~dB, so running multiple trials and selecting the best result is essential for obtaining high-quality designs.

\section{Sample Improved Fragmented Aperture Designs}

This section presents sample antenna designs produced using the improved pixel geometries described in Section~\ref{sec:IFAS4}.  All designs use the improved mutation algorithm of Section~\ref{sec:mutation}.

\subsection{First Approach Designs}

The skewed-lattice approach of Figure~\ref{fig:IFAF6} was used to design a series of fragmented aperture antennas spanning from 500~MHz to 2.0~GHz.  The lattice skew angle was chosen to be $\psi = \tan^{-1}(2) \approx 63.4^\circ$, which provides the left-right physical symmetry needed for linearly polarized broadside designs.  The square pixels were 10.8~mm on a side, and the total aperture area was 25.4~cm $\times$ 25.4~cm.  Each antenna was excited at a terminal pair at the center of the aperture with a $100~\Omega$ transmission line.

The aperture designs were performed using a genetic algorithm with FDTD evaluation of each candidate antenna (see Appendix~A for details on FDTD modeling of antennas).  For these designs, the 25.4~cm $\times$ 25.4~cm aperture contained 663 individual pixels.  Enforcing left-right and top-bottom symmetry reduced the number of independent degrees of freedom to 169.  With a single bit representing the state of each pixel (1 = conducting, 0 = non-conducting), this yields a 169-bit genetic code.  Using a population size of 32 antennas, approximately 100 GA generations were required to produce each design.  The fitness function rewarded good impedance match (return loss better than 15~dB) and maximum broadside realized gain.

Four representative aperture designs are shown in Figure~\ref{fig:IFAF10}.  The physical shapes clearly demonstrate that none of the designs suffer from diagonal touching---every connection between adjacent conducting pixels involves a full shared edge.

\begin{figure}
\includegraphics[angle=0,width=\linewidth]{/Users/jim.maloney/Book/images/improvedPatentFig10.png}
\caption{Four sample fragmented aperture designs produced using the skewed-lattice (First Approach) pixel geometry.  None of the designs exhibit diagonal touching.}
\label{fig:IFAF10}
\end{figure}

Figure~\ref{fig:IFAF11} shows the broadside realized gain of each design as a function of frequency.  The gains are compared with the aperture gain limit (black line), which for these ground-plane-free apertures is $2\pi A/\lambda^2$.  All four designs approach the aperture gain limit within their respective design bands.

\begin{figure}
\includegraphics[angle=0,width=\linewidth]{/Users/jim.maloney/Book/images/improvedPatentFig11.png}
\caption{Broadside realized gain for the four skewed-lattice designs shown in Figure~\ref{fig:IFAF10}.  The black line indicates the aperture gain limit $2\pi A/\lambda^2$.}
\label{fig:IFAF11}
\end{figure}

Figure~\ref{fig:IFAF12} shows the VSWR of each design.  The VSWR remains below 1.5 across the respective design bands, consistent with the fitness function requirement of return loss better than 15~dB.

\begin{figure}
\includegraphics[angle=0,width=\linewidth]{/Users/jim.maloney/Book/images/improvedPatentFig12.png}
\caption{VSWR for the four skewed-lattice designs shown in Figure~\ref{fig:IFAF10}.}
\label{fig:IFAF12}
\end{figure}

\subsection{Second Approach Designs}

The alternating-shape approach of Figure~\ref{fig:IFAF7} was also used to design fragmented aperture antennas.  This pixel geometry naturally supports both left-right and top-bottom symmetry when required.  The aperture area was again 25.4~cm $\times$ 25.4~cm, excited at the center with a $100~\Omega$ feed.  The aperture contained 841 shaped pixels, and with both symmetries enforced, the number of independent degrees of freedom was 221.

\textcolor{red}{\textbf{[Note: The Second Approach currently has two sample designs.  Four designs are planned to match the First Approach presentation.  Figures and text to be updated.]}}

Figure~\ref{fig:IFAF13} shows two sample designs for the 0.5--0.8~GHz and 0.8--1.2~GHz bands.  As with the first approach designs, no diagonal touching is present in the physical structures.

\begin{figure}
\includegraphics[angle=0,width=\linewidth]{/Users/jim.maloney/Book/images/improvedPatentFig13.png}
\caption{Two sample fragmented aperture designs produced using the alternating-shape (Second Approach) pixel geometry for the 0.5--0.8~GHz and 0.8--1.2~GHz bands.}
\label{fig:IFAF13}
\end{figure}

Figures~\ref{fig:IFAF14} and~\ref{fig:IFAF15} summarize the broadside gain and VSWR for these two designs.  The performance is consistent with the design objectives, with gain approaching the aperture limit within each design band and VSWR remaining below 1.5.

\begin{figure}
\begin{center}
\includegraphics[angle=0,scale=0.2]{/Users/jim.maloney/Book/images/improvedPatentFig14.png}
\caption{Broadside realized gain for the two alternating-shape designs shown in Figure~\ref{fig:IFAF13}.}
\label{fig:IFAF14}
\end{center}
\end{figure}

\begin{figure}
\begin{center}
\includegraphics[angle=0,scale=0.2]{/Users/jim.maloney/Book/images/improvedPatentFig15.png}
\caption{VSWR for the two alternating-shape designs shown in Figure~\ref{fig:IFAF13}.}
\label{fig:IFAF15}
\end{center}
\end{figure}

\subsection{Third Approach Designs}

The self-tessellating shape approach of Figure~\ref{fig:IFAF8} can also be used to design fragmented aperture antennas.  As noted earlier, for vertically or horizontally polarized elements with a broadside beam, the lack of inherent left-right and top-bottom symmetry in many self-tessellating shapes is a drawback.  However, for applications where the desired beam direction is not broadside or the desired polarization is circular or slant-linear, symmetry constraints are not needed and the third approach is fully competitive with the first and second.

\textcolor{red}{\textbf{[Missing content: Third Approach sample designs are needed.  Planned designs include: (a) broadside vertically polarized, (b) broadside horizontally polarized, (c) broadside slant-linear polarized, and (d) vertically polarized beam steered $45^\circ$ from broadside.  Corresponding gain, VSWR, and azimuth pattern figures are also needed.]}}

%\begin{figure}
%\includegraphics[angle=0,width=\linewidth]{/Users/jim.maloney/Book/images/improvedFragFig16.png}
%\caption{Four sample designs from the self-tessellating shape (Third Approach): (a) broadside vertically polarized, (b) broadside horizontally polarized, (c) broadside slant-linear polarized, (d) vertically polarized beam steered $45^\circ$ from broadside.}
%\label{fig:IFAF16}
%\end{figure}

%\begin{figure}
%\includegraphics[angle=0,width=\linewidth]{/Users/jim.maloney/Book/images/improvedFragFig17.png}
%\caption{Broadside realized gain for the four self-tessellating shape designs.}
%\label{fig:IFAF17}
%\end{figure}

%\begin{figure}
%\includegraphics[angle=0,width=\linewidth]{/Users/jim.maloney/Book/images/improvedFragFig18.png}
%\caption{VSWR for the four self-tessellating shape designs.}
%\label{fig:IFAF18}
%\end{figure}

%\begin{figure}
%\includegraphics[angle=0,width=\linewidth]{/Users/jim.maloney/Book/images/improvedFragFig19.png}
%\caption{Azimuth gain pattern at midband for the four self-tessellating shape designs.}
%\label{fig:IFAF19}
%\end{figure}


\FloatBarrier

% Bibliography for Chapter 3
% Uses chapter-specific .bib files organized by topic
\bibliography{../Literature/master_bibliography,%
              ../Literature/fragmented_aperture_core,%
              ../Literature/ml_ai_optimization_methods}
\bibliographystyle{IEEEtran}
