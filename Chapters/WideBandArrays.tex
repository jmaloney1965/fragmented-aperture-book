\chapter{Wideband Antenna Arrays}

\section{Introduction}

Students of antenna design are taught that the gain of an array antenna can be estimated by multiplying the pattern of a single element by the array factor.  This approach ignores mutual coupling between elements, which has traditionally been a major challenge for designers of phased arrays.  Mutual coupling introduces areas of scan blindness---combinations of frequency and scan angle for which the array is poorly matched---that can severely degrade array performance.  As the sophistication of numerical modeling codes has increased in concert with the availability of inexpensive parallel computing power, antenna designers have developed the ability to include the effects of mutual coupling in their performance predictions.  This capability, in turn, suggests the possibility of \emph{exploiting} mutual coupling rather than merely avoiding it.

To appreciate why wideband arrays are challenging, consider the performance comparison in Figure~\ref{fig:WBA1}.  A $10'' \times 10''$ aperture is populated with three different antenna types: a uniform current sheet (the theoretical ideal), a spiral antenna, and a bowtie antenna.  The uniform current sheet achieves the full aperture gain limit across frequency, but it is a theoretical construct, not a realizable antenna.  The spiral produces useful gain only over a limited bandwidth, and the bowtie exhibits narrow resonant peaks.  Neither conventional design approaches the broadband performance that the aperture could support.

\begin{figure}
\begin{center}
\includegraphics[angle=0,width=\linewidth]{/Users/jim.maloney/Book/images/BalanisHandbookFig12-42.png}
\caption{Predicted gain for three antenna types occupying a $10'' \times 10''$ aperture: a uniform current sheet (theoretical ideal), a spiral antenna, and a bowtie antenna.  Neither the spiral nor the bowtie approaches the broadband aperture gain limit \cite{BalanisHB12}.}
\label{fig:WBA1}
\end{center}
\end{figure}

The fragmented aperture design approach, applied to array elements, offers a path to achieving the broadband performance that conventional element designs cannot.  By allowing a genetic algorithm to optimize the conducting pattern of each unit cell---including connections between adjacent elements---the design process can exploit mutual coupling to produce arrays with bandwidths far exceeding those of conventional designs.

This chapter traces the development of wideband fragmented aperture arrays, beginning with the discovery that electrical connections between elements are essential for wide bandwidth.  It then describes the broadband screen backplane, a key innovation that mitigates the half-wave nulls introduced by a ground plane, enabling practical planar arrays with bandwidths of 10:1 and beyond.  The chapter continues with multi-layer radiator designs that use parasitic face sheets to enhance front-to-back ratio, further extending achievable bandwidths to 33:1.  Measured results from laboratory proof-of-concept arrays are presented throughout.

\section{Connected Fragmented Array Elements}

The fragmented aperture design approach extended naturally from single-element antennas to array elements.  The key insight that led to a breakthrough in achievable bandwidths was the recognition that DC electrical connection between adjacent elements was not merely tolerable but actually beneficial and should be exploited.  Subsequent multiple-octave array designs consistently featured these inter-element connections, which support continuous current paths spanning multiple elements.

The importance of connected arrays can be understood intuitively.  In an array with an 8:1 bandwidth, the radiated wavelength changes from approximately two element widths at the highest frequency to 16 element widths at the lowest.  For the array to radiate efficiently at the lowest frequencies, continuous conducting paths of sufficient length must be present on the aperture surface.  With a connected array, such paths naturally exist.

To demonstrate the importance of inter-element connections, the 6-cm elements shown in Figure~\ref{fig:WBA3} were designed to operate from 0.25--2.5~GHz in an array with no ground plane.  Two designs were compared: the first was optimized with electrical connections between elements permitted (a connected array), while the second was optimized with a boundary enforced around each element to prevent conducting pathways between elements.  The realized gain achieved by an $8 \times 8$ finite array of each element design is shown in Figure~\ref{fig:WBA3}(c).  Because of the continuous current paths across element boundaries, the connected design maintains a good impedance match over the full 10:1 bandwidth and achieves markedly superior performance.

\begin{figure}
\begin{center}
\includegraphics[angle=0,width=\linewidth]{/Users/jim.maloney/Book/images/BalanisHandbookFig12-44.png}
\caption{Design experiment comparing two $8 \times 8$ arrays: (a) connected array element, (b) unconnected array element, (c) embedded element gain comparison for a central element.  The connected array element far outperforms the unconnected element \cite{BalanisHB12}.}
\label{fig:WBA3}
\end{center}
\end{figure}

Another key feature of the connected geometry is that the overall size of the array becomes a limiting factor on the lowest operating frequency.  When the connected element design of Figure~\ref{fig:WBA3} was modeled in arrays of various sizes (again without a ground plane), the low-frequency performance scaled proportionally with array size, as shown in Figure~\ref{fig:WBA4}.  Arrays of $2 \times 2$, $4 \times 4$, $8 \times 8$, and $16 \times 16$ elements were simulated.  In all cases, the upper frequency limit remained relatively constant, being limited by the element lattice spacing and the onset of grating lobes.  The low-frequency limit, on the other hand, was approximately proportional to the overall array dimension.

\begin{figure}
\begin{center}
\includegraphics[angle=0,width=\linewidth]{/Users/jim.maloney/Book/images/BalanisHandbookFig12-45.png}
\caption{The connected element from Figure~\ref{fig:WBA3} simulated in arrays of various sizes.  The low-frequency performance limit is approximately proportional to overall array size \cite{BalanisHB12}.}
\label{fig:WBA4}
\end{center}
\end{figure}

To confirm the validity of these simulation predictions, a fragmented array with 3-cm elements was designed and measured in 1999.  The metric used was the embedded element realized gain (EERG), obtained by driving one central element while terminating all surrounding elements in matched resistive loads.  The EERG measures the performance of a single element in the array environment and is a powerful diagnostic because angle-pattern cuts of the EERG can be used to predict the scan performance of a fully driven array.  This approach greatly reduces measurement cost since beam-forming electronics are not required.  As shown in Figure~\ref{fig:WBA5}, the array achieves near aperture-limited gain at broadside over a 10:1 bandwidth, with excellent model-measurement agreement.

\begin{figure}
\begin{center}
\includegraphics[angle=0,width=\linewidth]{/Users/jim.maloney/Book/images/BalanisHandbookFig12-46.png}
\caption{Embedded element realized gain (EERG) for a central element of a $10 \times 17$ array with 3-cm square unit cells, demonstrating 10:1 bandwidth with excellent model-measurement agreement \cite{BalanisHB12}.}
\label{fig:WBA5}
\end{center}
\end{figure}

\section{Wideband Backplanes: Planar 10:1 Arrays}

Early explorations of fragmented arrays (2000 and earlier) focused on fundamental questions of element connections, bandwidth limits, and natural impedance values \cite{MaloneyKeslerHarms}--\cite{FriederichPringle}.  These investigations typically either used no ground plane behind the radiating surface or accepted the limitations of simple ground planes.  More recent work has demonstrated unbalanced feed designs for wideband phased arrays \cite{Landgren2017Unbalanced,Dykes2017Wideband} and broadband arrays with power combiners based on the fragmented aperture principle \cite{Landgren2019Broadband}.

Ideally, a ground plane should be located $\lambda/4$ behind the radiating surface of a planar antenna.  The backward-radiated energy travels a round-trip path of $\lambda/2$, accumulating $180^\circ$ of phase, and the $180^\circ$ phase inversion at the perfect electric conductor (PEC) surface of the ground plane causes the reflected energy to arrive in phase with the forward-going radiation.  Wideband antennas pose a fundamental difficulty, however, since $\lambda$ varies widely over the operating bandwidth.  When the ground plane is $\lambda/2$ behind the radiating surface (or an integer multiple of $\lambda/2$), the backward-going radiation is reflected and arrives exactly out of phase with the forward radiation, producing a deep null in the gain.

This situation is illustrated in Figure~\ref{fig:WBA6}, which shows the results of a simulation of a fragmented aperture radiator placed 2.5~cm in front of a PEC ground plane.  The broadside gain is normalized to the area gain.  Without the ground plane, the radiator is well matched across the band, but because it radiates in both directions, the forward radiation only approaches $-3$~dB, represented by the dashed line.  With the ground plane, the gain approaches the maximum around 3~GHz, where 2.5~cm represents a quarter of the free-space wavelength and the ground plane provides nearly 3~dB of gain enhancement.  At 6~GHz, however, the ground plane is a half wavelength behind the radiating surface, and a deep null appears.  This null repeats at every integer multiple of $\lambda/2$ (12~GHz, 18~GHz, etc.).  Practical experience indicates that fragmented aperture designs can be extended to approximately 8:1 bandwidths before the half-wave null must be addressed.

\begin{figure}
\begin{center}
\includegraphics[angle=0,scale=0.4]{/Users/jim.maloney/Book/images/BalanisHandbookFig12-47.png}
\caption{When a broadband radiating sheet is placed in front of a simple PEC ground plane, the resulting gain suffers nulls at frequencies where the separation distance is an integer multiple of $\lambda/2$ (in this case, 6~GHz for a 2.5-cm separation) \cite{BalanisHB12}.}
\label{fig:WBA6}
\end{center}
\end{figure}

The problem is further complicated if the array is intended to scan over a significant volume, because the null frequency depends on the scan angle, as illustrated in Figure~\ref{fig:WBA7}.  As the scan angle moves away from broadside, the null frequency increases.  The contour plot in the figure shows this trend clearly.

\begin{figure}
\begin{center}
\includegraphics[angle=0,width=\linewidth]{/Users/jim.maloney/Book/images/BalanisHandbookFig12-48.png}
\caption{The half-wave null frequency depends on scan angle.  Left: geometry illustration.  Right: contour plot showing the relationship between field intensity at the radiating surface, frequency, and scan angle for a 2.5-cm separation \cite{BalanisHB12}.}
\label{fig:WBA7}
\end{center}
\end{figure}

\subsection{The Broadband Screen Backplane}

Since the problem can be attributed to backward-radiated energy reflecting off the ground plane, one is tempted to address it with absorbing solutions.  Interestingly, the Salisbury screen absorbing structure has the desirable characteristic that its tuned absorption frequency increases with incidence angle, exactly analogous to the scan angle--frequency dependence of the half-wave null.  This was the inspiration for the \emph{broadband screen backplane}, which was developed to extend the frequency performance of fragmented apertures over a ground plane.

The broadband screen backplane consists of one or more resistive card (r-card) layers placed between the radiating aperture and the PEC ground plane.  Figure~\ref{fig:WBA8} shows the performance of a typical r-card layer placed halfway between the aperture layer and the ground plane.  The backplane is most absorptive at exactly the frequency/angle combinations where the half-wave null occurs (and at every odd multiple of half wavelengths).

\begin{figure}
\begin{center}
\includegraphics[angle=0,width=\linewidth]{/Users/jim.maloney/Book/images/BalanisHandbookFig12-49.png}
\caption{A $377~\Omega$/square r-card layer placed halfway between the radiating surface and the PEC ground plane (2.5-cm separation) eliminates the deep null at $\lambda/2$ \cite{BalanisHB12}.}
\label{fig:WBA8}
\end{center}
\end{figure}

Figure~\ref{fig:WBA9} shows the normalized realized gain at broadside with this first-generation broadband screen backplane.  The aperture has recovered enough gain at the problem frequency to achieve near 50\% efficiency.  However, one can do better.  For overall antenna performance, the impedance value, position, and even the number of r-card layers may be treated as free variables in the design.  For example, if the $377~\Omega$ r-card is replaced with a $225~\Omega$ card, the realized gain is maintained within 2~dB of the aperture limit across the operating band.

\begin{figure}
\begin{center}
\includegraphics[angle=0,scale=0.4]{/Users/jim.maloney/Book/images/BalanisHandbookFig12-50.png}
\caption{Normalized realized gain at broadside for the configuration of Figure~\ref{fig:WBA8}, showing that the deep null at 6~GHz has been improved to approximately 3~dB insertion loss \cite{BalanisHB12}.}
\label{fig:WBA9}
\end{center}
\end{figure}

\subsection{A 10:1 Proof-of-Concept Array}

Figure~\ref{fig:WBA10} shows a 10:1 bandwidth design that was developed as a proof of concept using a single r-card broadband screen.  The array demonstrated better than 50\% efficiency over the operating band of 1--10~GHz.  The plot on the right shows normalized predictions of realized gain, or equivalently, insertion loss.  The upper curve (normalized gain) shows the effects of resistive loss alone, while the lower curve (normalized realized gain) shows the combined effects of resistive and mismatch loss.  The separation between the two curves is a measure of the impedance match quality.

\begin{figure}
\begin{center}
\includegraphics[angle=0,width=\linewidth]{/Users/jim.maloney/Book/images/BalanisHandbookFig12-51.png}
\caption{The first array built using the broadband screen backplane: a 10:1 design with efficiency better than 50\% ($< 3$~dB insertion loss) from 1 to 10~GHz \cite{BalanisHB12}.}
\label{fig:WBA10}
\end{center}
\end{figure}

\subsection{Multi-Layer Broadband Screen Backplanes}

With a simple conducting ground plane behind a planar radiating surface, a standing wave occurs when the separation distance is $\lambda/2$ (or an integer number of half wavelengths), placing a field null at the radiating surface.  The resulting impedance mismatch causes the deep dropout in the gain curve.  In addition to the energy they dissipate, r-cards inserted in the backplane stack introduce additional reflection boundaries that redistribute the standing wave to avoid field cancellation at the radiating surface.

As the operating bandwidth of the array spans more octaves, a simple ground plane introduces more half-wave nulls, and the problem of defeating the standing wave becomes more complex.  Figure~\ref{fig:WBA11} shows an example of a radiating surface located three inches in front of a simple conducting ground plane.  The resulting standing wave produces interference nulls approximately every 2~GHz.  When the empty cavity is replaced by an optimized broadband screen with six r-card layers, the standing-wave nulls are eliminated.  Figure~\ref{fig:WBA12} compares the performance of the empty cavity with the broadband screen over frequency and scan angle, demonstrating that the nulls are effectively controlled to scan angles of $60^\circ$ or more.

\begin{figure}
\begin{center}
\includegraphics[angle=0,width=\linewidth]{/Users/jim.maloney/Book/images/BalanisHandbookFig12-52.png}
\caption{Aperture fields 3 inches in front of a PEC surface, with and without a broadband screen backplane \cite{BalanisHB12}.}
\label{fig:WBA11}
\end{center}
\end{figure}

\begin{figure}
\begin{center}
\includegraphics[angle=0,width=\linewidth]{/Users/jim.maloney/Book/images/BalanisHandbookFig12-53.png}
\caption{Contour plots comparing the configurations of Figure~\ref{fig:WBA11} over a range of scan angles.  The broadband screen effectively eliminates standing-wave nulls to scan angles beyond $60^\circ$ \cite{BalanisHB12}.}
\label{fig:WBA12}
\end{center}
\end{figure}

\section{Multi-Layer Radiators: 33:1 Bandwidth Arrays}

A broadband screen backplane can control half-wave nulls, but it uses a loss mechanism to do so.  While it is not necessary to attenuate all of the backward-radiated energy, some resistive loss is inevitable with this approach.  A more desirable strategy would be to radiate energy preferentially into the forward hemisphere, thereby reducing the amount of backward-radiated energy that can reflect off the ground plane.

\subsection{Directional Radiation from Thick Apertures}

As a thought experiment, consider the ideal planar radiator with no thickness shown on the left in Figure~\ref{fig:WBA13}.  Radiation must occur equally into both hemispheres since nothing distinguishes one side from the other.  However, if the radiating layer has some thickness, asymmetries may be introduced that cause the surface to radiate preferentially in one direction, as shown on the right in Figure~\ref{fig:WBA13}.  For example, if 90\% of the energy is directed into the forward hemisphere, then even if the backward-radiated 10\% is reflected and returns $180^\circ$ out of phase, it will only reduce the transmitted power to 80\% of the maximum value.

\begin{figure}
\begin{center}
\includegraphics[angle=0,width=\linewidth]{/Users/jim.maloney/Book/images/BalanisHandbookFig12-54.png}
\caption{Thought experiment demonstrating the benefit of preferential forward radiation.  A radiator with finite thickness can achieve asymmetric radiation (front-to-back ratio), mitigating the impact of ground plane reflections \cite{BalanisHB12}.}
\label{fig:WBA13}
\end{center}
\end{figure}

This principle can be exploited using multiple radiating layers in front of the ground plane.  The additional layers may be actively driven or parasitic, analogous to the directors in a Yagi-Uda antenna.

\subsection{Parasitic Layer Design Experiments}

In Figure~\ref{fig:WBA14}, two radiating layers approximately 8~mm apart with no ground plane were optimized using the fragmented aperture design process.  The design goal was to maximize gain in the forward hemisphere.  The plot shows the normalized forward-going and backward-going radiation, demonstrating good front-to-back ratio (F/B) over the upper octave of the design region (1--10~GHz).  As the wavelength gets longer, the electrical separation between radiating layers becomes insufficient to direct the radiation.

\begin{figure}
\begin{center}
\includegraphics[angle=0,width=\linewidth]{/Users/jim.maloney/Book/images/BalanisHandbookFig12-55.png}
\caption{Idealized design with two simultaneously optimized radiating layers and no ground plane.  The design goal was to maximize front-to-back ratio \cite{BalanisHB12}.}
\label{fig:WBA14}
\end{center}
\end{figure}

The region of effectiveness for the parasitic-layer approach is sufficient to produce the 10:1 design of Figure~\ref{fig:WBA15}, where the realized gain remains within 3~dB of the maximum across the band.  This design is for illustrative purposes, as it does not include realistic feed structures or material loss.

\begin{figure}
\begin{center}
\includegraphics[angle=0,scale=0.4]{/Users/jim.maloney/Book/images/BalanisHandbookFig12-56.png}
\caption{Parasitic layers directing radiation forward in the presence of a PEC backplane, keeping insertion loss below 2~dB over most of a 10:1 bandwidth \cite{BalanisHB12}.}
\label{fig:WBA15}
\end{center}
\end{figure}

The extent to which multiple radiating face-sheet layers improve the bandwidth depends on the number and spacing of the face sheets.  Figure~\ref{fig:WBA16} shows designs using two and three face sheets, respectively.  These simulations include realistic feed structures, but the ground planes have been replaced by perfectly matched absorbing layers.  With two face sheets, the antenna exhibits enhanced gain over most of the upper octave.  With three, antenna gain is enhanced over the upper two octaves.

\begin{figure}
\begin{center}
\includegraphics[angle=0,width=\linewidth]{/Users/jim.maloney/Book/images/BalanisHandbookFig12-57.png}
\caption{Design experiments with two and three radiating face sheets.  The ground plane has been replaced in each simulation with a perfectly absorbing back boundary.  Normalized gain levels above $-3$~dB indicate front-to-back ratio enhancement \cite{BalanisHB12}.}
\label{fig:WBA16}
\end{center}
\end{figure}

\subsection{33:1 Proof-of-Concept Arrays}

Design of fragmented elements for phased arrays with operating bandwidths beyond 10:1 requires a judicious combination of a multi-layer radiator with a broadband screen backplane.  In partnership with Northrop Grumman Electronics Systems, the Georgia Tech Research Institute (GTRI) built and measured two laboratory proof-of-concept radiators with 33:1 bandwidths, each incorporating both design strategies.  Each design consisted of a three-layer radiator stack over a six-r-card backplane stack.  In the first design, two face sheets were driven by the feeds and the third was parasitic.  To simplify the manufacturing process, the second design had only the innermost face sheet driven, with two parasitic outer layers.

Figure~\ref{fig:WBA17} shows the performance of the second design in a periodic simulation, which eliminates finite-array edge effects.  The simulation includes realistic feed structures and material properties.  Gain is normalized to the element area gain, so the 0~dB line represents ideal performance.  The design achieves approximately 1~dB or better insertion loss over the upper octave, with better than 3~dB insertion loss over the entire 0.3--10~GHz design bandwidth.

\begin{figure}
\begin{center}
\includegraphics[angle=0,width=\linewidth]{/Users/jim.maloney/Book/images/BalanisHandbookFig12-58.png}
\caption{Predicted performance of the 33:1 antenna design in a periodic simulation, including realistic feed structures.  Antenna efficiency is better than 50\% over the entire 0.3--10~GHz bandwidth \cite{BalanisHB12}.}
\label{fig:WBA17}
\end{center}
\end{figure}

\subsection{33:1 Measured Results}

The simulations were validated with measurements of a test piece on three different antenna ranges covering the entire operating bandwidth.  The test antenna was a $23 \times 23$ element, dual-linear-polarized array with the center element actively driven and all surrounding elements terminated in matched $188~\Omega$ impedances at the feed points.  Figure~\ref{fig:WBA18} illustrates the composition of one element of the array in cross section, with diagrams of the etched unit-cell pattern on each face sheet and a photograph of the test antenna.

\begin{figure}
\begin{center}
\includegraphics[angle=0,width=\linewidth]{/Users/jim.maloney/Book/images/BalanisHandbookFig12-59.png}
\caption{Construction details of the 33:1 antenna design, including a photograph of the $23 \times 23$ element test piece used to measure embedded element realized gain \cite{BalanisHB12}.}
\label{fig:WBA18}
\end{center}
\end{figure}

Broadside frequency scans of the EERG are plotted in Figure~\ref{fig:WBA19}.  The figure compares measurement results at broadside to predictions.  The element area gain (dashed line) represents the physical limit for antenna performance.  The predicted EERG at broadside is shown as a solid line.  Four measured data sets from three different antenna ranges are compared, including three different calibration horns spanning the 33:1 bandwidth.  Data was measured on both V-pol and H-pol feeds (both should be equivalent at broadside for this symmetric design).  The measured data show excellent consistency across ranges and at both sets of feed points, and excellent agreement with the predictions.

\begin{figure}
\begin{center}
\includegraphics[angle=0,scale=0.4]{/Users/jim.maloney/Book/images/BalanisHandbookFig12-60.png}
\caption{Compilation of measured EERG data at broadside for the 33:1 test antenna (three antenna ranges, two polarizations) compared with numerical predictions.  The dashed line represents the element area gain \cite{BalanisHB12}.}
\label{fig:WBA19}
\end{center}
\end{figure}

The measurements showed approximately 1~dB more insertion loss at the high end than predicted.  The difference exceeds what can be attributed to resistive loss in the feed cables and on the metal radiating surfaces; it is likely due to slight imperfections in the assembly of the radiator.  Performance at the high end is particularly sensitive to the position and planarity of the three face-sheet layers, and the planarity in the assembled test piece was affected by warping in the etched sheets.

This slight high-frequency offset is removed in Figure~\ref{fig:WBA20} to facilitate angle-pattern comparisons.  These patterns allow detailed comparison of measured and modeled EERG over angle cuts at several discrete frequencies.  Model-measurement agreement is excellent, with the models predicting features such as the ripple at 2~GHz due to finite-array edge effects and the narrowing of the scan volume above 8~GHz.

\begin{figure}
\begin{center}
\includegraphics[angle=0,width=\linewidth]{/Users/jim.maloney/Book/images/BalanisHandbookFig12-61.png}
\caption{Comparison of modeled (solid lines) and measured (data markers) EERG pattern cuts at several discrete frequencies, showing excellent agreement \cite{BalanisHB12}.}
\label{fig:WBA20}
\end{center}
\end{figure}

Figure~\ref{fig:WBA21} presents the measured EERG data from three overlapping data sets for H-plane scans and two for E-plane scans.  The contour plots show angle cuts plotted horizontally at each frequency.  Each angle cut has been normalized so that the maximum value at each frequency is zero (i.e., the frequency slope has been removed).  The resulting image shows the achievable scan volume as a function of frequency for a fully driven array with this element design.  The scan volume, as defined by the $-3$~dB points, is approximately $\pm 60^\circ$ over most of the band, with some narrowing above 9~GHz in the H-plane and above 7~GHz in the E-plane.  Notably, the scan volume shows no evidence of scan blindness anywhere in the operating bandwidth.

\begin{figure}
\begin{center}
\includegraphics[angle=0,width=\linewidth]{/Users/jim.maloney/Book/images/5996889-fig-2-hires.png}
\caption{Compilation of measured EERG angle cuts normalized to the maximum at each frequency.  The resulting contours indicate achievable scan volume.  No scan blindness is observed within the operating bandwidth \cite{PringleEtAl}.}
\label{fig:WBA21}
\end{center}
\end{figure}

\section{Design Rules and Scaling}

Experience with several wideband phased array designs has produced empirical evidence for a useful rule of thumb regarding the thickness of these wideband radiators.  Figure~\ref{fig:WBA22} compiles results for five fragmented array designs with bandwidths greater than an octave.  Designs with bandwidths less than 10:1 used simple ground planes; antennas with bandwidths of 10:1 or greater incorporated broadband screen backplanes.  In each case, the overall thickness is dictated not by the bandwidth but by the lowest operating frequency.  For cavities filled with air or low-dielectric foams, the antenna thickness is approximately $\lambda/12$ at the lowest operating frequency.

\begin{figure}
\begin{center}
\includegraphics[angle=0,width=\linewidth]{/Users/jim.maloney/Book/images/5996889-fig-11-hires.png}
\caption{Results of several design exercises for fragmented arrays.  For air-filled cavities, the antenna thickness is approximately $\lambda/12$ at the lowest operating frequency \cite{PringleEtAl}.}
\label{fig:WBA22}
\end{center}
\end{figure}

\section{Summary and Conclusions}

The successful design of ultra-wideband phased arrays has been enabled by several factors.  These designs require high-fidelity time-domain electromagnetic solvers.  The necessity to optimize over many frequencies simultaneously would be prohibitively time-consuming with frequency-domain codes.  All designs described in this chapter were developed using a finite-difference time-domain (FDTD) code (see Appendix~A), the results of which are Fourier transformed to produce the requisite range of frequency predictions.  In early design stages, it is important to move through iterations rapidly to converge on a good initial design.  In the final stages, accurate predictions are required to account for realistic details such as feed structures and material characteristics.

With the appropriate modeling tools and computing infrastructure, the essential features for ultra-wideband planar phased arrays were developed:

\begin{enumerate}
\item \textbf{Connected arrays} to span several octaves by supporting continuous current paths across element boundaries.
\item \textbf{Broadband screen backplanes} consisting of r-card layers to mitigate the half-wave nulls introduced by a conducting ground plane.
\item \textbf{Multi-layer radiators} using parasitic face sheets to enhance front-to-back ratio, leveraging the backplane improvements at lower frequencies.
\item \textbf{Fragmented aperture element design} to accomplish impedance matching in the presence of feed structures, material substrates, multiple layers, and mutual coupling.
\end{enumerate}

Measured results confirm the success of these design strategies for bandwidths up to 33:1.  Preliminary work suggests that phased array operation over bandwidths of 100:1 or more may be achievable.

\textcolor{red}{\textbf{[Note: The bullet-point outline for this chapter included a section on 100:1 bandwidth designs.  This material is not yet available and should be added when ready.]}}

\section{Acknowledgement}
The author would like to thank Mr.\ Paul Friederich for his efforts in converting the original ``How to make a 33:1 bandwidth array antenna'' presentation into the handbook chapter \cite{BalanisHB12} that served as the source material for this chapter.

\FloatBarrier

% Bibliography for Chapter 7
% Uses chapter-specific .bib files organized by topic
\bibliography{../Literature/master_bibliography,%
              ../Literature/fragmented_aperture_core}
\bibliographystyle{IEEEtran}
